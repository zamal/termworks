\newpage

\begin{center}
  \Large{\textbf{ВВЕДЕНИЕ}}
\end{center}
\addcontentsline{toc}{section}{Введение}

Высокие требования к достоверности передачи информации в современных
телекоммуникационных системах диктуют необходимость разработки и
совершенствования методов кодирования дискретных сообщений,
обеспечивающих обнаружение и исправление ошибок, возникающих в канале
связи. Принципиальная возможность решения этой задачи была обоснована
американским учёным Клодом Шенноном более пятидесяти лет тому назад,
который однако не указал конкретных способов построения таких
кодов. Поэтому с тех пор и по настоящее время интенсивно развивается
теория помехоустойчивого кодирования, предметом которой является поиск
алгоритмов кодирования и декодирования, обеспечивающих максимальную
корректирующую способность в каналах с различными свойствами при
минимально возможной избыточности и реализационных
затратах~\cite{{Sultanov}}.

Данная курсовая работа посвящена исследованию корректирующих кодов~---
кодов, служащих для обнаружения или исправления ошибок, возникающих при
передаче информации под влиянием помех, а также при её хранении.

Курсовая работа состоит из 3 разделов.

В первом разделе исследуется код Хемминга, для него найден порождающий
многочлен, сформирована разрешённая кодовая комбинация кода, в
искажённом сообщении исправлена ошибка.

Во втором разделе рассматривается код БЧХ, для него найден порождающий
многочлен, сформирована разрешённая кодовая комбинация, построен
регистр кодирующего устройства для кода БЧХ.

Третий раздел посвящён коду Рида"--~Соломона, построены таблицы
представления, сложения и умножения элементов в поле $GF(16)$, найден
порождающий многочлен, определены синдромы для принятой комбинации.

Для вычислений в конечных полях использовался язык высокого уровня
\textit{GNU Octave}, предназначенный, в основном, для численных
расчетов, и по сути являющийся альтернативой коммерческому
\textit{MatLab}. Пакет может работать в режиме сценариев, интерактивно
или посредством привязки к языку \textit{C/C++}. В \textit{Octave}
реализован богатый язык программирования, обладающий очень большой
библиотекой математических функций, в том числе специализированных
функций обработки сигналов, изображений, звука и т.~п.

В приложениях приведены регистр кодирующего устройства, таблица,
иллюстрирующую состояние ячеек в процессе работы регистра, программа
для нахождения этих состояний, расчёты в программе \textit{GNU
  Octave}.

\newpage


%%% Local Variables: 
%%% mode: latex
%%% TeX-master: "../TermWork_PDS"
%%% End: 
