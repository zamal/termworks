\section{Код Хемминга}
\label{sec:hemming}

\subsection{Условие задачи}

Определить порождающий многочлен $g(x)$ кода Хемминга, скорость
которого $R \geqslant r_0$, рассматривая его как код БЧХ, исправляющий
одиночные ошибки. Сформировать разрешённую комбинацию систематического
кода, соответствующую заданной информационной комбинации $a(x) =
10011000111$. Исправить ошибку в принимаемой кодовой комбинации
$V'(x) = 111110001000010$.

\subsection{Решение задачи}

В данном случае код Хемминга имеет размерность (15,11), т.~к. по
условию скорость $R \geqslant r_0$. При $k = 11$ и $n = 15$, получаем,
что $R = \frac{k}{n} = 0{,}73$, что превышает значение $r_0$, равное
$0{,}7$.

Так как код Хемминга предложено рассматривать как код БЧХ, то первым
этапом при расчёте кода будет определение порождающего многочлена
$g(x)$. Порождающий многочлен для кода БЧХ определяется из выражения:

\begin{equation*}
  g(x) = \text{НОК}\left[f_1(x), f_2(x) \ldots f_{2t_u} \right],
\end{equation*}

\begin{ESKDexplanation}
\item[где ] НОК "--- наименьшее общее кратное;
\item $f_1(x), f_2(x), \ldots$ "--- минимальные многочлены корней
  $\alpha^1, \alpha^2 \ldots$ порождающего многочлена.
\end{ESKDexplanation}

Корнем многочлена $g(x)$ называется число (элемент поля), при
подстановке которого в выражение многочлена вместо $x$, многочлен
обращается в~0. Минимальный многочлен элемента $\beta$ поля $GF(q^m)$
определяется из выражения:

\begin{equation*}
  f(x) = (x - \beta^{g^0})(x - \beta^{g^1}) \ldots (x - \beta^{g^{l-1}}),
\end{equation*}

\begin{ESKDexplanation}
\item[где ] $l$ "--- наименьшее целое число, при котором:
\end{ESKDexplanation}

\begin{equation*}
  \beta^{g^0} = \beta^{g^l}.
\end{equation*}

На практике для определения значения порождающего многочлена можно
воспользоваться таблицей минимальных неприводимых многочленов в поле
$GF(2^m)$~\cite{Kuznetsov}.

Для определения порождающего многочлена необходимо, во-первых, по
заданной длине кода $n$ определить из выражения $n = 2^m - 1$ значение
параметра $m$, который является степенью сомножителя $g(x)$. Затем из
выражения $j = 2t_u - 1$ определяем максимальный порядок
минимального многочлена, входящих в число сомножителей $g(x)$. После
этого, пользуясь таблицей минимальных многочленов, определяем
выражение для $g(x)$, зависимости от найденных $m$ и $j$. Для этого из
колонки соответствующей параметру $m$ выбираются многочлены с номерами
от $1$ до $j$, которые в результате перемножения дают выражение для
$g(x)$. В нашем случае $n = 15$, а $t_u = 1$, следовательно
пользуясь описанной выше методикой, получаем $m = 4$, $j = 1$, откуда
получаем $g(x)=010011$, или в виде многочлена $g(x)=x^4 + x + 1$.

Следующим этапом построения является построение производящей матрицы
$G_{(n,k)}$. Для систематического кода матрица $G_{(n, k)}$ имеет вид:

\begin{equation*}
  G_{(n,k)} = \left[I_k R_{(k, r)} \right],
\end{equation*}

\begin{ESKDexplanation}
\item[где ] $I_k$ "--- единичная матрица размером $k \times k$.
\end{ESKDexplanation}

Строки матрицы $R_{(k, r)}$ определяются из выражений:

\begin{equation}
\label{r_i(x)}
r_i(x) = R_{g(x)}\bigl(a_i(x) \times x^r \bigr),
\end{equation}

или

\begin{equation*}
r_i(x) = R_{g(x)}\left(x^{n-i} \right),
\end{equation*}

\begin{ESKDexplanation}
\item[где ] $a_i(x)$~--- полином, соответствующий $i$-той строке
  матрицы $I_k$;
\item $i$~--- номер строки матрицы $R_{(k, r)}$;
\item $R_{g(x)}\bigl(a(x)\bigr)$~--- остаток от деления $a(x)$ на $g(x)$.
\end{ESKDexplanation}

Выполнив деление многочленов, согласно формуле~(\ref{r_i(x)}),
получаем:

\begin{equation*}
  R_{(15, 11)} =  \left[
    \begin{array}{cccc}
      1 & 0 & 0 & 1 \\
      1 & 1 & 0 & 1 \\
      1 & 1 & 1 & 1 \\
      1 & 1 & 1 & 0 \\
      0 & 1 & 1 & 1 \\
      1 & 0 & 1 & 0 \\
      0 & 1 & 0 & 1 \\
      1 & 0 & 1 & 1 \\
      1 & 1 & 0 & 0 \\
      0 & 1 & 1 & 0 \\
      0 & 0 & 1 & 1  
    \end{array}
  \right].
\end{equation*}

Для нахождения остатков от деления использовалась программа
\textit{Octave} (см. Приложение~\ref{sec:octave},
стр.~\pageref{page1})

В данном случае для кода (15,11) матрица $I_k$,будет иметь следующий
вид:

\begin{equation*}
  I_{11} =  \left[
    \begin{array}{cccccccccccc}
      1 & 0 & 0 & 0 & 0 & 0 & 0 & 0 & 0 & 0 & 0\\
      0 & 1 & 0 & 0 & 0 & 0 & 0 & 0 & 0 & 0 & 0\\
      0 & 0 & 1 & 0 & 0 & 0 & 0 & 0 & 0 & 0 & 0\\
      0 & 0 & 0 & 1 & 0 & 0 & 0 & 0 & 0 & 0 & 0\\
      0 & 0 & 0 & 0 & 1 & 0 & 0 & 0 & 0 & 0 & 0\\
      0 & 0 & 0 & 0 & 0 & 1 & 0 & 0 & 0 & 0 & 0\\
      0 & 0 & 0 & 0 & 0 & 0 & 1 & 0 & 0 & 0 & 0\\
      0 & 0 & 0 & 0 & 0 & 0 & 0 & 1 & 0 & 0 & 0\\
      0 & 0 & 0 & 0 & 0 & 0 & 0 & 0 & 1 & 0 & 0\\
      0 & 0 & 0 & 0 & 0 & 0 & 0 & 0 & 0 & 1 & 0\\
      0 & 0 & 0 & 0 & 0 & 0 & 0 & 0 & 0 & 0 & 1 
    \end{array}
  \right].
\end{equation*}

Таким образом, производящая матрица будет иметь следующий вид:

\begin{equation*}
  G_{(15, 11)} =  \left[
    \begin{tabular}{l*{15}{c}r}
      1 & 0 & 0 & 0 & 0 & 0 & 0 & 0 & 0 & 0 & 0 & 1 & 0 & 0 & 1\\
      0 & 1 & 0 & 0 & 0 & 0 & 0 & 0 & 0 & 0 & 0 & 1 & 1 & 0 & 1\\ 
      0 & 0 & 1 & 0 & 0 & 0 & 0 & 0 & 0 & 0 & 0 & 1 & 1 & 1 & 1\\ 
      0 & 0 & 0 & 1 & 0 & 0 & 0 & 0 & 0 & 0 & 0 & 1 & 1 & 1 & 0\\ 
      0 & 0 & 0 & 0 & 1 & 0 & 0 & 0 & 0 & 0 & 0 & 0 & 1 & 1 & 1\\ 
      0 & 0 & 0 & 0 & 0 & 1 & 0 & 0 & 0 & 0 & 0 & 1 & 0 & 1 & 0\\ 
      0 & 0 & 0 & 0 & 0 & 0 & 1 & 0 & 0 & 0 & 0 & 0 & 1 & 0 & 1\\ 
      0 & 0 & 0 & 0 & 0 & 0 & 0 & 1 & 0 & 0 & 0 & 1 & 0 & 1 & 1\\ 
      0 & 0 & 0 & 0 & 0 & 0 & 0 & 0 & 1 & 0 & 0 & 1 & 1 & 0 & 0\\ 
      0 & 0 & 0 & 0 & 0 & 0 & 0 & 0 & 0 & 1 & 0 & 0 & 1 & 1 & 0\\ 
      0 & 0 & 0 & 0 & 0 & 0 & 0 & 0 & 0 & 0 & 1 & 0 & 0 & 1 & 1
    \end{tabular}
  \right].
\end{equation*}

Для получения кодовой комбинации необходимо вектор, соответствующий
кодовой комбинации $a(x)$, умножить на матрицу $G_{(15,
  11)}$. Полученный в результате умножения вектор и будет являться
разрешённой кодовой комбинацией. В соответствии с заданием
$a(x) = 10011000111$. Выполним умножение:

\begin{gather*}
  V(x) = a(x) \times G_{(15, 11)} = 10011000111 \times \\
  \times\left[\begin{tabular}{l*{15}{c}r}
      1 & 0 & 0 & 0 & 0 & 0 & 0 & 0 & 0 & 0 & 0 & 1 & 0 & 0 & 1\\
      0 & 1 & 0 & 0 & 0 & 0 & 0 & 0 & 0 & 0 & 0 & 1 & 1 & 0 & 1\\ 
      0 & 0 & 1 & 0 & 0 & 0 & 0 & 0 & 0 & 0 & 0 & 1 & 1 & 1 & 1\\ 
      0 & 0 & 0 & 1 & 0 & 0 & 0 & 0 & 0 & 0 & 0 & 1 & 1 & 1 & 0\\ 
      0 & 0 & 0 & 0 & 1 & 0 & 0 & 0 & 0 & 0 & 0 & 0 & 1 & 1 & 1\\ 
      0 & 0 & 0 & 0 & 0 & 1 & 0 & 0 & 0 & 0 & 0 & 1 & 0 & 1 & 0\\ 
      0 & 0 & 0 & 0 & 0 & 0 & 1 & 0 & 0 & 0 & 0 & 0 & 1 & 0 & 1\\ 
      0 & 0 & 0 & 0 & 0 & 0 & 0 & 1 & 0 & 0 & 0 & 1 & 0 & 1 & 1\\ 
      0 & 0 & 0 & 0 & 0 & 0 & 0 & 0 & 1 & 0 & 0 & 1 & 1 & 0 & 0\\ 
      0 & 0 & 0 & 0 & 0 & 0 & 0 & 0 & 0 & 1 & 0 & 0 & 1 & 1 & 0\\ 
      0 & 0 & 0 & 0 & 0 & 0 & 0 & 0 & 0 & 0 & 1 & 0 & 0 & 1 & 1
    \end{tabular}\right] = \\
  = 100110001111001.
\end{gather*}

Проверочная матрица в систематическом виде строится на основе матрицы
$G_{(n,k)}$, а именно:

\begin{equation*}
  H_{(n, k)} = \left[R^T_{(k, r)} I_r \right],
\end{equation*}

\begin{ESKDexplanation}
\item[где ]  $I_r$~--- единичная матрица; 
\item $R^T_{(k, r)}$~--- матрица $R_{(k, r)}$ из
$G_{(n,k)}$ в транспонированном виде. 
\end{ESKDexplanation}
  
В соответствии с матрицей $G_{(15,4)}$ получаем:

\begin{equation*}
  H_{(15, 11)} =  \left[
    \begin{tabular}{l*{15}{c}r}
      1 & 1 & 1 & 1 & 0 & 1 & 0 & 1 & 1 & 0 & 0 & 1 & 0 & 0 & 0\\
      0 & 1 & 1 & 1 & 1 & 0 & 1 & 0 & 1 & 1 & 0 & 0 & 1 & 0 & 0\\ 
      0 & 0 & 1 & 1 & 1 & 1 & 0 & 1 & 0 & 1 & 1 & 0 & 0 & 1 & 0\\ 
      1 & 1 & 1 & 0 & 1 & 0 & 1 & 1 & 0 & 0 & 1 & 0 & 0 & 0 & 1\\ 
    \end{tabular}
  \right].
\end{equation*}

Для определения синдрома необходимо умножить полученную кодовую
комбинацию на $H^T_{(n, r)}$. Полученный в результате умножения вектор
и будет являться синдромом, по которому можно судить о наличии и
расположении ошибки, или её отсутствии. Принятая кодовая комбинация
$V'(x) = 111110001000010$. Выполним умножение:

\begin{equation*}
  S(x) = V'(x) \times H^T_{(n, r)} = 111110001000010 \times
  \left[
    \begin{array}{cccc}
      1 & 0 & 0 & 1 \\
      1 & 1 & 0 & 1 \\
      1 & 1 & 1 & 1 \\
      1 & 1 & 1 & 0 \\
      0 & 1 & 1 & 1 \\
      1 & 0 & 1 & 0 \\
      0 & 1 & 0 & 1 \\
      1 & 0 & 1 & 1 \\
      1 & 1 & 0 & 0 \\
      0 & 1 & 1 & 0 \\
      0 & 0 & 1 & 1 \\
      1 & 0 & 0 & 0 \\
      0 & 1 & 0 & 0 \\
      0 & 0 & 1 & 0 \\
      0 & 0 & 0 & 1 
    \end{array}
  \right] = 1100.
\end{equation*}

В соответствии с этим синдромом определяем по матрице $H_{(15,4)}$,
что ошибка произошла в 12 разряде, следовательно, исправленная
комбинация будет 111010001000010.

\subsection{Ответ}

В результате решения были найдены образующий полином $g(x) = x^4 + x
+1$, разрешённая кодовая комбинация, соответствующая информационной
комбинации a(x)~--- V(x) = 100110001111001, была определена ошибка в
кодовой комбинации $V'(x) = 111110001000010$, после исправления была
получена исправленная кодовая комбинация~--- 111010001000010.
\newpage

\section{Код Боуза"--~Чоудхури"--~Хоквингема}
\label{sec:BCH}

\subsection{Условие задачи}

Определить порождающий многочлен $g(x)$ примитивного кода БЧХ над
$GF(2)$ длины $n = 2^m -1$, исправляющего ошибки кратностью
$t_u = 2$. Сформировать разрешённую комбинацию систематического
кода, соответствующую заданной информационной комбинации:

\begin{equation*}
a(x) = 100111000011111000000.
\end{equation*}

Построить регистр кодирующего устройства систематического циклического
кода с порождающим многочленом $g(x)$, привести таблицу,
иллюстрирующую состояние ячеек в процессе работы регистра при
поступлении на его вход информационного блока $a(x)$. Определить,
является ли разрешённой принимаемая кодовая комбинация:

\begin{equation*}
V'(x) = 1101111011011011000110001010000.
\end{equation*}


\subsection{Решение задачи}

Первым этапом решения задачи по синтезу кода БЧХ является определение
порождающего многочлена $g(x)$. Для этого необходимо воспользоваться
методикой, описанной выше при решении задачи синтеза кода Хемминга. По
условию задачи кратность исправляемых ошибок $t_u = 2$,
следовательно в соответствии с формулой $j = 2t_u - 1$,
получаем $j = 3$. Длина кода $n = 31$ и $m =5$. Таким образом, полином
$g(x)$ будет равен произведению полиномов записанных в первой, второй
и третьей строках 4-го столбца таблицы минимальных многочленов, т.~е.

\begin{gather*}
  g(x) = 45 \times 75 = (x^5 + x^2 + 1)(x^5 + x^4 + x^3 + x^2 + 1) = \\
  = x^{10} + x^9 + x^8 + x^6 + x^5 + x^3 + 1.
\end{gather*}

В двоичном виде $g(x) = 11101101001$.

После нахождения порождающего многочлена, необходимо сформировать
разрешённую кодовую комбинацию систематического кода, соответствующую
заданной информационной последовательности, по следующей формуле:

\begin{equation}
  \label{eq:bch_poly}
  V(x) = a(x) \times x^r + r(x),
\end{equation}


\begin{ESKDexplanation}
\item[где ] $r$~--- количество проверочных разрядов;
\item $r(x)$~--- остаток от деления $a(x) \times x^r$ на $g(x)$.
\end{ESKDexplanation}

Таким образом, первые $n-r$ разрядов будут совпадать с информационной
последовательностью,~а последние $r$ разрядов будут проверочными. По
условию задачи $a(x) = 100111000011111000000$, а полиному $x^{10}$
соответствует последовательность $10000000000$, таким образом,
получаем:

\begin{equation*}
  a(x) \cdot x^r = 1001110000111110000000000000000.
\end{equation*}

Остаток от деления $r(x)$ найден в программе
\textit{Octave} (см. Приложение~\ref{sec:octave},
стр.~\pageref{page2}):

\begin{equation*}
r(x) = 1111011111.
\end{equation*}


Таким образом, разрешённая кодовая комбинация по
формуле~(\ref{eq:bch_poly}):

\begin{equation*}
  V(x) = 1001110000111110000001111011111.
\end{equation*}

Следующим этапом решения задачи является построение регистра
кодирующего устройства и приведение таблицы переключений состояний
ячеек регистра при поступлении на вход информационной
последовательности $a(x)$. Схема регистра кодирующего устройства
приведена в приложении А на рисунке~\ref{fig:kdu}, а таблице
переключений соответствует таблица~\ref{tab:reg}.

После поступления на вход регистра информационно блока $a(x) =
100111000011111000000$ в его ячейках формируется проверочная
последовательность $r(x) = 1111011111$, которая соответствует остатку от
деления $a(x) \cdot x^r$ на $g(x)$.

По условию задачи так же необходимо определить, является ли разрешённой
кодовая комбинация:

\begin{equation*}
  V'(x) = 1101111011011011000110001010000.
\end{equation*}

Для этого необходимо найти синдром $S(x)$, т.е.  произвести деление
многочлена $V'(x)$ на порождающий многочлен $g(x)$. Если остаток от
деления будет равен нулю, то комбинация является разрешённой.  В
противном случае~--- неразрешённой. Операция производилась в
программе \textit{Octave}.

В результате выполнения деления, остаток равен 0, следовательно,
принятая комбинация является разрешённой.

\subsection{Ответ}

В процессе решения задачи был определён порождающий многочлен:

\begin{equation*}
g(x) = x^{10} + x^9 + x^8 + x^6 + x^5 + x^3 + 1;
\end{equation*}
сформирована разрешённая кодовая комбинация, соответствующая
информационной последовательности $a(x)$:

\begin{equation*}
V(x) = 1001110000111110000001111011111;
\end{equation*}
построен регистр кодирующего устройства и построена таблица состояний
ячеек регистра при поступлении на его вход информационной
последовательности $a(x)$. Так же было определено, что принятая кодовая
комбинация $V'(x)$ является разрешённой.
\newpage

\section{Код Рида"--~Соломона}
\label{sec:Reed-Solomon}

\subsection{Условие задачи}

Построить таблицы представления, сложения и умножения элементов в поле
$GF(q)$, $q = 2^4$. Определить порождающий многочлен кода
Рида"--~Соломона над полем, исходя из условия, что код должен
исправлять $t_u = 3$ неправильно принятых $q$-ичных
символов. Сформировать разрешённую кодовую комбинацию систематического
кода, соответствующую заданной информационной комбинации:

\begin{equation*}
  a(x) = 000000000100000001011010000100001001.
\end{equation*}

Вычислить синдром и определить, является ли разрешённой принимаемая
кодовая комбинация:

\begin{gather*}
  V'(x) = 0111110001100110011001101101000001001111010000010010\\10000101.
\end{gather*}

\subsection{Решение задачи}

Первым этапом решения задачи является построение таблицы представлений
поля $GF(16)$, построенного на основе многочлена $p(z)=z^4+z+1$ c
примитивным элементом $z$. Результат представлен в
таблице~\ref{tab:predst_gf16}.

\begin{table}[h!]
  % \centering
  \caption{Таблица представлений поля $GF(16)$}
  \label{tab:predst_gf16}
  \begin{tabular}{|l|l|l|l|}
    \hline
    Степенное & Многочленное & Кодовое & Десятичное \\
    представление & представление & представление & представление \\\hline
    $\alpha^0   $ & $1          $ & 0001 & 1 \\\hline
    $\alpha^1   $ & $z          $ & 0010 & 2 \\\hline
    $\alpha^2   $ & $z^2        $ & 0100 & 4 \\\hline
    $\alpha^3   $ & $z^3        $ & 1000 & 8 \\\hline
    $\alpha^4   $ & $z+1        $ & 0011 & 3 \\\hline
    $\alpha^5   $ & $z^2+z      $ & 0110 & 6 \\\hline
    $\alpha^6   $ & $z^3+z^2    $ & 1100 & 12 \\\hline
    $\alpha^7   $ & $z^3+z+1    $ & 1011 & 11 \\\hline
    $\alpha^8   $ & $z^2+1      $ & 0101 & 5 \\\hline
    $\alpha^9   $ & $z^3+z      $ & 1010 & 10 \\\hline
    $\alpha^{10}$ & $z^2+z+1    $ & 0111 & 7 \\\hline
    $\alpha^{11}$ & $z^3+z^2+z  $ & 1110 & 14 \\\hline
    $\alpha^{12}$ & $z^3+z^2+z+1$ & 1111 & 15 \\\hline
    $\alpha^{13}$ & $z^3+z^2+1  $ & 1101 & 13 \\\hline
    $\alpha^{14}$ & $z^3+1      $ & 1001 & 9 \\\hline
  \end{tabular}
\end{table}

Многочлен, представляющий элемент поля в графе <<многочленное
представление>>, может быть получен как остаток от деления элементов
степенного представления на образующий многочлен~$p(z)$, при этом
элемент степенного представления $\alpha^i$ записывается как $z^i$.

\begin{table}[h!]
\caption{Сложение в поле GF(16)}\scriptsize
  \begin{tabular}{|l|l|l|l|l|l|l|l|l|l|l|l|l|l|l|l|l|}
  \hline
  + & 0 & 1 & $\alpha^1$ & $\alpha^2$ & $\alpha^3$ & $\alpha^4$ & $\alpha^5$ & $\alpha^6$ & $\alpha^7$ & $\alpha^8$ & $\alpha^9$ & $\alpha^{10}$ & $\alpha^{11}$ & $\alpha^{12}$ & $\alpha^{13}$ & $\alpha^{14}$ \\
  \hline
  0 & 0 & 1 & $\alpha^1$ & $\alpha^2$ & $\alpha^3$ & $\alpha^4$ & $\alpha^5$ & $\alpha^6$ & $\alpha^7$ & $\alpha^8$ & $\alpha^9$ & $\alpha^{10}$ & $\alpha^{11}$ & $\alpha^{12}$ & $\alpha^{13}$ & $\alpha^{14}$ \\
  \hline
  1 & 1 & 0 & $\alpha^4$ & $\alpha^8$ & $\alpha^{14}$ & $\alpha^1$ & $\alpha^{10}$ & $\alpha^{13}$ & $\alpha^9$ & $\alpha^2$ & $\alpha^7$ & $\alpha^5$ & $\alpha^{12}$ & $\alpha^{11}$ & $\alpha^6$ & $\alpha^3$ \\
  \hline
  $\alpha^1$& $\alpha^1$& $\alpha^4$ & 0 & $\alpha^5$ & $\alpha^9$ & 1 & $\alpha^2$ & $\alpha^{11}$ & $\alpha^{14}$ & $\alpha^{10}$ & $\alpha^3$ & $\alpha^8$ & $\alpha^6$ & $\alpha^{13}$ & $\alpha^{12}$ & $\alpha^7$ \\
  \hline
  $\alpha^2$ & $\alpha^2$ & $\alpha^8$ & $\alpha^5$ & 0 & $\alpha^6$ & $\alpha^{10}$ & $\alpha^1$& $\alpha^3$ & $\alpha^{12}$ & 1 & $\alpha^{11}$ & $\alpha^4$ & $\alpha^9$ & $\alpha^7$ & $\alpha^{14}$ & $\alpha^{13}$ \\
  \hline
  $\alpha^3$ & $\alpha^3$ & $\alpha^{14}$ & $\alpha^9$ & $\alpha^6$ & 0 & $\alpha^7$ & $\alpha^{11}$ & $\alpha^2$ & $\alpha^4$ & $\alpha^{13}$ & $\alpha^1$& $\alpha^{12}$ & $\alpha^5$ & $\alpha^{10}$ & $\alpha^8$ & 1 \\
  \hline
  $\alpha^4$ & $\alpha^4$ & $\alpha^1$& 1 & $\alpha^{10}$ & $\alpha^7$ & 0 & $\alpha^8$ & $\alpha^{12}$ & $\alpha^3$ & $\alpha^5$ & $\alpha^{14}$ & $\alpha^2$ & $\alpha^{13}$ & $\alpha^6$ & $\alpha^{11}$ & $\alpha^9$ \\
  \hline
  $\alpha^5$ & $\alpha^5$ & $\alpha^{10}$ & $\alpha^2$ & $\alpha^1$& $\alpha^{11}$ & $\alpha^8$ & 0 & $\alpha^9$ & $\alpha^{13}$ & $\alpha^4$ & $\alpha^6$ & 1 & $\alpha^3$ & $\alpha^{14}$ & $\alpha^7$ & $\alpha^{12}$ \\
  \hline
  $\alpha^6$ & $\alpha^6$ & $\alpha^{13}$ & $\alpha^{11}$ & $\alpha^3$ & $\alpha^2$ & $\alpha^{12}$ & $\alpha^9$ & 0 & $\alpha^{10}$ & $\alpha^{14}$ & $\alpha^5$ & $\alpha^7$ & $\alpha^1$& $\alpha^4$ & 1 & $\alpha^8$ \\
  \hline
  $\alpha^7$ & $\alpha^7$ & $\alpha^9$ & $\alpha^{14}$ & $\alpha^{12}$ & $\alpha^4$ & $\alpha^3$ & $\alpha^{13}$ & $\alpha^{10}$ & 0 & $\alpha^{11}$ & 1 & $\alpha^6$ & $\alpha^8$ & $\alpha^2$ & $\alpha^5$ & $\alpha^1$\\
  \hline
  $\alpha^8$ & $\alpha^8$ & $\alpha^2$ & $\alpha^{10}$ & 1 & $\alpha^{13}$ & $\alpha^5$ & $\alpha^4$ & $\alpha^{14}$ & $\alpha^{11}$ & 0 & $\alpha^{12}$ & $\alpha^1$& $\alpha^7$ & $\alpha^9$ & $\alpha^3$ & $\alpha^6$ \\
  \hline
  $\alpha^9$ & $\alpha^9$ & $\alpha^7$ & $\alpha^3$ & $\alpha^{11}$ & $\alpha^1$& $\alpha^4$ & $\alpha^6$ & $\alpha^5$ & 1 & $\alpha^{12}$ & 0 & $\alpha^{13}$ & $\alpha^2$ & $\alpha^8$ & $\alpha^{10}$ & $\alpha^4$ \\
  \hline
  $\alpha^{10}$ & $\alpha^{10}$ & $\alpha^5$ & $\alpha^8$ & $\alpha^4$ & $\alpha^{12}$ & $\alpha^2$ & 1 & $\alpha^7$ & $\alpha^6$ & $\alpha^1$& $\alpha^{13}$ & 0 & $\alpha^{14}$ & $\alpha^3$ & $\alpha^9$ & $\alpha^{11}$ \\
  \hline
  $\alpha^{11}$ & $\alpha^{11}$ & $\alpha^{12}$ & $\alpha^6$ & $\alpha^9$ & $\alpha^5$ & $\alpha^{13}$ & $\alpha^3$ & $\alpha^1$& $\alpha^8$ & $\alpha^7$ & $\alpha^2$ & $\alpha^{14}$ & 0 & 1 & $\alpha^4$ & $\alpha^{10}$ \\
  \hline
  $\alpha^{12}$ & $\alpha^{12}$ & $\alpha^{11}$ & $\alpha^{13}$ & $\alpha^7$ & $\alpha^{10}$ & $\alpha^6$ & $\alpha^{14}$ & $\alpha^4$ & $\alpha^2$ & $\alpha^9$ & $\alpha^8$ & $\alpha^3$ & 1 & 0 & $\alpha^1$& $\alpha^5$ \\
  \hline
  $\alpha^{13}$ & $\alpha^{13}$ & $\alpha^6$ & $\alpha^{12}$ & $\alpha^{14}$ & $\alpha^8$ & $\alpha^{11}$ & $\alpha^7$ & 1 & $\alpha^5$ & $\alpha^3$ & $\alpha^{10}$ & $\alpha^9$ & $\alpha^4$ & $\alpha^1$& 0 & $\alpha^2$ \\
  \hline
  $\alpha^{14}$ & $\alpha^{14}$ & $\alpha^3$ & $\alpha^7$ & $\alpha^{13}$ & 1 & $\alpha^9$ & $\alpha^{12}$ & $\alpha^8$ & $\alpha^1$& $\alpha^6$ & $\alpha^6$ & $\alpha^{11}$ & $\alpha^{10}$ & $\alpha^5$ & $\alpha^2$ & 0 \\
  \hline
\end{tabular}
\label{tab:slogenie_gf16}
\end{table}
\normalsize


\begin{table}[h!]
  \caption{Умножение в поле $GF(16)$}
  \label{tab:times_gf16}

  \scriptsize
  \begin{tabular}{|l|l|l|l|l|l|l|l|l|l|l|l|l|l|l|l|l|}
    \hline
    $\times$ & 0 & 1 & $\alpha^1$ & $\alpha^2$ & $\alpha^3$ & $\alpha^4$ & 5 & $\alpha^6$ & $\alpha^7$ & $\alpha^8$ & $\alpha^9$ & 10 & $\alpha^{11}$ & $\alpha^{12}$ & $\alpha^{13}$ & $\alpha^{14}$ \\
    \hline
    0 & 0 & 0 & 0 & 0 & 0 & 0 & 0 & 0 & 0 & 0 & 0 & 0 & 0 & 0 & 0 & 0 \\
    \hline
    1 & 0 & 1 & $\alpha^1$ & $\alpha^2$ & $\alpha^3$ & $\alpha^4$ & $\alpha^5$ & $\alpha^6$ & $\alpha^7$ & $\alpha^8$ & $\alpha^9$ & $\alpha^{10}$ & $\alpha^{11}$ & $\alpha^{12}$ & $\alpha^{13}$ & $\alpha^{14}$ \\
    \hline
    $\alpha^1$ & 0 & $\alpha^1$ & $\alpha^2$ & $\alpha^3$ & $\alpha^4$ & $\alpha^5$ & $\alpha^6$ & $\alpha^7$ & $\alpha^8$ & $\alpha^9$ & $\alpha^{10}$ & $\alpha^{11}$ & $\alpha^{12}$ & $\alpha^{13}$ & $\alpha^{14}$ & 1 \\
    \hline
    $\alpha^2$ & 0 & $\alpha^2$ & $\alpha^3$ & $\alpha^4$ & $\alpha^5$ & $\alpha^6$ & $\alpha^7$ & $\alpha^8$ & $\alpha^9$ & $\alpha^{10}$ & $\alpha^{11}$ & $\alpha^{12}$ & $\alpha^{13}$ & $\alpha^{14}$ & 1 & $\alpha^1$ \\
    \hline
    $\alpha^3$ & 0 & $\alpha^3$ & $\alpha^4$ & $\alpha^5$ & $\alpha^6$ & $\alpha^7$ & $\alpha^8$ & $\alpha^9$ & $\alpha^{10}$ & $\alpha^{11}$ & $\alpha^{12}$ & $\alpha^{13}$ & $\alpha^{14}$ & 1 & $\alpha^1$ & $\alpha^2$ \\
    \hline
    $\alpha^4$ & 0 & $\alpha^4$ & $\alpha^5$ & $\alpha^6$ & $\alpha^7$ & $\alpha^8$ & $\alpha^9$ & $\alpha^{10}$ & $\alpha^{11}$ & $\alpha^{12}$ & $\alpha^{13}$ & $\alpha^{14}$ & 1 & $\alpha^1$ & $\alpha^2$ & $\alpha^3$ \\
    \hline
    $\alpha^5$ & 0 & $\alpha^5$ & $\alpha^6$ & $\alpha^7$ & $\alpha^8$ & $\alpha^9$ & $\alpha^{10}$ & $\alpha^{11}$ & $\alpha^{12}$ & $\alpha^{13}$ & $\alpha^{14}$ & 1 & $\alpha^1$ & $\alpha^2$ & $\alpha^3$ & $\alpha^4$ \\
    \hline
    $\alpha^6$ & 0 & $\alpha^6$ & $\alpha^7$ & $\alpha^8$ & $\alpha^9$ & $\alpha^{10}$ & $\alpha^{11}$ & $\alpha^{12}$ & $\alpha^{13}$ & $\alpha^{14}$ & 1 & $\alpha^1$ & $\alpha^2$ & $\alpha^3$ & $\alpha^4$ & $\alpha^5$ \\
    \hline
    $\alpha^7$ & 0 & $\alpha^7$ & $\alpha^8$ & $\alpha^9$ & $\alpha^{10}$ & $\alpha^{11}$ & $\alpha^{12}$ & $\alpha^{13}$ & $\alpha^{14}$ & 1 & $\alpha^1$ & $\alpha^2$ & $\alpha^3$ & $\alpha^4$ & $\alpha^5$ & $\alpha^6$ \\
    \hline
    $\alpha^8$ & 0 & $\alpha^8$ & $\alpha^9$ & $\alpha^{10}$ & $\alpha^{11}$ & $\alpha^{12}$ & $\alpha^{13}$ & $\alpha^{14}$ & 1 & $\alpha^1$ & $\alpha^2$ & $\alpha^3$ & $\alpha^4$ & $\alpha^5$ & $\alpha^6$ & $\alpha^7$ \\
    \hline
    $\alpha^{9}$ & 0 & $\alpha^9$ & $\alpha^{10}$ & $\alpha^{11}$ & $\alpha^{12}$ & $\alpha^{13}$ & $\alpha^{14}$ & 1 & $\alpha^1$ & $\alpha^2$ & $\alpha^3$ & $\alpha^4$ & $\alpha^5$ & $\alpha^6$ & $\alpha^7$ & $\alpha^8$ \\
    \hline
    $\alpha^{10}$ & 0 & $\alpha^{10}$ & $\alpha^{11}$ & $\alpha^{12}$ & $\alpha^{13}$ & $\alpha^{14}$ & 1 & $\alpha^1$ & $\alpha^2$ & $\alpha^3$ & $\alpha^4$ & $\alpha^5$ & $\alpha^6$ & $\alpha^7$ & $\alpha^8$ & $\alpha^9$ \\
    \hline
    $\alpha^{11}$ & 0 & $\alpha^{11}$ & $\alpha^{12}$ & $\alpha^{13}$ & $\alpha^{14}$ & 1 & $\alpha^1$ & $\alpha^2$ & $\alpha^3$ & $\alpha^4$ & $\alpha^5$ & $\alpha^6$ & $\alpha^7$ & $\alpha^8$ & $\alpha^9$ & $\alpha^{10}$ \\
    \hline
    $\alpha^{12}$ & 0 & $\alpha^{12}$ & $\alpha^{13}$ & $\alpha^{14}$ & 1 & $\alpha^1$ & $\alpha^2$ & $\alpha^3$ & $\alpha^4$ & $\alpha^5$ & $\alpha^6$ & $\alpha^7$ & $\alpha^8$ & $\alpha^9$ & $\alpha^{10}$ & $\alpha^{11}$ \\
    \hline
    $\alpha^{13}$ & 0 & $\alpha^{13}$ & $\alpha^{14}$ & 1 & $\alpha^1$ & $\alpha^2$ & $\alpha^3$ & $\alpha^4$ & $\alpha^5$ & $\alpha^6$ & $\alpha^7$ & $\alpha^8$ & $\alpha^9$ & $\alpha^{10}$ & $\alpha^{11}$ & $\alpha^{12}$ \\
    \hline
    $\alpha^{14}$ & 0 & $\alpha^{14}$ & 1 & $\alpha^1$ & $\alpha^2$ & $\alpha^3$ & $\alpha^4$ & $\alpha^5$ & $\alpha^6$ & $\alpha^7$ & $\alpha^8$ & $\alpha^9$ & $\alpha^{10}$ & $\alpha^{11}$ & $\alpha^{12}$ & $\alpha^{13}$ \\
    \hline
  \end{tabular}
\end{table}
\normalsize

После того, как определены таблицы сложения и умножения в поле
$GF(16)$ можно приступать к синтезу кода. Первым этапом синтеза
является определение порождающего многочлена $g(x)$. Для кода
Рида"--~Соломона порождающий многочлен определяется по формуле:

\begin{equation*}
  g(x) = \prod_{i = v}^{i = 2t_u + v -1}(x - \alpha^i), 
\end{equation*}

\begin{ESKDexplanation}
\item[где ] $t_u$~--- кратность исправляемых ошибок;
\item $v$~--- чаще всего принимается равным 1.
\end{ESKDexplanation}

Таким образом:

\begin{gather*}
  g(x) = \prod_{i =1}^6 (x - \alpha^i) = \\
  = (x - \alpha)(x - \alpha^2)(x - \alpha^3)(x - \alpha^4)(x - \alpha^5)(x - \alpha^6) = \\
  = (x -2)(x - 4)(x -8)(x-3)(x-6)(x-12) = \\
  = x^6 + x^5\alpha^{10} + x^4\alpha^{14} + x^3\alpha^4 + x^2\alpha^6
  + x\alpha^9 + \alpha^6.
\end{gather*}

После нахождения порождающего многочлена, необходимо сформировать
разрешённую кодовую комбинацию систематического кода, соответствующую
заданной информационной комбинации $a(x)$, по формуле:

\begin{equation*}
  V(x) = a(x)x^r + r(x),
\end{equation*}

\begin{ESKDexplanation}
\item[где ] $r$~--- количество проверочных разрядов;
\item $r(x)$~--- остаток от деления $a(x)x^r$ на $g(x)$.
\end{ESKDexplanation}

\begin{gather*}
  N = 2^m - 1 = 15;\\
  d = 2t_u + 1 = 7;\\
  k = N - d + 1 = 9;\\
  r = N - k = 6.
\end{gather*}

Таким образом,

\begin{equation*}
  r(x) = R_{g(x)} = \left[a(x)x^6\right].
\end{equation*}

Все дальнейшие операции производились в программе \textit{Octave}
(см. Приложение~\ref{sec:octave}, стр.~\pageref{page3}).

Умножая $a(x)$ на $x^6$ и находя остаток от деления $r(x)$, получаем
кодовую комбинацию в десятичном представлении
$V(x) =$ 0    0    4    0    5   10    1    0    9    9   10   13    4
15    8. Или в двоичном виде:

\begin{gather*}
  V(x) = [0000000001000000010110100001000010011001101011010100\\
  11111000].
\end{gather*}

Синдром для кода Рида"--~Соломона определяется по формуле:

\begin{gather*}
  S = S_{d-2} \ldots S_1S_0,\\
  S_J = \sum _{i=0} ^{N-1} V_i \cdot z_i^{m_0+j} \, (j = 0
  \ldots d-2)
\end{gather*}

\begin{ESKDexplanation}
\item[где ] $m_0 = 1$;
\item $V_i$~--- символы коэффициентов полинома, представляющего
  кодовую комбинацию;
\item $z_i = \alpha^i$~--- локаторы, где $\alpha$~--- примитивный
  элемент поля $GF(q)$.
\end{ESKDexplanation}

Таким образом, синдромы:

\begin{gather*}
  S_0 = \sum_{i = 0}^{14}V_iz_i^1 = V_0 + V_1z_1 + V_2z_2 + V_3z_3 +
  V_4z_4 + V_5z_5 + V_6z_6 + V_7z_7 + \\ + V_8z_8 + V_9z_9 +
  V_{10}z_{10} + V_{11}z_{11} + V_{12}z_{12} + V_{13}z_{13} +
  V_{14}z_{14} =  \\ = \alpha^8 + \alpha^4 + \alpha^3 + \alpha^3 +
  \alpha^6 + \alpha^2 + \alpha^8 + \alpha^6 + \alpha^{14} + 1 + \alpha
  + \\ + \alpha^2 + \alpha^4 + \alpha^9 = 0;
\end{gather*}

Остальные синдромы $S_1 - S_5$ в соответствии с описанной методикой
вычислялись в программе \textit{Octave} и также равны 0. В результате
вычислений синдром $S = 000000$, т.\,е. принятая кодовая комбинация не
содержит ошибок.

\subsection{Ответ}

В результате решения задачи были построены таблицы представлений,
сложения и умножения в $GF(16)$, на основе примитивного многочлена
$p(z)=z^4+z+1$ c примитивным элементом $z$. Была определена кодовая
комбинация, соответствующая информационной комбинации $a(x)$:

\begin{gather*}
  V(x) = [0000000001000000010110100001000010011001101011010100\\
  11111000].
\end{gather*}


Также было определено, что принятая комбинация $V'(x)$, является
разрешённой кодовой комбинацией.

\subsection{Дополнительная задача}

Найти расположение ошибок и их значения в принятой кодовой комбинации:

\begin{center}
  $V'(x) = $ 3 8 2 0 5 0 11 13 4 3 9 9 4 3 1.
\end{center}

в поле $GF(q)$, $q = 2^4$, $t_u = 3$.

На практике известна только принятая кодовая комбинация $V'(x)$,
однако также известно, что если полином ошибок имеет вид:

\begin{equation}
  \label{eq:errors_poly}
  e(x) = e_{j1}x^{j1} + e_{j2}x^{j} + \cdots + e_{j\nu}x^{j\nu},
\end{equation}

\begin{ESKDexplanation}
\item[где ] индексы $1, 2, 3, \ldots , \nu$ обозначают номер ошибки;
\item $j_1, j_2, \ldots , j_{\nu}$~--- характер расположения ошибки с
  этим номером, то $2t_u$ значения синдрома, определенным показанным
  выше способом позволяют записать систему уравнений:
\end{ESKDexplanation}

\begin{equation}
  \label{eq:symptoms}
  \begin{gathered}
    S_1 = e_{j1}\beta_1 + e_{j2}\beta_2 + \ldots +
    e_{j\nu}\beta_{\nu};\\
    S_2 = e_{j1}\beta_1^2 + e_{j2}\beta_2^2 + \ldots +
    e_{j\nu}\beta_{\nu}^2;\\
    \ldots \\
    S_{2tu} = e_{j1}\beta_1^{2t_u} + e_{j2}\beta_2^{2t_u} + \ldots +
    e_{j\nu}\beta_{\nu}^{2t_u}.
  \end{gathered}
\end{equation}

Система содержит $2t_u$ уравнений относительно $2t_u$ неизвестных
$t_u$ расположений ошибок и $t_u$ значений ошибок, характеризуемых
значением $j$.

Однако эта система является нелинейной, т.~к. $\beta_i$ входит в
уравнения в различных степенях.

Общего метода решения таких систем не существует. В данной задаче
рассмотрена методика, называемая алгоритмом декодирования
Рида"--~Соломона.

Введём в рассмотрение полином локатора ошибок:

\begin{multline}
  \label{eq:error_locator}
  \sigma(x) = (1 + \beta_1x)(1 + \beta_2x) \ldots (1 + \beta_{\nu}x) =
  \\ = 1 + \sigma_1x + \sigma_2x^2 + \ldots + \sigma_{\nu}x^{nu}.
\end{multline}

Корнем $\sigma(x)$  будут:

\begin{equation*}
  \frac{1}{\beta_1},\quad \frac{1}{\beta_2},\quad \ldots ,\quad \frac{1}{\beta_i}.
\end{equation*}

Величины $\beta_i$ $(i = \overline{1,\nu})$ представляют номера
расположения ошибок в полиноме $e(x)$.

Определение коэффициентов $\sigma_1 \ldots \sigma_2$
полинома~(\ref{eq:error_locator}) осуществляется на основе
т.~н. авторегрессионной методики моделирования, в соответствии с
которой составляется матричное уравнение, связывающее первые $2t_u-1$
значений синдрома со следующим $2t_u$-м значением.

В общем виде это уравнение выглядит так:

\begin{equation}
  \label{eq:autoreg_matrix}
  \begin{bmatrix}
    S_1 & S_2 & S_3 & \cdots & S_{t_u-1} & S_{t_u}\\
    S_2 & S_3 & S_4 & \cdots & S_{t_u} & S_{t_u+1}\\
    \vdots & \vdots & \vdots  & \ddots & \vdots & \vdots \\
    S_{t_u-1} & S_{t_u} &  S_{t_u+1} & \cdots  & S_{t_u-3} & S_{t_u-2}\\
    S_{t_u} & S_{t_u+1} &  S_{t_u+2} & \cdots  & S_{t_u-2} & S_{t_u-1}\\
  \end{bmatrix}
  \begin{bmatrix}
    \sigma_{t_u}\\
    \sigma_{t_u-1}\\
    \vdots\\
    \sigma_{2}\\
    \sigma_{2}\\
  \end{bmatrix}
  = \begin{bmatrix}
    -S_{2t_u+1}\\
    -S_{2t_u+2}\\
    \vdots\\
    -S_{2t_u-1}\\
    -S_{2t_u}\\
  \end{bmatrix}.
\end{equation}

Поскольку для элементов конечных полей справедливо равенство $-S_i =
S_i$, знаком <<$-$>> в правой части можно пренебречь.

На основании~(\ref{eq:autoreg_matrix}):

\begin{equation}
  \label{eq:my_autoreg_matrix}
  \begin{bmatrix}
    S_1 & S_2 & S_3 \\
    S_2 & S_3 & S_4 \\
    S_3 & S_4 & S_5
  \end{bmatrix}
  \begin{bmatrix}
    \sigma_3\\
    \sigma_2\\
    \sigma_1
  \end{bmatrix}
  \begin{bmatrix}
    S_4\\
    S_5\\
    S_6
  \end{bmatrix}.
\end{equation}

Значения синдромов были вычислены в программе \textit{Octave}
(см. Приложение~\ref{sec:octave}, стр.~\pageref{page4}), с учётом их
значений:

\begin{equation}
  \label{eq:rs_extra_main}
  \begin{bmatrix}
    15 & 1 & 3 \\
    1 & 3 & 1 \\
    3 & 1 & 12
  \end{bmatrix}
  \begin{bmatrix}
    \sigma_3\\
    \sigma_2\\
    \sigma_1
  \end{bmatrix}
  \begin{bmatrix}
    1\\
    12\\
    1
  \end{bmatrix}.
\end{equation}

Стандартный метод решения~(\ref{eq:rs_extra_main}) предполагает
нахождение обратной матрицы $m\_s^{-1}$.

\begin{equation*}
  m\_s^{-1} = \
  \begin{bmatrix}
    7 & 4 & 11 \\
    4 & 8 & 14 \\
    11 & 14 & 10
  \end{bmatrix}.
\end{equation*}

\begin{equation*}
  \begin{bmatrix}
    \sigma_3\\
    \sigma_2\\
    \sigma_1
  \end{bmatrix} = m\_s^{-1}
  \begin{bmatrix}
    1\\
    12\\
    1
  \end{bmatrix} =
  \begin{bmatrix}
    7 & 4 & 11 \\
    4 & 8 & 14 \\
    11 & 14 & 10
  \end{bmatrix}
  \begin{bmatrix}
    1\\
    12\\
    1
  \end{bmatrix} =
  \begin{bmatrix}
    9\\
    0\\
    5
  \end{bmatrix}
\end{equation*}

В соответствии с~(\ref{eq:error_locator}) полином локаторов ошибок
$\sigma(x)$:

\begin{equation}
  \label{eq:error_locator_conc}
  \sigma = 1 + 5x + 9x^3.
\end{equation}

Корни полинома~(\ref{eq:error_locator_conc}):

\begin{equation*}
  \beta_1 = 6(\alpha^5), \quad
  \beta_2 = 14(\alpha^{11}),\quad
  \beta_3 = 13(\alpha^{13}).
\end{equation*}

Поскольку рассматриваемый код является недвоичным, помимо расположения
ошибок, нужно найти и их значения. Это можно сделать так: найденные
значения $\beta_1$, $\beta_2$, $\beta_3$ позволяют записать значения
элементов синдрома $S_1$, $S_2$, $S_3$, определённые из полинома
ошибок~(\ref{eq:errors_poly}) в виде:

\begin{equation}
  \label{sindr_elements}
  \begin{gathered}
    S_1 = e(\alpha) = e_1\beta_1 + e_2\beta_2;\\
    S_2 = e(\alpha^2) = e_1\beta_1^2 + e_2\beta_2^2;\\
    S_3 = e(\alpha^3) = e_1\beta_1^3 + e_2\beta_2^3.
  \end{gathered}
\end{equation}

Выражение~(\ref{sindr_elements}) представляет собой линейную систему с
3-мя неизвестными $e_1$, $e_2$, $e_3$, представляющих собой искомое
значение ошибок. В рассматриваемом примере систему можно решить с
помощью матрицы.

В матричной форме записи выражение~(\ref{sindr_elements}) принимает
вид:

\begin{equation*}
  \begin{bmatrix}
    \beta_1 & \beta_2 & \beta_3 \\
    \beta_1^2 & \beta_2^2 & \beta_3^2 \\
    \beta_1^3 & \beta_2^3 & \beta_3^3 \\
  \end{bmatrix}
  \begin{bmatrix}
    e_1\\
    e_2\\
    e_3
  \end{bmatrix} =
  \begin{bmatrix}
    S_1\\
    S_2\\
    S_3
  \end{bmatrix}.
\end{equation*}

\begin{equation*}
  \begin{bmatrix}
    e_1\\
    e_2\\
    e_3
  \end{bmatrix} = 
  \begin{bmatrix}
    6 & 14 & 13 \\
    6^2 & 14^2 & 13^2\\
    6^3 & 14^3 & 13^3
  \end{bmatrix}^{-1}
  \begin{bmatrix}
    15\\
    1\\
    3
  \end{bmatrix} =
  \begin{bmatrix}
    10 & 3 & 1 \\
    1 & 3 & 15 \\
    11 & 10 & 12
  \end{bmatrix}
  \begin{bmatrix}
    15\\
    1\\
    3
  \end{bmatrix} =
  \begin{bmatrix}
    12\\
    14\\
    14\\
  \end{bmatrix}.
\end{equation*}

Таким образом,

\begin{equation*}
  e(x) = 14x^{13} + 14x^{11} + 12x^{5} = \alpha^{11}x^{13} +
  \alpha^{11}x^{11} + \alpha^6x^5.
\end{equation*}

Складывая найденный полином ошибок $e(x)$ с ошибочно принятой
комбинацией $V'(x)$, получаем исправленную кодовую комбинацию:

\begin{center}
  $V(x) = $ 3 6 2 14 5 0 11 13 4 15 9 9 4 3 1.
\end{center}


В ошибочно принятой комбинации исправлены все 3 символьных ошибки.

%%%Local Variables: 
%%% mode: latex
%%% TeX-master: "../TermWork_PDS"
%%% End: 



