\section{Код Хемминга}
\label{sec:hemming}

\subsection{Условие задачи}

Определить порождающий многочлен $g(x)$ кода Хемминга, скорость
которого $R \geqslant r_0$, рассматривая его как код БЧХ, исправляющий
одиночные ошибки. Сформировать разрешённую комбинацию систематического
кода, соответствующую заданной информационной комбинации $a(x) =
10011000111$. Исправить ошибку в принимаемой кодовой комбинации
$V'(x) = 111110001000010$.

\subsection{Решение задачи}

В данном случае код Хемминга имеет размерность (15, 11), т.~к. по
условию скорость $R \geqslant r_0$. При $k = 11$ и $n = 15$, получаем,
что $R = \frac{k}{n} = 0{,}73$, что превышает значение $r_0$, равное
$0{,}7$.

Так как код Хемминга предложено рассматривать как код БЧХ, то первым
этапом при расчёте кода будет определение порождающего многочлена
$g(x)$. Порождающий многочлен для кода БЧХ определяется из выражения:

\begin{equation*}
  g(x) = \text{НОК}\left[f_1(x), f_2(x) \ldots f_{2t_{\text{и}}} \right],
\end{equation*}

\begin{ESKDexplanation}
\item[где ] НОК "--- наименьшее общее кратное;
\item $f_1(x), f_2(x), \ldots$ "--- минимальные многочлены корней
  $\alpha^1, \alpha^2 \ldots$ порождающего многочлена.
\end{ESKDexplanation}

Корнем многочлена $g(x)$ называется число (элемент поля) при
подстановке которого в выражение многочлена вместо $х$ многочлен
обращается в~0. Минимальный многочлен элемента $\beta$ поля $GF(q^m)$
определяется из выражения:

\begin{equation*}
  f(x) = (x - \beta^{g^0})(x - \beta^{g^1}) \ldots (x - \beta^{g^{l-1}}),
\end{equation*}

\begin{ESKDexplanation}
\item[где ] $l$ "--- наименьшее целое число, при котором:
\end{ESKDexplanation}

\begin{equation*}
  \beta^{g^0} = \beta^{g^l}.
\end{equation*}

На практике для определения значения порождающего многочлена можно
воспользоваться таблицей минимальных неприводимых многочленов в поле
$GF(2^m)$, в которой приведены минимальные многочлены.

Для определения порождающего многочлена необходимо, во-первых, по
заданной длине кода $n$ определить из выражения $n = 2^m - 1$ значение
параметра $m$, который является степенью сомножителя $g(x)$. Затем из
выражения $j = 2t_{\text{и}} - 1$ определяем максимальный порядок
минимального многочлена, входящих в число сомножителей $g(x)$. После
этого, пользуясь таблицей минимальных многочленов, определяем
выражение для $g(x)$, зависимости от найденных $m$ и $j$. Для этого из
колонки соответствующей параметру $m$ выбираются многочлены с номерами
от $1$ до $j$, которые в результате перемножения дают выражение для
$g(x)$. В нашем случае $n = 15$, а $t_{\text{и}} = 1$, следовательно
пользуясь описанной выше методикой, получаем $m = 4$, $j = 1$, откуда
получаем $g(x)=010011$, или в виде многочлена $g(x)=x^4 + x + 1$.

Следующим этапом построения является построение производящей матрицы
$G_{(n,k)}$. Для систематического кода матрица $G_{(n, k)}$ имеет вид:

\begin{equation*}
  G_{(n,k)} = \left[I_k R_{(k, r)} \right],
\end{equation*}

\begin{ESKDexplanation}
\item[где ] $I_k$ "--- единичная матрица размером $k \times k$.
\end{ESKDexplanation}

Строки матрицы $R_{(k, r)}$ определяются из выражений:

\begin{equation}
\label{r_i(x)}
r_i(x) = R_{g(x)}\bigl(a_i(x) \times x^r \bigr),
\end{equation}

или

\begin{equation*}
r_i(x) = R_{g(x)}\left(x^{n-i} \right),
\end{equation*}

\begin{ESKDexplanation}
\item[где ] $a_i(x)$~--- полином, соответствующий $i$-той строке
  матрицы $I_k$, 
\item $i$~--- номер строки матрицы $R_{(k, r)}$, 
\item $R_{g(x)}\bigl(a(x)\bigr)$~---остаток от деления $a(x)$ на $g(x)$.
\end{ESKDexplanation}

Выполнив деление многочленов, согласно формуле~(\ref{r_i(x)})
получаем:

\begin{equation*}
  R_{(15, 11)} =  \left[
    \begin{array}{cccc}
      1 & 0 & 0 & 1 \\
      1 & 1 & 0 & 1 \\
      1 & 1 & 1 & 1 \\
      1 & 1 & 1 & 0 \\
      0 & 1 & 1 & 1 \\
      1 & 0 & 1 & 0 \\
      0 & 1 & 0 & 1 \\
      1 & 0 & 1 & 1 \\
      1 & 1 & 0 & 0 \\
      0 & 1 & 1 & 0 \\
      0 & 0 & 1 & 1  
    \end{array}
  \right].
\end{equation*}


В данном случае для кода (15, 11) матрица $I_k$,будет иметь следующий
вид:

\begin{equation*}
  I_{11} =  \left[
    \begin{array}{cccccccccccc}
      1 & 0 & 0 & 0 & 0 & 0 & 0 & 0 & 0 & 0 & 0\\
      0 & 1 & 0 & 0 & 0 & 0 & 0 & 0 & 0 & 0 & 0\\
      0 & 0 & 1 & 0 & 0 & 0 & 0 & 0 & 0 & 0 & 0\\
      0 & 0 & 0 & 1 & 0 & 0 & 0 & 0 & 0 & 0 & 0\\
      0 & 0 & 0 & 0 & 1 & 0 & 0 & 0 & 0 & 0 & 0\\
      0 & 0 & 0 & 0 & 0 & 1 & 0 & 0 & 0 & 0 & 0\\
      0 & 0 & 0 & 0 & 0 & 0 & 1 & 0 & 0 & 0 & 0\\
      0 & 0 & 0 & 0 & 0 & 0 & 0 & 1 & 0 & 0 & 0\\
      0 & 0 & 0 & 0 & 0 & 0 & 0 & 0 & 1 & 0 & 0\\
      0 & 0 & 0 & 0 & 0 & 0 & 0 & 0 & 0 & 1 & 0\\
      0 & 0 & 0 & 0 & 0 & 0 & 0 & 0 & 0 & 0 & 1 
    \end{array}
  \right].
\end{equation*}

Таким образом, производящая матрица будет иметь следующий вид:

\begin{equation*}
  G_{(15, 11)} =  \left[
    \begin{tabular}{l*{15}{c}r}
      1 & 0 & 0 & 0 & 0 & 0 & 0 & 0 & 0 & 0 & 0 & 1 & 0 & 0 & 1\\
      0 & 1 & 0 & 0 & 0 & 0 & 0 & 0 & 0 & 0 & 0 & 1 & 1 & 0 & 1\\ 
      0 & 0 & 1 & 0 & 0 & 0 & 0 & 0 & 0 & 0 & 0 & 1 & 1 & 1 & 1\\ 
      0 & 0 & 0 & 1 & 0 & 0 & 0 & 0 & 0 & 0 & 0 & 1 & 1 & 1 & 0\\ 
      0 & 0 & 0 & 0 & 1 & 0 & 0 & 0 & 0 & 0 & 0 & 0 & 1 & 1 & 1\\ 
      0 & 0 & 0 & 0 & 0 & 1 & 0 & 0 & 0 & 0 & 0 & 1 & 0 & 1 & 0\\ 
      0 & 0 & 0 & 0 & 0 & 0 & 1 & 0 & 0 & 0 & 0 & 0 & 1 & 0 & 1\\ 
      0 & 0 & 0 & 0 & 0 & 0 & 0 & 1 & 0 & 0 & 0 & 1 & 0 & 1 & 1\\ 
      0 & 0 & 0 & 0 & 0 & 0 & 0 & 0 & 1 & 0 & 0 & 1 & 1 & 0 & 0\\ 
      0 & 0 & 0 & 0 & 0 & 0 & 0 & 0 & 0 & 1 & 0 & 0 & 1 & 1 & 0\\ 
      0 & 0 & 0 & 0 & 0 & 0 & 0 & 0 & 0 & 0 & 1 & 0 & 0 & 1 & 1
    \end{tabular}
  \right].
\end{equation*}

Для получения кодовой комбинации необходимо вектор, соответствующий
кодовой комбинации $a(x)$, умножить на матрицу $G_{(15,
  11)}$. Полученный в результате умножения вектор и будет являться
разрешённой кодовой комбинацией. В соответствии с заданием
$a(x) = 10011000111$. Выполним умножение:

\begin{gather*}
  V(x) = a(x) \times G_{(15, 11)} = 10011000111 \times \\
  \times\left[\begin{tabular}{l*{15}{c}r}
      1 & 0 & 0 & 0 & 0 & 0 & 0 & 0 & 0 & 0 & 0 & 1 & 0 & 0 & 1\\
      0 & 1 & 0 & 0 & 0 & 0 & 0 & 0 & 0 & 0 & 0 & 1 & 1 & 0 & 1\\ 
      0 & 0 & 1 & 0 & 0 & 0 & 0 & 0 & 0 & 0 & 0 & 1 & 1 & 1 & 1\\ 
      0 & 0 & 0 & 1 & 0 & 0 & 0 & 0 & 0 & 0 & 0 & 1 & 1 & 1 & 0\\ 
      0 & 0 & 0 & 0 & 1 & 0 & 0 & 0 & 0 & 0 & 0 & 0 & 1 & 1 & 1\\ 
      0 & 0 & 0 & 0 & 0 & 1 & 0 & 0 & 0 & 0 & 0 & 1 & 0 & 1 & 0\\ 
      0 & 0 & 0 & 0 & 0 & 0 & 1 & 0 & 0 & 0 & 0 & 0 & 1 & 0 & 1\\ 
      0 & 0 & 0 & 0 & 0 & 0 & 0 & 1 & 0 & 0 & 0 & 1 & 0 & 1 & 1\\ 
      0 & 0 & 0 & 0 & 0 & 0 & 0 & 0 & 1 & 0 & 0 & 1 & 1 & 0 & 0\\ 
      0 & 0 & 0 & 0 & 0 & 0 & 0 & 0 & 0 & 1 & 0 & 0 & 1 & 1 & 0\\ 
      0 & 0 & 0 & 0 & 0 & 0 & 0 & 0 & 0 & 0 & 1 & 0 & 0 & 1 & 1
    \end{tabular}\right] = \\
  = 100110001111001.
\end{gather*}

Проверочная матрица в систематическом виде строится на основе матрицы
$G_{(n,k)}$, а именно:

\begin{equation*}
  H_{(n, k)} = \left[R^T_{(k, r)} I_r \right],
\end{equation*}

\begin{ESKDexplanation}
\item[где ]  $I_r$~--- единичная матрица, 
\item $R^T_{(k, r)}$~--- матрица $R_{(k, r)}$ из
$G_{(n,k)}$ в транспонированном виде. 
\end{ESKDexplanation}
  
В соответствии с матрицей G$_{(15,4)}$ получаем:

\begin{equation*}
  H_{(15, 11)} =  \left[
    \begin{tabular}{l*{15}{c}r}
      1 & 1 & 1 & 1 & 0 & 1 & 0 & 1 & 1 & 0 & 0 & 1 & 0 & 0 & 0\\
      0 & 1 & 1 & 1 & 1 & 0 & 1 & 0 & 1 & 1 & 0 & 0 & 1 & 0 & 0\\ 
      0 & 0 & 1 & 1 & 1 & 1 & 0 & 1 & 0 & 1 & 1 & 0 & 0 & 1 & 0\\ 
      1 & 1 & 1 & 0 & 1 & 0 & 1 & 1 & 0 & 0 & 1 & 0 & 0 & 0 & 1\\ 
    \end{tabular}
  \right].
\end{equation*}

Для определения синдрома необходимо умножить полученную кодовую
комбинацию на $H^T_{(n, r)}$. Полученный в результате умножения вектор
и будет являться синдромом, по которому можно судить о наличии и
расположении ошибки, или её отсутствии. Принятая кодовая комбинация
$V'(x) = 111110001000010$. Выполним умножение:

\begin{equation*}
  S(x) = V'(x) \times H^T_{(n, r)} = 111110001000010 \times
  \left[
    \begin{array}{cccc}
      1 & 0 & 0 & 1 \\
      1 & 1 & 0 & 1 \\
      1 & 1 & 1 & 1 \\
      1 & 1 & 1 & 0 \\
      0 & 1 & 1 & 1 \\
      1 & 0 & 1 & 0 \\
      0 & 1 & 0 & 1 \\
      1 & 0 & 1 & 1 \\
      1 & 1 & 0 & 0 \\
      0 & 1 & 1 & 0 \\
      0 & 0 & 1 & 1 \\
      1 & 0 & 0 & 0 \\
      0 & 1 & 0 & 0 \\
      0 & 0 & 1 & 0 \\
      0 & 0 & 0 & 1 
    \end{array}
  \right] = 1100.
\end{equation*}

В соответствии с этим синдромом определяем по матрице $H_{(15,4)}$,
что ошибка произошла в 12 разряде, следовательно, исправленная
комбинация будет 111110001001010.

\subsection{Ответ}

В результате решения были найдены образующий полином $g(x) = x^4 + x
+1$, разрешённая кодовая комбинация, соответствующая информационной
комбинации a(x)~--- V(x) = 100110001111001, была определена ошибка в
кодовой комбинации $V'(x) = 111110001000010$, после исправления была
получена исправленная кодовая комбинация~--- 111110001001010.



%%% Local Variables: 
%%% mode: latex
%%% TeX-master: "../TermWork_PDS"
%%% End: 
