Формула~(\ref{integ}) как пример длинной формулы с фигурной скобкой.
\begin{multline}
  S_{\text{вых}}(x_2, y_2) = \iint dx_0 dy_0 A_0 g(x_0, y_0) \cdot h(x_2-x_0, y_2 -y_0) = \\
  = A_0 \underbrace{\iint dx_0 dy_0 \; g(x_0, y_0) \cdot h(x_2-x_0,
    y_2 -y_0)}_{\text{\textit{по определению это есть свёртка} }} = A_0 g
  \otimes h
  \label{integ}
\end{multline}

\begin{gather*}
a x + b = 0 \\
a x^2 + b x + c = 0 \\
a x^3 + b x^2 + c x + d = 0
\end{gather*}

\begin{flalign*}
10xy^2+15x^2y-5xy & = 5\left(2xy^2+3x^2y-xy\right) = \\
   & = 5x\left(2y^2+3xy-y\right) = \\
   & = 5xy\left(2y+3x-1\right)
\end{flalign*}

\begin{flalign}
10xy^2+15x^2y-5xy & = 5\left(2xy^2+3x^2y-xy\right) = \\
   & = 5x\left(2y^2+3xy-y\right) = \\
   & = 5xy\left(2y+3x-1\right)
\end{flalign}

\begin{multline}
\left(1+x\right)^n = 1 + nx + \frac{n\left(n-1\right)}{2!}x^2 +\\
+ \frac{n\left(n-1\right)\left(n-2\right)}{3!}x^3 +\\
+ \frac{n\left(n-1\right)\left(n-2\right)\left(n-3\right)}{4!}x^4 + \dots
\end{multline}

\begin{equation}
A_{m,n} =
\begin{pmatrix}
a_{1,1} & a_{1,2} & \cdots & a_{1,n} \\
a_{2,1} & a_{2,2} & \cdots & a_{2,n} \\
\vdots & \vdots   & \ddots & \vdots  \\
a_{m,1} & a_{m,2} & \cdots & a_{m,n} \\
\end{pmatrix}
\end{equation}
\[ 
u(x) =
\begin{cases}
1 & \text{if } x \leqslant 0 \\
0 & \text{if } x \geqslant  0
\end{cases}
 \]

%%% Local Variables: 
%%% mode: latex
%%% TeX-master: "../TermWork"
%%% End:
