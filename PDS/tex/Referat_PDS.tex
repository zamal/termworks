\ESKDthisStyle{formII}
\begin{center}
  \Large{\textbf{РЕФЕРАТ}}
\end{center}

Пояснительная записка \ESKDtotal{page}~с., \ESKDtotal{figure}~рис.,
\ESKDtotal{table}~табл., \ESKDtotal{bibitem}~источников,
\ESKDtotal{appendix}~прил.

ПОМЕХОУСТОЙЧИВОЕ КОДИРОВАНИЕ, КОД ХЕММИНГА, КОД
БОУЗА"--~ЧОУДХУРИ"--~ХОКВИНГЕМА, КОД РИДА"--~СОЛОМОНА, OCTAVE

Объектом исследования являются помехоустойчивые коды.

Цель работы~--- решение задач по синтезу следующих помехоустойчивых
кодов: код Хемминга, код Боуза"--~Чоудхури"--~Хоквингема и код
Рида"--~Соломона.

В процессе выполнения курсовой работы были рассчитаны помехоустойчивые
коды в соответствии с заданием. Для расчётов в полях $GF(q)$ была
использована программа \textit{Octave}.

В результате исследования были получены навыки по расчёту
помехоустойчивых кодов с заданными характеристиками.
\newpage


%%% Local Variables: 
%%% mode: latex
%%% TeX-master: "../TermWork_PDS"
%%% End: 
