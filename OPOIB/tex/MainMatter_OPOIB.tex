\section{Общие положения}

\point Данный документ был создан в целях проверки коллекции
пакетов и классов eskdx.

\point Проверке подлежат:

\begin{itemize}
\item титульный лист;
\item оглавление;
\item рубрикация (разделы, пункты, приложения):
\item основные элементы текста (абзацы, перечни);
\item сноски;
\item формулы;
\item таблицы;
\item иллюстрации;
\item список литературы;
\end{itemize}

\section{Рубрикация}

\point Текст документа при необходимости разделяют на разделы и
подразделы.

\point Разделы должны иметь порядковые номера в пределах всего
документа (части, книги), обозначенные арабскими цифрами без точки
и записанные с абзацного отступа. Подразделы должны иметь
нумерацию в пределах каждого раздела. Номер подраздела состоит из
номеров раздела и подраздела, разделенных точкой. В конце номера
подраздела точка не ставится.  Разделы, как и подразделы, могут
состоять из одного или нескольких пунктов.

\point Разделы, подразделы должны иметь заголовки. Пункты, как
правило, заголовков не имеют.

\point Заголовки следует печатать с прописной буквы без точки в
конце, не подчеркивая. Переносы слов в заголовках не допускаются.
Если заголовок состоит из двух предложений, их разделяют точкой.

\point Расстояние между заголовком и текстом при выполнении
документа машинописным способом должно быть равно 3, 4 интервалам,
при выполнении рукописным способом "--- $15~\text{мм}$. Расстояние
между заголовками раздела и подраздела "--- 2 интервала, при
выполнении рукописным способом "--- $8~\text{мм}$.

\section{Основные элементы текста}

\point Абзацы в тексте начинают отступом, равным пяти ударам пишущей
машинки (от $15$ до $17~\text{мм}$).

Проверка абзаца, проверка абзаца, проверка, абзаца.

\point Внутри пунктов или подпунктов могут быть приведены
перечисления.  Перед каждой позицией перечисления следует ставить
дефис или при необходимости ссылки в тексте документа на одно из
перечислений, строчную букву, после которой ставится скобка. Для
дальнейшей детализации перечислений необходимо использовать
арабские цифры, после которых ставится скобка, а запись
производится с абзацного отступа.

\point Ненумерованное перечисление:

\begin{itemize}
\item первый элемент;
\item второй элемент;
\item третий элемент.
\end{itemize}

\point Нумерованное перечисление:

\begin{enumerate}
\item первый элемент;
\item второй элемент состоит из:
  \begin{enumerate}
  \item первого подэлемента;
  \item второго подэлемента;
  \end{enumerate}
\item третий элемент.
\end{enumerate}

\point Каждый пункт, подпункт и перечисление записывают с абзацного
отступа.

\section{Сноски}

\point Если необходимо пояснить отдельные данные, приведенные в
документе, то эти данные следует обозначать надстрочными знаками
сноски.

Сноски в тексте располагают с абзацного отступа в конце страницы, на
которой они обозначены, и отделяют от текста короткой тонкой
горизонтальной линией с левой стороны, а к данным, расположенным в
таблице, в конце таблицы над линией, обозначающей окончание таблицы.

\point Знак сноски ставят непосредственно после того слова, числа,
символа, предложения, к которому дается пояснение, и перед текстом
пояснения.

\point Знак сноски выполняют арабскими цифрами со скобкой и помещают
на уровне верхнего обреза шрифта.

Пример "--- ,,$\ldots$ печатающее устройство\footnote{текст сноски}$\ldots$''.

\section{Формулы}

\point В формулах в качестве символов следует применять
обозначения, установленные соответствующими государственными
стандартами. Пояснения символов и числовых коэффициентов, входящих
в формулу, если они не пояснены ранее в тексте, должны быть
приведены непосредственно под формулой. Пояснения каждого символа
следует давать с новой строки в той последовательности, в которой
символы приведены в формуле. Первая строка пояснения должна
начинаться со слова ``где'' без двоеточия после него.

Плотность каждого образца $\rho, \text{кг}/\text{м}^3$, вычисляют
по формуле

\begin{equation}
\label{eq:1}
\rho = \frac{m}{V},
\end{equation}

\begin{ESKDexplanation}
\item[где ] $m$ "--- масса образца, кг;
\item $V$ "--- объем образца, $\text{м}^3$.
\end{ESKDexplanation}

\point Формулы, за исключением формул, помещаемых в приложении,
должны нумероваться сквозной нумерацией арабскими цифрами, которые
записывают на уровне формулы справа в круглых скобках. Одну формулу
обозначают "--- (1).

Ссылки в тексте на порядковые номера формул дают в скобках,
например, в формуле \eqref{eq:1}.

\section{Таблицы}

Название таблицы следует помещать над таблицей. Проверка
таблицы~\ref{t:1}.

\begin{table}[b]
\caption{Заголовок таблицы}
\label{t:1}
\begin{tabular}{|c|c|c|c|}
\hline
\multicolumn{1}{|p{3.5cm}|}{Диаметр стержня крепежной детали, мм}&
\multicolumn{1}{p{3.5cm}|}{Масса $1000~\text{шт.}$ стальных шайб, кг}&
\multicolumn{1}{p{3.5cm}|}{Диаметр стержня крепежной детали, мм}&
\multicolumn{1}{p{3.5cm}|}{Масса $1000~\text{шт.}$ стальных шайб,
кг}\\\hline
$1{,}1$ & $0{,}045$ & $2{,}0$ & $0{,}192$\\\hline
$1{,}2$ & $0{,}043$ & $2{,}5$ & $0{,}350$\\\hline
$1{,}4$ & $0{,}111$ & $3{,}0$ & $0{,}553$\\\hline
\end{tabular}
\end{table}

Текст текст текст текст текст текст текст текст текст текст текст
текст текст текст текст текст текст текст текст текст текст текст
текст текст текст текст текст текст текст текст текст текст текст
текст текст текст текст текст текст текст текст текст текст текст
текст текст текст текст текст текст текст текст текст текст текст
текст текст текст текст текст текст текст текст текст текст текст.

\section{Иллюстрации}

Пример иллюстрации изображен на рисунке~\ref{f:1}.

\begin{figure}[t]
\begin{center}
\setlength{\unitlength}{50mm}
\begin{picture}(1,1)
\linethickness{\ESKDlineThin}
\put(0,0){\line(1,0){1}}
\put(1,0){\line(0,1){1}}
\put(0,1){\line(1,0){1}}
\put(0,0){\line(0,1){1}}
\put(0,0){\line(1,1){1}}
\put(1,0){\line(-1,1){1}}
\end{picture}
\end{center}
\caption{Перечеркнутый квадрат}
\label{f:1}
\end{figure}

Текст текст текст текст текст текст текст текст текст текст текст
текст текст текст текст текст текст текст текст текст текст текст
текст текст текст текст текст текст текст текст текст текст текст
текст текст текст текст текст текст текст текст текст текст текст
текст текст текст текст текст текст текст текст текст текст текст
текст текст текст текст текст текст текст текст текст текст текст
текст текст текст текст текст текст текст текст текст текст текст
текст текст текст текст текст текст текст текст текст текст текст
текст текст текст текст текст текст текст текст текст текст текст
текст текст текст текст текст текст текст текст текст текст текст
текст текст текст текст текст текст текст текст текст текст текст
текст текст текст текст текст текст текст текст текст текст текст.

%%% Local Variables: 
%%% mode: latex
%%% TeX-master: "../TermWork"
%%% End: 
