\newpage

\begin{center}
  \Large{\textbf{ВВЕДЕНИЕ}}
\end{center}
\addcontentsline{toc}{section}{Введение}

Информация~--- это имущество (или активы), которое, подобно другим
важным деловым активам, имеет ценность для организации и,
следовательно, должна быть защищена соответствующим
образом. Конфиденциальность, целостность и доступность информации
могут быть существенными аспектами для поддержания
конкурентоспособности, денежного оборота, доходности, юридической
гибкости и коммерческого имиджа. Информационная безопасность защищает
информацию от широкого диапазона угроз как раз для того, чтобы
обеспечить уверенность в непрерывности бизнеса, доведения до минимума
ущерба, наносимого бизнесу, и доведения до максимума возвращения
инвестиций и возможностей бизнеса.

Информационная безопасность достигается реализацией соответствующего
множества средств управления, которые могут быть представлены
политиками, методами, процедурами, организационными структурами и
функциями программного обеспечения. Эти средства управления необходимо
устанавливать таким образом, чтобы обеспечивать уверенность в том, что
определенные цели безопасности организации достигнуты~\cite{1}.

Данная курсовая работа посвящена анализу информационной системы
отделения Фонда социального страхования и по его результатам
разработана спецификация требований, призванных обеспечивать
руководство организации управлением и поддержкой информационной
безопасности, т. е. политика информационной безопасности.

Курсовая работа состоит из 3 разделов.

В первом разделе происходит идентификация объектов защиты в
информационной системе отделения фонда социального страхования.

Во втором разделе рассматриваются наиболее актуальные угрозы для
информационных ресурсов отделения фонда социального страхования.

Третий раздел посвящён разработке политике информационной безопасности
организации.

\newpage
