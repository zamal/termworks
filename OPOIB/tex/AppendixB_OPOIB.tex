\ESKDappendix{обязательное}{Оценка рисков реализации угроз
  информационным ресурсам отделения фонда социального страхования}
\label{AppendixB}

\begin{sidewaystable}[h]
\small
  \begin{longtable}{|p{2.5cm}|p{1.5cm}|p{2cm}|p{2cm}|p{1.5cm}|p{1.5cm}|p{1.5cm}|p{1.5cm}|p{3cm}|p{2cm}|p{2cm}|}
    \caption{Описание угроз}
    \label{tab:appb} \\
    \hline
    \multicolumn{11}{|c|}{Угроза внедрения в программу почтового сервера программы-вируса}\\
    \multicolumn{11}{|c|}{Место локализации угрозы: почтовый сервер}\\\hline
    Уязвимость & Возмож\-ность нападения & Метод нападения & Объект
    нападения & Тип потери & Масш\-таб ущерба & Источник угрозы & Опыт &
    Знание & Доступные ресурсы & Возможная мотивация действий\\\hline
    \multirow{2}{3cm}{Несвоевре\-менное обновление вирусных баз антивирусной программы,
    версий ПО; наличие уязвимости типа «buffer overflow»
    Зона локализации уязвимости: ПО, относящееся к среде
    эксплуатируемых систем} & Частая & Несанк\-циони\-рован\-ная пересылка
  вирусом отчетов посторонним лицам по Интернет  &  Отчёты о работе страхователей,
  полученные программой почтового сервера & Конфи\-ден\-циаль\-ность & Серь\-ёз\-ный &
  Програм\-мист-зло\-умыш\-лен\-ник, не имеющий отношения к фонду & Профес\-сиональ\-ный уровень
  & Деталь\-ное знание принципов работы антивирусных программ; высокий
  уровень знания языков программирования & Персональ\-ный компьютер,
  наличие подключения к Интернет & Умышлен\-ное причинение вреда в
  корыстных целях\\\cline{2-11}
  & Частая & Исполь\-зова\-ние имеющейся уязвимости для искажения информации в
  письмах, приходящих на почтовый сервер & Отчёты о работе
  страхователей, полученные программой почтового сервера & Целос\-тность
  & Серьё\-зный & Програм\-мист-злоу\-мышлен\-ник, не имеющий отношения к
  отделению фонда & Профес\-сиональ\-ный уровень & Детальное знание
  принципов работы антивирусных программ; высокий уровень знания
  языков программирования & Персональ\-ный компьютер, наличие
  подключения к Интернет & Умыш\-лен\-ное причинение вреда для повышения
  самооценки \\\hline
  \end{longtable}
\end{sidewaystable}
\newpage

\begin{sidewaystable}[h]
Продолжение таблицы~\ref{tab:appb}
\small
  \begin{longtable*}{|p{2.5cm}|p{1.5cm}|p{2cm}|p{2cm}|p{1.5cm}|p{1.5cm}|p{1.5cm}|p{1.5cm}|p{3cm}|p{2cm}|p{2cm}|}
    \hline
    \multicolumn{11}{|c|}{Угроза утечки аутентификационных данных
      сотрудников фонда}\\
    \multicolumn{11}{|c|}{Место локализации угрозы: почтовый сервер}\\\hline
    Уязвимость & Возмож\-ность нападения & Метод нападения & Объект
    нападения & Тип потери & Масш\-таб ущерба & Источник угрозы & Опыт &
    Знание & Доступные ресурсы & Возможная мотивация действий\\\hline
    Отсутствие средств мониторинга за действиями
    администраторов. Зона локализации уязвимости: программное
    обеспечение, относящееся к среде эксплуатируемых систем и
    персонал ИС & Мало\-вероят\-ная & Адми\-нист\-ратор ИБ решает опубликовать в
    интернете пароли для доступа к почтовым ящикам всех сотрудников фонда
    &  Отчёты о работе страхователей, полученные программой почтового
    сервера  & Конфи\-ден\-циаль\-ность & Серь\-ёз\-ный &
    Адми\-нист\-ратор ИБ & Профес\-сиональ\-ный уровень
    & Деталь\-ное знание принципов работы программы почтового сервера
    & Не обязательны & Умышлен\-ное причинение вреда в
    целях мести\\\hline
  \end{longtable*}
\end{sidewaystable}

\newpage

\begin{sidewaystable}[h]
Продолжение таблицы~\ref{tab:appb}
\small
  \begin{longtable*}{|p{2.5cm}|p{1.5cm}|p{2cm}|p{2cm}|p{1.5cm}|p{1.5cm}|p{1.5cm}|p{1.5cm}|p{3cm}|p{2cm}|p{2cm}|}
    \hline
    \multicolumn{11}{|c|}{Угроза атаки на SSH-сервер}\\
    \multicolumn{11}{|c|}{Место локализации угрозы: почтовый сервер}\\\hline
    Уязвимость & Возмож\-ность нападения & Метод нападения & Объект
    нападения & Тип потери & Масш\-таб ущерба & Источник угрозы & Опыт &
    Знание & Доступные ресурсы & Возможная мотивация действий\\\hline
    Некорректная настройка SSH-сервера. Зона локализации уязвимости: программное
    обеспечение, относящееся к среде эксплуатируемых систем   &
    Частая &  Несанк\-циони\-рован\-ное подключение к почтовому серверу
    через Интернет и изменение информации в письмах, хищение
    конфиденциальной информации
    &  Отчёты о работе страхователей, полученные программой почтового
    сервера  & Конфи\-ден\-циаль\-ность, целос\-тность & Серь\-ёз\-ный &
    Компью\-терный взломщик, пользователь сети Internet, не имеющий
    отношения к фонду & Не имеет значения
    & Хорошее знание принципов работы SSH-сервера
    & Персо\-наль\-ный компьютер, наличие подключения к Интернет  & Умышлен\-ное причинение вреда в
    для повышения самооценки\\\hline
  \end{longtable*}
\end{sidewaystable}


\newpage

\begin{sidewaystable}[h]
Продолжение таблицы~\ref{tab:appb}
\small
  \begin{longtable*}{|p{2.5cm}|p{1.5cm}|p{2cm}|p{2cm}|p{1.5cm}|p{1.5cm}|p{1.5cm}|p{1.5cm}|p{3cm}|p{2cm}|p{2cm}|}
    \hline
    \multicolumn{11}{|c|}{Угроза атаки на SSH-сервер}\\
    \multicolumn{11}{|c|}{Место локализации угрозы: почтовый сервер}\\\hline
    Уязвимость & Возмож\-ность нападения & Метод нападения & Объект
    нападения & Тип потери & Масш\-таб ущерба & Источник угрозы & Опыт &
    Знание & Доступные ресурсы & Возможная мотивация действий\\\hline
    Некорректная настройка SSH-сервера. Зона локализации уязвимости: программное
    обеспечение, относящееся к среде эксплуатируемых систем   &
    Частая &  Несанк\-циони\-рован\-ное подключение к почтовому серверу
    через Интернет и изменение информации в письмах, хищение
    конфиденциальной информации
    &  Отчёты о работе страхователей, полученные программой почтового
    сервера  & Конфи\-ден\-циаль\-ность, целос\-тность & Серь\-ёз\-ный &
    Компью\-терный взломщик, пользователь сети Интернет, не имеющий
    отношения к фонду & Не имеет значения
    & Хорошее знание принципов работы SSH-сервера
    & Персо\-наль\-ный компьютер, наличие подключения к Интернет  & Умышлен\-ное причинение вреда в
    для повышения самооценки\\\hline
  \end{longtable*}
\end{sidewaystable}

\newpage

\begin{sidewaystable}[h]
Продолжение таблицы~\ref{tab:appb}
\small
  \begin{longtable*}{|p{2.5cm}|p{1.5cm}|p{2cm}|p{2cm}|p{1.5cm}|p{1.5cm}|p{1.5cm}|p{1.5cm}|p{3cm}|p{2cm}|p{2cm}|}
    \hline
    \multicolumn{11}{|c|}{Угроза внедрения в ОС программы-вируса}\\
    \multicolumn{11}{|c|}{Место локализации угрозы: автоматизированные рабочие места и серверы}\\\hline
    Уязвимость & Возмож\-ность нападения & Метод нападения & Объект
    нападения & Тип потери & Масш\-таб ущерба & Источник угрозы & Опыт &
    Знание & Доступные ресурсы & Возможная мотивация действий\\\hline
    \multirow{3}{2.5cm} {Несвоевре\-менное обновление вирусных баз
    антивирусной программы. Зона локализации уязвимости:
    программное обеспечение, относящееся к среде эксплуатируемых систем}
    &
    Частая &  \multirow{2}{2cm}{Изменение или уничтожение вирусом системных и загрузочных файлов ОС}
    &   \multirow{2}{2cm}{Системные и загрузочные файлы операционных
      систем, под управлением которых находятся АРМ и серверы}   & Целост\-ность & Серь\-ёз\-ный &
    Созда\-тели антивирусной программы
    & Профес\-сиональ\-ный уровень
    & Детальное знание принципов работы антивирусных программ; высокий уровень знания языков программирования
    & Штат опытных программистов;  наличие подключения к Интернет  &
    Коммерчес\-кая выгода\\\cline{7-11}
    &&&&&&Програм\-мист-злоумыш\-ленник, не имеющий отношения к фонду
    & Профес\-сиональ\-ный уровень
    & Детальное знание принципов работы антивирусных программ; высокий
    уровень знания языков программирования & Персональ\-ный компьютер,
    наличие подключения к Интернет & Умышлен\-ное причинение вреда для повышения самооценки\\\cline{3-11}
    &&Получение НСД к информации путем использования программ типа
    <<spyware>> & Системные и загрузочные файлы операционных систем,
    под управлением которых находятся АРМ и серверы &
    Конфи\-ден\-циаль\-ность & Серьёз\-ный & Програм\-мист-злоумы\-шленник,
    не имеющий отношения к фонду & Профес\-сиональ\-ный уровень &
    Детальное знание принципов работы антивирусных программ; высокий
    уровень знания языков программирования & Персональ\-ный компьютер,
    наличие подключения к Интернет & Хищение информации для
    использования ее в корыстных целях\\\hline
  \end{longtable*}
\end{sidewaystable}

\begin{sidewaystable}[h]
Продолжение таблицы~\ref{tab:appb}
\small
  \begin{longtable}{|p{2.7cm}|p{1.5cm}|p{2cm}|p{2cm}|p{1.5cm}|p{1.5cm}|p{1.5cm}|p{1.5cm}|p{3cm}|p{2cm}|p{2cm}|}
    \hline
    \multicolumn{11}{|c|}{Угроза уничтожения или некорректной настройки критических файлов ОС администратором ИБ}\\
    \multicolumn{11}{|c|}{Место локализации угрозы: автоматизированные рабочие места и серверы}\\\hline
    Уязвимость & Возмож\-ность нападения & Метод нападения & Объект
    нападения & Тип потери & Масш\-таб ущерба & Источник угрозы & Опыт &
    Знание & Доступные ресурсы & Возможная мотивация действий\\\hline
    \multirow{2}{3cm}{Доступность редактир-ния системных и загрузочных файлов 
      (config.sys, boot.ini). Зона локализации уязвимости: персонал ИС
      и программное обеспечение, относящееся к среде эксплуатируемых
      систем} & Малове\-роятная & Некоррек\-тное редактирование
    системных и загрузочных файлов и некорректное использование
    механизма восстановления ОС   &  Системные и загрузочные файлы
    операционных систем, под управлением которых находятся АРМ и
    серверы  & Целост\-ность & Серь\-ёз\-ный &
  Админи\-стратор ИБ & Профес\-сиональ\-ный уровень
  & Детальное знание принципов работы антивирусных программ и сетевых
  приложений & Не обязательны & Умышлен\-ное причинение вреда в
  целях мести\\\cline{11-11}
  &&&&&&&&&& Неумыш\-ленное причинение вреда ввиду переутомления на
  рабочем месте \\\hline
  \end{longtable}
\end{sidewaystable}
\newpage

\newpage

\begin{sidewaystable}[h]
  \begin{longtable}{|p{2.7cm}|p{6cm}|p{0.7cm}|p{0.7cm}|p{0.7cm}|p{0.7cm}|p{0.7cm}|p{0.7cm}|p{0.7cm}|p{0.7cm}|p{0.7cm}|p{0.7cm}|p{0.7cm}|p{0.7cm}|}
    \caption{Матрица оценки рисков нарушения деятельности организации}
    \label{tab:matr_ocenki} \\
    \hline
    \multicolumn{2}{|p{8.7cm}|}{Угрозы по уровням инфраструктуры} &
    \multicolumn{3}{p{2.1cm}|}{Риск денежной потери} &
    \multicolumn{3}{p{2.1cm}|}{Риск потери производительности} &
    \multicolumn{3}{p{2.1cm}|}{Риск затруднения деятельности} &
    \multicolumn{3}{p{2.1cm}|}{Общий риск} \\\hline
    \multirow{4}{3cm}{Уровень сетевых прило\-жений} &
    Угроза внедрения в программу почтового сервера вредоносного кода
    &&&Н&В&&&В&&&В&&\\\cline{2-14}
    & Угроза утечки аутентификационных данных сотрудников фонда
    &&&Н&&&Н&В&&&&С&\\\cline{2-14}
    & Угроза атаки на SSH-сервер
    &&&Н&&С&&В&&&&С&\\\cline{2-14}
    & Угроза атаки на программу почтового сервера
    &&&Н&&&Н&В&&&&С&\\\hline
    \multirow{4}{3cm}{Уровень ОС} &
    Угроза внедрения в ОС программы-вируса
    &&С&&В&&&&&Н&В&&\\\cline{2-14}
    & Угроза уничтожения или некорректной настройки критических файлов
    ОС администратором ИБ
    &&С&&В&&&&&Н&В&&\\\hline
  \end{longtable}
\end{sidewaystable}

%%% Local Variables: 
%%% mode: latex
%%% TeX-master: "../TermWork_OPOIB"
%%% End: 
