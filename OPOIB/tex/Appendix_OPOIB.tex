\ESKDappendix{обязательное}{Приложение c текстом и формулами}

Формулы, помещаемые в приложениях, должны нумероваться отдельной
нумерацией арабскими цифрами в пределах каждого приложения с
добавлением перед каждой цифрой обозначения приложения, например
формула \eqref{tabwidth}.

Расчет ширины колонки таблицы можно выполнить по формуле:

\begin{equation}
\label{tabwidth}
W_\text{кол} = K \cdot W_\text{textwidth} - 2 \cdot W_\text{tabcolsep}
- N_\text{rule} \cdot W_\text{arrayrulewidth},
\end{equation}

\begin{ESKDexplanation}
\item[где ] $K$ "--- коэффициент ширины;
\item $W_\text{textwidth}$ "--- ширина текста на странице;
\item $W_\text{tabcolsep}$ "--- ширина промежутка между
вертикальной линией, отделяющей ячейку таблицы, и текстом ячейки;
\item $N_\text{rule}$ "--- число вертикальных линий отводимых на
ячейку (для внутренних ячеек $1$, для внешних $1{,}5$);
\item $W_\text{arrayrulewidth}$ "--- толщина вертикальной линии, что
отделяет ячейки.
\end{ESKDexplanation}

\ESKDappendix{обязательное}{Исходный код программы}

\lstinputlisting[caption = {seed.cpp}, label = {seed.cpp}]{./src/seed.cpp}

\newpage

%%% Local Variables: 
%%% mode: latex
%%% TeX-master: "../Term_Work"
%%% End: 
