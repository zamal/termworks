\documentclass[article, 12pt, russian, oneside]{ncc}

\usepackage[unicode, pdfstartview=FitH, urlcolor=cyan,%
linkcolor=red, citecolor=green, filecolor=magenta]{hyperref}
\usepackage[utf8]{inputenc}
\usepackage[russian]{babel}
\usepackage{setspace}
\usepackage{indentfirst}
\usepackage{pscyr}
\usepackage[headings]{ncchdr}
\usepackage{wrapfig}
\graphicspath{{pictures/}}
% \onehalfspacing
%\renewcommand{\rmdefault}{ftm}
\newcommand{\HRule}{\rule{\linewidth}{0.5mm}}
\indentaftersection
\sectionstyle{parindent}

%************************************************************
\hypersetup{
 pdftitle=Реферат,
 pdfauthor=Захаров М. А.,
 pdfsubject={Приоритетные национальные проекты},
 pdfkeywords={разное, черновик}
}
%************************************************************

\begin{document}
\thispagestyle{empty}
\begin{center}
  \textsc{\large ПЕНЗЕНСКИЙ ГОСУДАРСТВЕННЫЙ УНИВЕРСИТЕТ}\\[0.5cm]  
Кафедра <<Экономическая теория и мировая экономика>>\\[1.5cm]

% Upper part of the page
\includegraphics[width=0.4\textwidth]{./NPR}\\[1cm]
\textsc{\Large Реферат}\\[0.5cm]
% Title
\HRule \\[0.4cm]
{ \LARGE \bfseries Приоритетные национальные проекты}\\[0.4cm]
\HRule \\[1.5cm]
% Author and supervisor
\begin{minipage}{0.4\textwidth}
\begin{flushleft} \large
\emph{Выполнил:}\\
М.\,А. Захаров
\end{flushleft}
\end{minipage}
\begin{minipage}{0.4\textwidth}
\begin{flushright} \large
\emph{Проверил:} \\
О.\,В. Сальникова
\end{flushright}
\end{minipage}
\vfill
% Bottom of the page
{\large \today}
\end{center}
\newpage

%%% Local Variables: 
%%% mode: latex
%%% TeX-master: "main"
%%% End: 
 \thispagestyle{empty}
\tableofcontents
\newpage

\section*{Введение}
\addcontentsline{toc}{section}{Введение}

Приоритетные национальные проекты~--- программа по росту <<человеческого капитала>> в России, объявленная президентом В.\,В.~ Путиным и реализующаяся с 2006.

По сути стали стартовой площадкой для предвыборной гонки Дмитрия Медведева.

Национальные проекты (НП) стали для Дмитрия Медведева своеобразным трамплином для прыжка в президентское кресло. Во многом благодаря им он завоевал симпатии избирателей, а Владимир Путин таким образом устроил экзаменовку для преемника.
% Нация в проекте Какая судьба ожидает приоритетные национальные проекты http://versia.ru/articles/2008/mar/11/nats_proekti
\newpage

\section{Как зарождались национальные проекты}

\begin{quote}
  \emph{<<Концентрация бюджетных и административных ресурсов на
    повышении качества жизни граждан России~--- это необходимое и
    логичное развитие нашего с вами экономического курса, который мы
    проводили и будем проводить дальше. Проводили в течение предыдущих
    пяти лет и будем проводить дальше. Это гарантия от инертного
    проедания средств без ощутимой отдачи. Я уже говорил на одной из
    встреч: у нас не должно быть бюджетопроедания. Но это курс на
    инвестиции в человека, а значит, и в будущее России>>.}
\end{quote}
\begin{flushright}
  (Из выступления В.\,Путина~\cite{Putin_RG})
\end{flushright}

Повышение качества жизни граждан России~--- ключевой вопрос
государственной политики. Казалось бы, бесспорная декларация. Именно
так она воспринимается сейчас. В том числе~--- когда звучит в устах
власти. Но еще сравнительно недавний исторический опыт показывает, что
всего лишь несколько лет назад ее бесспорность вовсе не была столь
очевидной.

Опасная дезинтеграция государственных институтов, системный
экономический кризис, издержки приватизации в сочетании с
политическими спекуляциями на естественном стремлении людей к
демократии, серьезные просчеты при проведении экономических и
социальных реформ,~--- последнее десятилетие ТЕ века стало периодом
катастрофической демодернизации страны и социального упадка. За чертой
бедности оказалась фактически треть населения. Массовым явлением стали
многомесячные задержки с выплатой пенсий, пособий, заработных
плат. Люди были напуганы дефолтом, потерей в одночасье своих
сбережений. Не верили уже и в то, что государство сможет исполнять
даже минимальные социальные обязательства.

Вот с чем столкнулась власть, начавшая работать в 2000 году. Вот в
каких условиях необходимо было одновременно и решать острейшие
каждодневные проблемы, и работать на то, чтобы заложить новые~---
долгосрочные~--- тенденции роста.

Путь, пройденный страной за пять последних лет, очень коротко, но
наглядно описывается выдержками из ежегодных посланий Президента
России Федеральному Собранию. Вспомним, с какими словами Владимир
Путин обращался к нации в разные годы~\cite{NPR_Idea}.  \newline

\textbf{2000-й:}

\emph{<<Сегодня мы, прежде всего, ставим задачу наведения порядка в
  органах власти. Это~---лишь самый первый этап государственной
  модернизации. Соединение ресурсов федеральной, региональных и
  местных властей потребуется для решения других сложных
  задач>>}\cite{Putin_2000}.

\textbf{2001-й:}

\emph{<<Сегодня уже можно сказать: период ,,расползания``
  государственности позади. Дезинтеграция государства
  остановлена. 2000 год наглядно показал, что мы можем работать
  вместе, а теперь всем надо учиться работать эффективно>>}\cite{Putin_2001}.

\textbf{2002-й:}

\emph{<<Мы должны сделать Россию процветающей и зажиточной
  страной. Чтобы жить в ней было комфортно и безопасно. Чтобы люди
  могли свободно трудиться, без ограничений и страха зарабатывать для
  себя и для своих детей. И чтобы они стремились ехать в Россию, а не
  из нее. Воспитывать здесь своих детей. Строить здесь свой дом>>}\cite{Putin_2002}.

\textbf{2003-й:}

\emph{<<За трехлетний период мы не только основательно разобрали
  завалы проблем~--- а заниматься ими практически в ежедневном режиме
  нас заставляла сама жизнь,~--- но и добились некоторых положительных
  результатов. Сейчас надо сделать следующий шаг. И все наши решения,
  все наши действия~--- подчинить тому, чтобы уже в обозримом будущем
  Россия прочно заняла место среди действительно сильных, экономически
  передовых и влиятельных государств мира. Это~--- качественно новая
  задача. Качественно новая ступень для страны. Ступень, на которую мы
  раньше не могли подняться из-за целого ряда, из-за множества других,
  неотложных проблем. Такая возможность у нас есть. И мы обязаны ею
  воспользоваться>>}\cite{Putin_2003}.

\textbf{2004-й:}

\emph{<<Мы подошли к возможности развития высокими темпами, к
  возможности решения масштабных, общенациональных задач. И сейчас мы
  имеем и достаточный опыт, и необходимые инструменты, чтобы ставить
  перед собой действительно долгосрочные цели. Сегодня, впервые за
  долгое время, мы можем прогнозировать нашу жизнь не на несколько
  месяцев~--- даже не на год,~--- а на десятилетия вперед. И
  достижения последних лет дают нам основание приступить наконец к
  решению проблем, с которыми можно справиться, но можно
  справиться~--- только имея определенные экономические возможности,
  политическую стабильность и активное гражданское общество>>}\cite{Putin_2004}.

\textbf{2005-й:}

\emph{<<Прошу рассматривать прошлогоднее и нынешнее послания
  Федеральному Собранию как единую программу действий, как нашу
  совместную программу на ближайшее десятилетие>>}\cite{Putin_2005}.

Но необходим был еще один шаг. Успехи страны, достигнутую
макроэкономическую стабильность,~--- как обратить это к людям, чтобы
цифры экономического роста перестали быть абстрактными еще для очень
многих граждан России? И этот шаг был сделан.

5 сентября 2005 года Президент собрал вместе Правительство, парламент
и руководителей регионов.

Из выступления В.\,Путина~\cite{Putin_Gov}:

\emph{<<Сегодняшние возможности России вполне позволяют добиться более
  ощутимых результатов повышения благосостояния народа
  России. Добиться, не нарушая баланса основных экономических
  показателей и не допуская всплеска инфляции. И потому уже
  открывающиеся в российской экономике возможности не должны быть нами
  упущены.}
    
\emph{Сегодня хотел бы особо остановиться на практических шагах в
  реализации приоритетных национальных проектов в таких областях, как
  здравоохранение, образование, жилье.}

\emph{Во-первых, именно эти сферы определяют качество жизни людей и
  социальное самочувствие общества. И, во-вторых, в конечном счете,
  решение именно этих вопросов прямо влияет на демографическую
  ситуацию в стране и, что крайне важно, создает необходимые стартовые
  условия для развития так называемого человеческого капитала>>.}
\newpage


\section{Четыре приоритета}
\subsection{Здоровье}

\subsubsection{Основные проблемы}

России необходимо почти в два раза больше участковых врачей, чем работают сейчас.

В 2005 году высокотехнологичную медицинскую помощь получал лишь каждый четвертый нуждающийся в ней.

Система профилактики заболеваний требует кардинального улучшения.

Можно выделить следующие основные проблемы отечественной системы здравоохранения:

\begin{itemize}
\item В 2005 году укомплектованность поликлиник врачами составляла 56\%, коэффициент совместительства~--- 1,45, 30\% врачей участковой службы не проходили специализацию более 5 лет.
\item Износ медицинского оборудования, санитарного автотранспорта достигал 65\%.
\item Оснащенность медицинских учреждений диагностическим оборудованием была недостаточна, что значительно увеличивало срок ожидания диагностических исследований.
\item Удовлетворение потребности населения в высокотехнологичных видах медицинской помощи было низким. Финансирование оказания высокотехнологичных видов медицинской помощи составляло около 30\% от необходимого объема.
\item Недостаточное финансирование национального календаря прививок, в части вакцинации против гепатита В, краснухи и полиомиелита для детей группы риска.
\item Недостаточное финансирование мер по пропаганде здорового образа жизни~--- 15\% от потребности.
\end{itemize}

\subsubsection{Цели и задачи проекта}

Меры по решению основных проблем здравоохранения предполагают эффективное расходование бюджетных средств, ориентированное на конечный результат, смещение акцента оказания медицинской помощи в первичное звено (догоспитальный этап), профилактическую направленность здравоохранения.

Основные цели приоритетного национального проекта в сфере здравоохранения:

\begin{itemize}
\item Укрепление здоровья населения России, снижение уровня заболеваемости, инвалидности, смертности.
\item Повышение доступности и качества медицинской помощи.
\item Укрепление первичного звена здравоохранения, создание условий для оказания эффективной медицинской помощи на догоспитальном этапе.
\item Развитие профилактической направленности здравоохранения.
\item Удовлетворение потребности населения в высокотехнологичной медицинской помощи.
\end{itemize}

Задачи, решаемые в 2006--2007 годах:

\begin{itemize}
\item повышение уровня квалификации врачей участковой службы (увеличение количества врачей, прошедших подготовку, на 24 805 человек);
\item снижение коэффициента совместительства в учреждениях, оказывающих первичную медико-санитарную помощь до 1,1;
\item сокращение сроков ожидания диагностических исследований в поликлиниках до одной недели;
\item обновление парка санитарного автотранспорта службы скорой медицинской помощи на 12\,782 машины;
\item снижение числа заразившихся ВИЧ~--- инфекцией не менее чем на 1\,000 человек в год;
\item снижение заболеваемости~--- гепатитом В и С не менее чем в 3 раза, краснухой не менее чем в 10 раз, в том числе ликвидация врожденной краснухи, а также гриппом в период эпидемии и снижение выраженности его проявления у заболевших;
\item обследование не менее 95\% новорожденных детей с целью выявления наследственных заболеваний;
\item снижение материнской смертности до 24 случаев на 100 тысяч родившихся живыми, младенческой смертности до 10,1 случаев на 1\,000 родившихся живыми;
\item снижение у больных хроническими заболеваниями, по сравнению со случаями заболеваний в 2005 году, частоты обострений и осложнений не менее чем на 30\% и снижение временной нетрудоспособности населения не менее чем на 20\%;
\item повышение уровня обеспеченности населения высокотехнологичными видами медицинской помощи не менее чем до 45\% потребности.
\end{itemize}

\subsubsection{Ожидания от проекта}

Основные ожидаемые результаты проекта:

\begin{itemize}
\item повышение престижа труда медицинских работников первичного звена здравоохранения, в участковую службу должны прийти молодые квалифицированные специалисты; В 2006 году 1 914 молодых специалистов~--- врачей, окончивших интернатуру (1 457 чел.) и ординатуру (457 чел.)~--- пришли работать в первичное звено здравоохранения.
\item первичная медицинская помощь станет более доступной и качественной;
\item повысится квалификация участковых врачей (24 805 переподготовленных специалистов за два года);
\item амбулаторно-поликлинические учреждения будут оснащены необходимым диагностическим оборудованием, а значит, снизятся сроки ожидания диагностических исследований;
\item будут поставлены в регионы 12 782 новых автомобилей скорой медицинской помощи, вследствие чего повысится оперативность работы службы <<скорой помощи>>;
\item будет организована дополнительная бесплатная иммунизация населения;
\item будет организовано массовое обследование новорожденных детей на наследственные врожденные заболевания;
\item благодаря строительству новых медицинских центров снизятся сроки ожидания и повысится доступность высокотехнологичной медицинской помощи, особенно для детей и жителей сельских районов и отдаленных территорий;
\item будет обеспечена <<прозрачность>> очереди на получение высокотехнологичной медицинской помощи за счет введения системы <<листов ожидания (учета)>>.
\end{itemize}

\subsection{Жилье}

\subsubsection{Проблемы и решения}

Готовясь к реализации приоритетного национального проекта, мы постарались выделить основные проблемы российского жилищного рынка:

\begin{itemize}
\item Большинство людей нуждается в жилье, но не может себе позволить его покупку.
\item В России отсутствует эффективная система долгосрочного жилищного кредитования.
\item Нынешних объемов жилищного строительства не хватает для удовлетворения потребностей населения.
\item В стране не выработана эффективная схема реализации земельных участков и выделения земель под жилищное строительство.
\item В муниципальных образованиях отсутствуют схемы территориального планирования и градостроительная документация.
\item Качество жилищных и коммунальных услуг остается очень плохим, а уровень износа коммунальной инфраструктуры~--- высоким.
\item Социальное жилье и жилье для инвалидов, ветеранов и других категорий граждан выделяется очень низкими темпами.
\item Процедуры согласования строительной документации затруднены.
\item Граждане слабо защищены от махинаций при покупке и продаже жилья.
\end{itemize}

Для решения этих проблем Правительство Российской Федерации утвердило федеральную целевую программу <<Жилище>>, которая является базовым механизмом реализации национального проекта.

Особенность новой программы~--- ее сбалансированность. Государство не сможет способствовать созданию рынка доступного жилья, занимаясь каждым направлением в отдельности.

Обеспечивая сбалансированное стимулирование спроса и предложения на жилищном рынке, поддерживая строительство нового жилья рыночными механизмами, государство рассчитывает уже в течение ближайших пяти лет переломить ситуацию. К 2010 году объемы жилищного строительства должны увеличиться как минимум до 80 млн кв. м, а ставки по ипотечным кредитам уменьшатся до 8\% годовых. При этом каждый третий гражданин страны должен иметь возможность приобрести жилье с помощью собственных или заемных средств.

Кроме того, национальный проект предусматривает меры, которые раньше не применялись в рамках федеральных целевых программ. Внедряя рыночные механизмы, мы намерены продолжить работу и по совершенствованию жилищного законодательства. Правительство будет способствовать развитию массовой жилищной застройки, сокращать сроки согласования строительной документации и упрощать выделение земель под застройку; ликвидировать <<коррупционные схемы>> и локальные монополии; развивать малоэтажное домостроение и содействовать формированию рынка строительных материалов.


\subsubsection{Реализация}

Приоритетный национальный проект <<Доступное и комфортное жилье~--- гражданам России>> рассчитан на~6~лет.

На первом этапе (2006--2007~гг.) Президентом России определены четыре приоритета:

\begin{itemize}
\item Увеличение объемов ипотечного жилищного кредитования.
\item Повышение доступности жилья.
\item Увеличение объемов жилищного строительства и модернизация объектов коммунальной инфраструктуры.
\item Выполнение государственных обязательств по обеспечению жильем установленных категорий граждан.
\end{itemize}

Каждый из них реализуется по-своему.

Для реализации \emph{первого приоритета} предусмотрена государственная поддержка системы рефинансирования ипотечных жилищных кредитов и развитие инфраструктуры рынка ипотечных ценных бумаг. Это позволит увеличить объем ипотечного кредитования и снизить процентную ставку по кредитам.

В рамках \emph{второго приоритета} оказывается государственная поддержка молодым семьям в приобретении квартир или строительстве индивидуального жилья на собственные средства или с помощью ипотечных жилищных кредитов.

Для реализации \emph{третьего приоритета} государство будет сокращать административные барьеры, совершенствовать процедуры предоставления земельных участков под застройку, оказывать помощь в реализации крупных инвестиционных проектов, предоставит государственные гарантии по кредитам на обеспечение земельных участков инженерной инфраструктурой и будет субсидировать процентную ставку по таким кредитам. Кроме того, у муниципалитетов появится право залога муниципальных и неразграниченных земель для получения инвестиционных кредитов на строительство инженерной инфраструктуры.

В рамках \emph{четвертого приоритета} государство интенсифицирует выполнение своих обязательств перед ветеранами и инвалидами, муниципалитеты увеличат объемы предоставляемого социального жилья. Военнослужащим и некоторым другим категориям граждан будут предоставляться субсидии на приобретение жилья посредством реализации программы государственных жилищных сертификатов.

\subsubsection{Ожидания}

Что можно ожидать при успешной реализации проекта:

\begin{itemize}
\item Увеличить ежегодные объемы жилищного строительства в России до 80 млн~кв.~м.
\item Увеличить долю семей, которым будет доступно приобретение жилья (соответствующего стандартам), с 9 до 30\%.
\item Добиться существенного увеличения объемов ипотечного жилищного кредитования (до~415 млрд руб.).
\item Внедрить новые институты, позволяющие приобрести жилье в кредит, и совершенствовать механизм долевого строительства жилья.
\item Содействовать в приобретении или строительстве жилья 181,7~тыс. молодых семей.
\item Улучшить жилищные условия более 132,3~тыс. семей граждан, относящихся к категориям, установленным федеральным законодательством.
\item Совершенствовать нормативную правовую базу РФ.
\item Совершенствовать процедуры предоставления земельных участков под застройку.
\item Существенно сократить сроки согласования разрешительной документации на строительство жилья и государственной экспертизы.
\item Увеличить долю малоэтажного жилья в общем объеме строительства.
\item Обеспечить поддержку крупным инвестиционным проектам в области жилищного строительства.
\item Снизить среднюю продолжительность ожидания в очереди на улучшение жилищных условий с 20 до 5--7 лет.
\item Повысить уровень адресной поддержки населения, связанной с оплатой жилых помещений и коммунальных услуг.
\item Повысить качество коммунальных услуг, снизить уровень износа основных фондов предприятий ЖКХ с 60 до 50\%.
\end{itemize}

\subsection{Образование}

\subsubsection{Цели и задачи проекта}

Сегодня связь между современным, качественным образованием и перспективой построения гражданского общества, эффективной экономики и безопасного государства очевидна. Для страны, которая ориентируется на инновационный путь развития, жизненно важно дать системе образования стимул к движению вперед~--- это и есть первоочередная задача приоритетного национального проекта <<Образование>>.

Для реализации данной задачи в проекте предусмотрено несколько взаимодополняющих подходов. Во-первых, выявление и поддержка <<точек роста>>. Государство стимулирует учреждения и целые регионы, внедряющие инновационные программы и проекты, поощряет лучших учителей, выплачивает премии талантливой молодежи~--- то есть делает ставку на лидеров и содействует распространению их опыта. Государство поощряет тех, кто может и хочет работать,~--- это касается и учащихся школ, и студентов вузов, и преподавателей. Поддержку получают наиболее эффективные и востребованные образовательные практики~--- образцы качественного образования, обеспечивающего прогресс и профессиональный успех.

С другой стороны, ряд направлений проекта нацелен на обеспечение доступности, выравнивание условий получения образования: обеспечение для всех школ высокоскоростного доступа к глобальным информационным ресурсам, размещенным в сети Интернет, поставки учебного оборудования и школьных автобусов, организация образования для военнослужащих.

При этом проект предполагает внедрение новых управленческих механизмов. Создание в школах попечительских и управляющих советов, привлечение общественных организаций (советы ректоров, профсоюзы и~ т.\,д.) к управлению образованием~--- вот способы сделать образовательную систему более прозрачной и восприимчивой к запросам общества. Реализацию этого подхода обеспечивают конкурсные процедуры поддержки, предусмотренные в большинстве мероприятий проекта.

Кроме того, проект привносит значительные изменения в механизмы финансирования образовательных учреждений. Бюджетные средства на реализацию программ развития, как правило, направляются непосредственно в образовательные учреждения, что способствует развитию их финансовой самостоятельности. Распределение средств в общем образовании осуществляется на основе подушевого принципа финансирования с учетом объективных особенностей организации образования для городских и сельских учащихся. Принципы установления поощрений лучшим учителям и доплат за классное руководство задают основы для введения новой системы оплаты труда учителей, ориентированной на стимулирование качества и результативности педагогической работы.

\subsubsection{Основные направления}

\emph{Дополнительное вознаграждение за классное руководство}. Для осуществления дополнительных выплат классным руководителям общеобразовательных учреждений из федерального бюджета в бюджеты субъектов РФ перечисляются средства из расчета 1 тыс. руб. в месяц в классе наполняемостью 25 человек для городской местности и 14 и более человек для сельской местности, в классе с меньшей наполняемостью~--- с учетом уменьшения размера вознаграждения пропорционально численности обучающихся.

\emph{Поощрение лучших учителей}. Ежегодно начиная с 2006 года путем открытых региональных конкурсов на основе общественной экспертизы отбираются 10 тыс. учителей, достигших востребованного и признанного обществом уровня педагогической работы. Они получают денежное вознаграждение в размере 100 тыс. руб.

\emph{Стимулирование общеобразовательных учреждений}, внедряющих инновационные образовательные программы. Ежегодно 3 тыс. школ, отобранных в субъектах Российской Федерации на конкурсной основе, получают государственную поддержку в размере 1 млн руб. каждая.

\emph{Информатизация образования}. До конца 2007 года все российский школы, в том числе сельские, не имеющие доступа к сети Интернет, получат качественное подключение к глобальной сети со скоростью не менее 128Кбит/с.

\emph{ Оснащение российских школ учебным оборудованием}. В рамках этого направления в российские школы производится поставка интерактивных аппаратных комплексов, нового учебного и учебно-наглядного оборудования для кабинетов физики, химии, географии и биологии, а также аппаратно-программного комплекса (интерактивные доски). В 2007 году, объявленном Годом русского языка, этот перечень дополнен кабинетами русского языка и литературы.

\emph{Поставка школьных автобусов в сельские территории}. Для облегчения доставки сельских школьников к местам обучения проводится закупка школьных автобусов за счет средств федерального и региональных бюджетов.

\emph{Поддержка подготовки рабочих кадров и специалистов для высокотехнологичных производств} в учреждениях начального и среднего профессионального образования. В рамках этого направления в 2007 году на конкурсной основе государственная поддержка будет оказана 76 образовательным учреждениям начального и среднего профессионального образования (НПО и СПО). Каждое из учреждений-победителей получит от 20 до 30 млн руб. из федерального бюджета, в зависимости от объемов софинансирования, которое оно сможет привлечь.

\emph{Стимулирование учреждений высшего профессионального образования}, активно внедряющих инновационные образовательные программы. На конкурсной основе инновационным вузам предоставляются субсидии из федерального бюджета на закупку лабораторного оборудования, приобретение и разработку программного и методического обеспечения, повышение квалификации. В настоящее время по итогам федеральных конкурсов 2006 и 2007 годов производится государственная поддержка 57 инновационных вузов-победителей.

\emph{Формирование сети национальных университетов и бизнес-школ}. Цель создания национальных университетов~--- комплексное кадровое и научное обеспечение перспективного социально-экономического развития регионов. В 2006 году путем объединения региональных вузов было начато создание двух новых крупных университетов в Сибирском и Южном федеральных округах~--- Сибирского федерального университета и Южного федерального университета. Сверхзадача новых бизнес-школ~--- подготовка современных управленческих кадров мирового уровня. В рамках этого направления ведется работа по созданию двух бизнес-школ: Высшей школы менеджмента в Санкт-Петербурге на базе факультета менеджмента федерального государственного образовательного учреждения высшего профессионального образования <<Санкт-Петербургский государственный университет>> и Московской школы управления <<Сколково>>. С начала 2007 года в школах сформированы попечительские советы, избраны руководители (деканы), прошли выборы в ученый совет Высшей школы менеджмента СПбГУ.

\emph{Государственная поддержка талантливой молодежи}. Ежегодно начиная с 2006 года 1250 молодых людей~--- победителей международных и всероссийских олимпиад и конкурсов получают индивидуальные премии по 60 тыс. руб., а 4100 победителей и призеров всероссийских олимпиад и конкурсов, а также победителей региональных и межрегиональных олимпиад поощряются премиями в размере 30 тыс. руб.

\emph{Организация профессионального образования для военнослужащих}, проходящих военную службу по призыву и по контракту. В течение 2006--2007 годов в 24 воинских частях организованы учебные центры, в которых проводится эксперимент по расширению возможностей получения начального профессионального образования военнослужащими, проходящими службу по призыву. В его рамках военнослужащие во время службы по призыву помимо диплома по военно-учетной специальности образца Министерства обороны РФ смогут также получить дипломы государственного образца о начальном профессиональном образовании. Кроме того, в рамках данного направления в 2007 году за счет средств федерального бюджета будет организовано обучение на подготовительных отделениях вузов 5 тыс. человек, отслуживших не менее трех лет по контракту в Вооруженных Силах РФ в воинских должностях солдат, матросов, сержантов, старшин.

\emph{Государственная поддержка субъектов РФ, внедряющих комплексные проекты модернизации образования}. В 2007 году на конкурсной основе отобран 21 субъект РФ, который в 2007--2009 годах получит субсидии из федерального бюджета. Эти субсидии будут предназначены для введения новой системы оплаты труда работников образования, целями которой являются повышение доходов учителей, переход на нормативно-подушевое финансирование общеобразовательных учреждений, развитие региональной системы оценки качества образования, развитие сети общеобразовательных учреждений регионов и расширение общественного участия в управлении образованием.

\subsubsection{Ожидаемые результаты}

К концу 2006 года были достигнуты следующие основные показатели:

\begin{itemize}
\item На дополнительные выплаты классным руководителям было направлено 11,7 млрд руб. Ежемесячное вознаграждение получали около 900 тыс. педагогических работников.
\item 10 тыс. лучших учителей получили поощрения в размере 100 тыс. руб. каждый.
\item 3 тыс. инновационных общеобразовательных учреждений получили по 1 млн руб. и реализовали свои годовые программы развития.
\item Более 18 тыс. школ были подключены к сети Интернет.
\item За счет средств федерального бюджета было закуплено и поступило в российские школы 5113 комплектов учебного оборудования для кабинетов физики, химии, биологии и географии, а также интерактивных аппаратных комплексов.
\item Федеральным центром было закуплено и поставлено в регионы 1769 сельских школьных автобусов. Не мене 1750 единиц школьного автотранспорта регионы закупили за счет собственных средств.
\item 17 вузов~--- победителей конкурсного отбора получили в общей сложности 5 млрд руб. и приступили к реализации своих инновационных программ, рассчитанных на 2006--2007 годы.
\item Было создано два федеральных университета в Сибирском и Южном федеральных округах и проведена предварительная работа по созданию двух бизнес-школ международного уровня в Санкт-Петербурге и Москве.
\item 5350 представителей талантливой молодежи России получили премии из федерального бюджета.
\item В трех учебных центрах России начался эксперимент по расширению возможностей получения начального профессионального образования военнослужащими, проходящими службу по призыву.
\end{itemize}

К концу 2007 года:

\begin{itemize}
\item Дополнительные выплаты за классное руководство получат не менее 800 тыс. человек, на что в федеральном бюджете предусмотрено 11,7 млрд руб.
\item 10 тыс. лучших учителей~--- победителей конкурса 2007 года получат государственную поддержку в размере 100 тыс. руб.
\item 3 тыс. общеобразовательных учреждений на реализацию их инновационных программ будет перечислено по 1 млн руб.
\item Все российские школы, в том числе сельские, будут подключены к сети Интернет.
\item За счет средств федерального бюджета для общеобразовательных учреждений России будет закуплено не менее 9100 комплектов учебного и учебно-наглядного оборудования для кабинетов физики, химии, географии, биологии, русского языка и литературы, а также интерактивных аппаратных комплексов.
\item За счет средств федерального и региональных бюджетов будет закуплено не менее 3500 автобусов для сельских школ.
\item На конкурсной основе государственную поддержку получат 76 учреждений начального и среднего профессионального образования.
\item 17 вузов получат 5 млрд руб. и завершат свои программы развития, начатые в 2006 году. 40 инновационных вузов~--- победителей конкурса 2007 года получат в общей сложности 10 млрд руб. и приступят к реализации своих инновационных программ, рассчитанных на 2007-2008 годы.
\item Будет осуществлен первый набор студентов в Сибирский и Южный федеральные университеты, а также набор первых слушателей в Высшую школу менеджмента в Санкт-Петербурге.
\item Не менее 5350 представителей талантливой молодежи России получат премии.
\item В 21 учебном центре России начнется, а в трех центрах продолжится эксперимент по расширению возможностей получения начального профессионального образования военнослужащими, проходящими службу по призыву.
\item Будет организовано обучение на подготовительных отделениях вузов 5 тыс. человек, отслуживших не менее трех лет по контракту в Вооруженных Силах РФ в воинских должностях солдат, матросов, сержантов, старшин.
\item Будет на конкурсной основе оказана поддержка 21 субъекту РФ, внедряющему комплексные проекты модернизации образования, рассчитанные на 2007-2009 годы. Субъекты~--- победители конкурса получат средства федерального бюджета в объеме 4 млрд 54,7 млн руб., необходимые им в 2007 году, и приступят к реализации своих проектов.
\end{itemize}

Успешная реализация национального проекта позволит обеспечить системные изменения по основным направлениям развития образования России, а также будет эффективно содействовать становлению институтов гражданского общества.

\subsection{Развитие АПК}

\subsubsection{О проекте}

Приоритетный национальный проект <<Развитие АПК>> включает в себя три направления:

\begin{enumerate}
\item <<Ускоренное развитие животноводства>>;
\item <<Стимулирование развития малых форм хозяйствования>>;
\item <<Обеспечение доступным жильем молодых специалистов на селе>>.
\end{enumerate}

Реализация \emph{первого направления} Национального проекта позволит повысить рентабельность животноводства и промышленного рыбоводства, провести техническое перевооружение действующих животноводческих комплексов (ферм), предприятий промышленного рыбоводства (аквакультуры) и ввести в эксплуатацию новые мощности.

Это станет возможным за счет:

\begin{itemize}
\item повышения доступности долгосрочных кредитов, привлекаемых на срок до 5 и 8 лет;
\item роста поставок по системе федерального лизинга племенного скота, техники и оборудования для животноводства и промышленного рыбоводства благодаря увеличению уставного капитала ОАО <<Росагролизинг>>, снижению ставки за использование средств уставного капитала ОАО <<Росагролизинг>> и продлению срока лизинга техники и оборудования для животноводческих комплексов и предприятий промышленного рыбоводства (аквакультуры) до 10 лет;
\item совершенствования мер таможенно-тарифного регулирования путем утверждения объемов квот и таможенных пошлин на мясо в 2006-2007 годах и вплоть до 2009 года и отмены ввозных таможенных пошлин на технологическое оборудование для животноводства, не имеющее отечественных аналогов.
\end{itemize}

В октябре 2006 года решением Совета при Президенте РФ по реализации приоритетных национальных проектов в направление <<Ускоренное развитие животноводства>> включено пять дополнительных мероприятий.

В их числе: государственная поддержка племенного животноводства, отечественного овцеводства, северного оленеводства и табунного коневодства, развитие промышленного рыбоводства (аквакультуры), а также дополнение в связи с этим финансовой составляющей нацпроекта субсидированием 5-летних кредитов на развитие животноводства и аквакультуры.

\emph{Второе направление} Национального проекта направлено на увеличение объема реализации продукции, произведенной крестьянскими (фермерскими) хозяйствами и гражданами, ведущими личное подсобное хозяйство.

Это предполагается достичь путем:

\begin{itemize}
\item удешевления кредитных ресурсов, привлекаемых малыми формами хозяйствования АПК;
\item развития инфраструктуры обслуживания малых форм хозяйствования в АПК~--- сети сельскохозяйственных потребительских кооперативов (заготовительных, снабженческо-сбытовых, перерабатывающих, кредитных).
\item развития системы земельно-ипотечного кредитования, что позволит выдавать кредиты под залог земельных участков и поможет решить проблему отсутствия залоговой базы для малых форм хозяйствования в АПК.
\end{itemize}

Реализация третьего направления позволит обеспечить доступным жильем молодых специалистов на селе, создаст условия для формирования эффективного кадрового потенциала агропромышленного комплекса.

\subsubsection{Цели}

По направлению <<Ускоренное развитие животноводства>>:

\begin{itemize}
\item увеличение производства мяса на 7\%, молока на 4,5\% при стабилизации поголовья крупного рогатого скота (КРС), в том числе коров, не ниже уровня 2005 г.;
\item увеличение реализации молодняка племенных животных на 15\% к уровню 2006 г.
\item увеличение численности поголовья овец и коз на 3\% к уровню 2005 года.
\item увеличение численности поголовья к уровню 2006 г.: оленей на 3,2\%, лошадей на 2,8\% к уровню 2005 года.
\item увеличение выпуска товарной продукции аквакультуры на 4,0\% к уровню 2005 г.
\end{itemize}


По направлению <<Стимулирование развития малых форм хозяйствования>>:

\begin{itemize}
\item увеличение объемов реализации продукции, произведенной ЛПХ и КФХ к 2008 году на 6\% (относительно уровня 2005 года);
\item развитие сети сельскохозяйственных потребительских кооперативов (снабженческо-сбытовых, заготовительных, перерабатывающих, кредитных кооперативов);
\item создание системы земельно-ипотечного кредитования;
\end{itemize}

По направлению <<Обеспечение жильем молодых специалистов на селе>>:

\begin{itemize}
\item строительство (приобретение) 1392,9 тыс. кв. м жилья и улучшение жилищных условий не менее 31,64 тыс. молодых специалистов на селе.
\end{itemize}


\subsubsection{Мероприятия}

Основные мероприятия Национального проекта в 2006--2007 годах
предусматривают:

\emph{По направлению <<Ускоренное развитие животноводства>>:}

\begin{itemize}
\item создание условий для привлечения инвестиционных ресурсов,
  необходимых для развития животноводства и промышленного рыбоводства
  за счет выделения дополнительных бюджетных средств на субсидирование
  части затрат на уплату процентов по кредитам на срок до 8 лет,
  направленным на строительство и модернизацию животноводческих
  комплексов и предприятий промышленного рыбоводства, а также по
  кредитам на срок до 5 лет на приобретение племенного скота,
  племенного материала рыб, техники и оборудования для
  животноводческих комплексов (ферм) и предприятий промышленного
  рыбоводства. Кроме того, начиная с 2007 года, предусматривается
  субсидирование части затрат на уплату процентов по кредитам на срок
  до 1 года.
\item развитие лизинга племенного скота, оборудования для
  животноводства и промышленного рыбоводства;
\item повышение эффективности таможенно-тарифной политики;
\item субсидирование расходов на поддержку племенного животноводства,
  а также с 2007 года северного оленеводства, табунного коневодства и
  овцеводства.
\end{itemize}

\emph{По направлению <<Стимулирование развития малых форм
  хозяйствования>>}:

\begin{itemize}
\item создание условий для привлечения малыми формами хозяйствования в
  АПК кредитов и займов для улучшения их материально-технической базы
  за счет субсидирования части затрат на уплату процентов по кредитам
  и займам, полученным ЛПХ, КФХ и создаваемыми ими
  сельскохозяйственными потребительскими кооперативами в российских
  кредитных организациях и сельскохозяйственных кредитных
  потребительских кооперативах;
\item развитие сети сельскохозяйственных потребительских кооперативов,
  в том числе: кредитных, перерабатывающих, заготовительных и
  снабженческо-сбытовых;
\item развитие (создание) системы земельно-ипотечного кредитования,
  выдача кредитов под залог земельных участков.
\end{itemize}

\emph{По направлению <<Обеспечение жильем молодых специалистов (или их
  семей) на селе>>}:

\begin{itemize}
\item строительство (приобретение) жилья для молодых специалистов на
  селе.
\end{itemize}

\begin{thebibliography}{30}
\bibitem{Putin_RG} Это курс на инвестиции в человека, а значит, и в
  будущее России // Российская газета~--- Центральный выпуск~---
  2005. — \No3867.~--- URL:
  \url{http://www.rg.ru/2005/09/07/prezident-poslanie.html} (дата
  обращения: \today).

\bibitem{NPR_Idea} Инвестиции в человека [Электронный ресурс] //
  Официальный сайт Совета при Президенте России по реализации
  приоритетных национальных проектов и демографической политике. Дата
  обновления: 16.03.2006. URL:
  \url{http://www.rost.ru/main/what/01/01.shtml}

\bibitem{Putin_2000} Послание Федеральному Собранию Российской
  Федерации [Электронный ресурс] // Президент России. Дата обновления:
  08.07.2000.  URL:
  \url{http://archive.kremlin.ru/appears/2000/07/08/0000_type63372type63374type82634_28782.shtml}
  (дата обращения: \today).

\bibitem{Putin_2001}Послание Президента Российской Федерации
  В.\,В.~Путина Федеральному Собранию Российской Федерации
  [Электронный ресурс] // Интернет-портал Правительства Российской
  Федерации. 2001.  URL:
  \url{http://www.government.ru/content/31051180-3d4a-416a-a145-4c754f12f38b.htm}
  (дата обращения: \today).

\bibitem{Putin_2002} Послание Федеральному Собранию Российской
  Федерации [Электронный ресурс] // Президент России. Дата обновления:
  18.04.2002. URL:
  \url{http://archive.kremlin.ru/appears/2002/04/18/0001_type63372type63374type82634_28876.shtml}
  (дата обращения: \today).

\bibitem{Putin_2003} Послание Федеральному Собранию Российской
  Федерации [Электронный ресурс] // Президент России. Дата обновления:
  16.05.2003. URL:
  \url{http://archive.kremlin.ru/text/appears/2003/05/44623.shtml}
  (дата обращения: \today).

\bibitem{Putin_2004} Послание Федеральному Собранию Российской
  Федерации [Электронный ресурс] // Президент России. Дата обновления:
  26.05.2004. URL:
  \url{http://archive.kremlin.ru/text/appears/2004/05/71501.shtml}
  (дата обращения: \today).

\bibitem{Putin_2005} Послание Федеральному Собранию Российской
  Федерации [Электронный ресурс] // Президент России. 2005. Дата
  обновления: 25.04.2005. URL:
  \url{http://archive.kremlin.ru/text/appears/2005/04/87049.shtml}
  (дата обращения: \today).

\bibitem{Putin_Gov} Выступление на встрече с членами Правительства,
  руководством Федерального Собрания и членами президиума
  Государственного совета 5 сентября 2005 года [Электронный ресурс] // Президент
  России. Дата обновления: 05.09.2005. URL:
  \url{http://archive.kremlin.ru/appears/2005/09/05/1531_type63374type63378type82634_93296.shtml}
  (дата обращения: \today).
  
\end{thebibliography}


\end{document}

