\documentclass[article, 12pt, russian, oneside]{ncc}

\usepackage[unicode, pdfstartview=FitH, urlcolor=cyan,%
linkcolor=red, citecolor=green, filecolor=magenta]{hyperref}
\usepackage[utf8]{inputenc}
\usepackage[russian]{babel}
\usepackage{setspace}
\usepackage{indentfirst}
\usepackage{pscyr}
\usepackage[headings]{ncchdr}
\usepackage{wrapfig}
\graphicspath{{pictures/}}
% \onehalfspacing \renewcommand{\rmdefault}{ftm}
\newcommand{\HRule}{\rule{\linewidth}{0.5mm}}
\indentaftersection
\sectionstyle{parindent}
\makeatletter
\renewcommand{\@biblabel}[1]{#1.\hfil}
\makeatother
\addto\captionsrussian{\def\refname{Список использованных источников}}
% ************************************************************
\hypersetup{pdftitle=Реферат, pdfauthor=Захаров М. А.,
  pdfsubject={Приоритетные национальные проекты},
  pdfkeywords={экономика, нацпроекты} }
% ************************************************************
\usepackage{wrapfig}
\newcounter{rom}
\setcounter{rom}{20}
\begin{document}
\thispagestyle{empty}
\begin{center}
  \textsc{\large ПЕНЗЕНСКИЙ ГОСУДАРСТВЕННЫЙ УНИВЕРСИТЕТ}\\[0.5cm]  
Кафедра <<Экономическая теория и мировая экономика>>\\[1.5cm]

% Upper part of the page
\includegraphics[width=0.4\textwidth]{./NPR}\\[1cm]
\textsc{\Large Реферат}\\[0.5cm]
% Title
\HRule \\[0.4cm]
{ \LARGE \bfseries Приоритетные национальные проекты}\\[0.4cm]
\HRule \\[1.5cm]
% Author and supervisor
\begin{minipage}{0.4\textwidth}
\begin{flushleft} \large
\emph{Выполнил:}\\
М.\,А. Захаров
\end{flushleft}
\end{minipage}
\begin{minipage}{0.4\textwidth}
\begin{flushright} \large
\emph{Проверил:} \\
О.\,В. Сальникова
\end{flushright}
\end{minipage}
\vfill
% Bottom of the page
{\large \today}
\end{center}
\newpage

%%% Local Variables: 
%%% mode: latex
%%% TeX-master: "main"
%%% End: 
 \thispagestyle{empty}
\tableofcontents
\newpage

\noheadingtag
\section{Введение}

Приоритетные национальные проекты~--- программа по росту
<<человеческого капитала>> в России, объявленная экс-президентом
В.\,В.~Путиным и реализующаяся с 2006.

Появление национальных проектов было следствием во многом безуспешных
попыток реформирования социальной сферы в 1990-х~гг. и в первые годы
нынешнего десятилетия. На фоне появившихся в бюджете сверхдоходов от
продажи нефти и другого сырья уже было невозможно поддерживать на
крайне низком уровне финансирование образования, здравоохранения и
жилищной сферы.

Приоритетные направления <<инвестиций в человека>>:

\begin{itemize}
\item здравоохранение;
\item образование;
\item жильё;
\item сельское хозяйство.
\end{itemize}

Что касается образования, здравоохранения и жилья, то эти сферы были
определены сразу. В последний момент добавили развитие
агропромышленного комплекса, но забыли культурную сферу, о чем
Владимир Путин впоследствии неоднократно публично
сожалел~\cite{Spero}. После Президентского Послания 2006~г. фактически
в ранг национального проекта были сведены предложения по улучшению
демографической ситуации в стране.

Национальные проекты стали для Дмитрия Медведева своеобразным \linebreak
трамплином для прыжка в президентское кресло. Во многом благодаря им
он завоевал симпатии избирателей, а Владимир Путин таким образом
устроил экзаменовку для преемника. После выборов судьба приоритетных
проектов оставалась неясной. С одной стороны, официальные лица
заверяли, что курс по реализации этих программ будет не просто
продолжен, но за их реализацию возьмутся с удвоенной энергией. С
другой~--- стали появляться определённые симптомы, что их потихоньку
будут сворачивать, или же, согласно другому мнению, нацпроекты будут
тихо переформатированы под новые задачи, а само название постепенно
исчезнет из лексикона политиков~\cite{Versia}.

Однако, 20~марта~2008~г. Владимир Путин заявил, что нацпроекты
останутся неотъемлемой частью государственной политики вплоть до 2020
года, и он уже в качестве премьера возглавить работу в этом
направлении~\cite{Gazeta}.
% http://www.gazeta.ru/news/lenta/2008/03/20/n_1195164.shtml

На основе четырех нацпроектов на данный момент разработаны
соответствующие программы на 2009--2012 годы~\cite{PNP}.

В реферате рассматриваются политические истоки национальных проектов,
производится анализ структуры и хода реализации национальных проектов,
выделены основные достоинства и недостатки каждого проекта, подведены
промежуточные итоги из реализации.

\newpage

\section{Как зарождались национальные проекты}

\begin{quote}
  \emph{<<Концентрация бюджетных и административных ресурсов на
    повышении качества жизни граждан России~--- это необходимое и
    логичное развитие нашего с вами экономического курса, который мы
    проводили и будем проводить дальше. Проводили в течение предыдущих
    пяти лет и будем проводить дальше. Это гарантия от инертного
    проедания средств без ощутимой отдачи. Я уже говорил на одной из
    встреч: у нас не должно быть бюджетопроедания. Но это курс на
    инвестиции в человека, а значит, и в будущее России>>.}
\end{quote}
\begin{flushright}
  (Из выступления В.\,Путина~\cite{Putin_RG})
\end{flushright}

Повышение качества жизни граждан России~--- ключевой вопрос
государственной политики. Казалось бы, бесспорная декларация. Именно
так она воспринимается сейчас. В том числе~--- когда звучит в устах
власти. Но еще сравнительно недавний исторический опыт показывает, что
всего лишь несколько лет назад ее бесспорность вовсе не была столь
очевидной.

Опасная дезинтеграция государственных институтов, системный
экономический кризис, издержки приватизации в сочетании с
политическими спекуляциями на естественном стремлении людей к
демократии, серьезные просчеты при проведении экономических и
социальных реформ,~--- последнее десятилетие \Roman{rom} века стало
периодом катастрофической демодернизации страны и социального
упадка. За чертой бедности оказалась фактически треть
населения. Массовым явлением стали многомесячные задержки с выплатой
пенсий, пособий, заработных плат. Люди были напуганы дефолтом, потерей
в одночасье своих сбережений. Не верили уже и в то, что государство
сможет исполнять даже минимальные социальные обязательства.

Вот с чем столкнулась власть, начавшая работать в 2000 году. Вот в
каких условиях необходимо было одновременно и решать острейшие
каждодневные проблемы, и работать на то, чтобы заложить новые~---
долгосрочные~--- тенденции роста.

Путь, пройденный страной за пять последних лет, очень коротко, но
наглядно описывается выдержками из ежегодных посланий Президента
России Федеральному Собранию. Вспомним, с какими словами Владимир
Путин обращался к нации в разные годы~\cite{NPR_Idea}.

\textbf{2000-й:}

\emph{<<Сегодня мы, прежде всего, ставим задачу наведения порядка в
  органах власти. Это~---лишь самый первый этап государственной
  модернизации. Соединение ресурсов федеральной, региональных и
  местных властей потребуется для решения других сложных
  задач>>}\cite{Putin_2000}.

\textbf{2001-й:}

\emph{<<Сегодня уже можно сказать: период ,,расползания``
  государственности позади. Дезинтеграция государства
  остановлена. 2000 год наглядно показал, что мы можем работать
  вместе, а теперь всем надо учиться работать
  эффективно>>}\cite{Putin_2001}.

\textbf{2002-й:}

\emph{<<Мы должны сделать Россию процветающей и зажиточной
  страной. Чтобы жить в ней было комфортно и безопасно. Чтобы люди
  могли свободно трудиться, без ограничений и страха зарабатывать для
  себя и для своих детей. И чтобы они стремились ехать в Россию, а не
  из нее. Воспитывать здесь своих детей. Строить здесь свой
  дом>>}\cite{Putin_2002}.

\textbf{2003-й:}

\emph{<<За трехлетний период мы не только основательно разобрали
  завалы проблем~--- а заниматься ими практически в ежедневном режиме
  нас заставляла сама жизнь,~--- но и добились некоторых положительных
  результатов. Сейчас надо сделать следующий шаг. И все наши решения,
  все наши действия~--- подчинить тому, чтобы уже в обозримом будущем
  Россия прочно заняла место среди действительно сильных, экономически
  передовых и влиятельных государств мира. Это~--- качественно новая
  задача. Качественно новая ступень для страны. Ступень, на которую мы
  раньше не могли подняться из-за целого ряда, из-за множества других,
  неотложных проблем. Такая возможность у нас есть. И мы обязаны ею
  воспользоваться>>}\cite{Putin_2003}.

\textbf{2004-й:}

\emph{<<Мы подошли к возможности развития высокими темпами, к
  возможности решения масштабных, общенациональных задач. И сейчас мы
  имеем и достаточный опыт, и необходимые инструменты, чтобы ставить
  перед собой действительно долгосрочные цели. Сегодня, впервые за
  долгое время, мы можем прогнозировать нашу жизнь не на несколько
  месяцев~--- даже не на год,~--- а на десятилетия вперед. И
  достижения последних лет дают нам основание приступить наконец к
  решению проблем, с которыми можно справиться, но можно
  справиться~--- только имея определенные экономические возможности,
  политическую стабильность и активное гражданское
  общество>>}\cite{Putin_2004}.

\textbf{2005-й:}

\emph{<<Прошу рассматривать прошлогоднее и нынешнее послания
  Федеральному Собранию как единую программу действий, как нашу
  совместную программу на ближайшее десятилетие>>}\cite{Putin_2005}.

Но необходим был еще один шаг. Успехи страны, достигнутую
макроэкономическую стабильность,~--- как обратить это к людям, чтобы
цифры экономического роста перестали быть абстрактными еще для очень
многих граждан России? И этот шаг был сделан.

5 сентября 2005 года Президент собрал вместе Правительство, парламент
и руководителей регионов.

Из выступления В.\,Путина~\cite{Putin_Gov}:

\emph{<<Сегодняшние возможности России вполне позволяют добиться более
  ощутимых результатов повышения благосостояния народа
  России. Добиться, не нарушая баланса основных экономических
  показателей и не допуская всплеска инфляции. И потому уже
  открывающиеся в российской экономике возможности не должны быть нами
  упущены.}
    
\emph{Сегодня хотел бы особо остановиться на практических шагах в
  реализации приоритетных национальных проектов в таких областях, как
  здравоохранение, образование, жилье.}

\emph{Во-первых, именно эти сферы определяют качество жизни людей и
  социальное самочувствие общества. И, во-вторых, в конечном счете,
  решение именно этих вопросов прямо влияет на демографическую
  ситуацию в стране и, что крайне важно, создает необходимые стартовые
  условия для развития так называемого человеческого капитала>>.}
\newpage


\section{Здоровье}

\subsection{Основные проблемы}

\begin{wrapfigure}{r}{0.2\textwidth}
  \begin{center}
    \includegraphics[width=0.2\textwidth]{NPR_Health}
  \end{center}
\end{wrapfigure}

России необходимо почти в два раза больше участковых врачей, чем
работают сейчас~\cite{Health_Problems}.


В 2005 году высокотехнологичную медицинскую помощь получал лишь каждый
четвертый нуждающийся в ней.

Система профилактики заболеваний требует кардинального улучшения.


Можно выделить следующие основные проблемы отечественной системы
здравоохранения:

\begin{itemize}
\item В 2005 году укомплектованность поликлиник врачами составляла
  56\%, коэффициент совместительства~--- 1,45, 30\% врачей участковой
  службы не проходили специализацию более 5 лет.
\item Износ медицинского оборудования, санитарного автотранспорта
  достигал 65\%.
\item Оснащенность медицинских учреждений диагностическим
  оборудованием была недостаточна, что значительно увеличивало срок
  ожидания диагностических исследований.
\item Удовлетворение потребности населения в высокотехнологичных видах
  медицинской помощи было низким. Финансирование оказания
  высокотехнологичных видов медицинской помощи составляло около 30\%
  от необходимого объема.
\item Недостаточное финансирование национального календаря прививок, в
  части вакцинации против гепатита В, краснухи и полиомиелита для
  детей группы риска.
\item Недостаточное финансирование мер по пропаганде здорового образа
  жизни~--- 15\% от потребности.
\end{itemize}

\subsection{Цели и задачи проекта}

Меры по решению основных проблем здравоохранения предполагают
эффективное расходование бюджетных средств, ориентированное на
конечный результат, смещение акцента оказания медицинской помощи в
первичное звено (догоспитальный этап), профилактическую направленность
здравоохранения.

Основные цели приоритетного национального проекта в сфере
здравоохранения~\cite{Health_Goals}:

\begin{itemize}
\item Укрепление здоровья населения России, снижение уровня
  заболеваемости, инвалидности, смертности.
\item Повышение доступности и качества медицинской помощи.
\item Укрепление первичного звена здравоохранения, создание условий
  для оказания эффективной медицинской помощи на догоспитальном этапе.
\item Развитие профилактической направленности здравоохранения.
\item Удовлетворение потребности населения в высокотехнологичной
  медицинской помощи.
\item повышение уровня квалификации врачей участковой службы
  (увеличение количества врачей, прошедших подготовку, на 24\,805
  человек);
\item снижение коэффициента совместительства в учреждениях,
  оказывающих первичную медико-санитарную помощь до 1,1;
\item сокращение сроков ожидания диагностических исследований в
  поликлиниках до одной недели;
\item обновление парка санитарного автотранспорта службы скорой
  медицинской помощи на 12\,782 машины;
\item снижение числа заразившихся ВИЧ~--- инфекцией не менее чем на
  1\,000 человек в год;
\item снижение заболеваемости~--- гепатитом В и С не менее чем в 3
  раза, краснухой не менее чем в 10 раз, в том числе ликвидация
  врожденной краснухи, а также гриппом в период эпидемии и снижение
  выраженности его проявления у заболевших;
\item обследование не менее 95\% новорожденных детей с целью выявления
  наследственных заболеваний;
\item снижение материнской смертности до 24 случаев на 100 тысяч
  родившихся живыми, младенческой смертности до 10,1 случаев на 1\,000
  родившихся живыми;
\item снижение у больных хроническими заболеваниями, по сравнению со
  случаями заболеваний в 2005 году, частоты обострений и осложнений не
  менее чем на 30\% и снижение временной нетрудоспособности населения
  не менее чем на 20\%;
\item повышение уровня обеспеченности населения высокотехнологичными
  видами медицинской помощи не менее чем до 45\% потребности.
\end{itemize}

\subsection{Текущее положение дел}

Финансирование приоритетных национальных проектов осталось одним из
приоритетов при подготовке новой редакции федерального
бюджета\cite{Health_Waitings}.

Об этом сообщил Председатель Правительства Российской Федерации
Владимир Путин, открывая в среду 25 февраля очередное заседание
президиума Совета при Президенте РФ по реализации приоритетных
национальных проектов и демографической политике.

По мнению премьер-министра, этому принципу должны следовать и
региональные власти при внесении изменений в региональные и
муниципальные бюджеты.

<<Это необходимо сделать потому, что нацпроекты показали свою
востребованность, экономическую и социальную эффективность, –
подчеркнул Владимир Путин. --- Было бы ошибкой растратить сделанные
заделы>>.

Глава Правительства привел данные, что в результате реализации
нацпроектов ожидаемая продолжительность жизни в России увеличилась
почти на три года. В 2008 году родилось на 260 тыс. детей больше, чем
тремя годами ранее, это самый высокий показатель с 1992 года.

В 2009 году не только будет продолжена работа по ранее намеченным
направлениям национальных проектов, но появится ряд новых, сообщил
премьер. Важнейшее из них --- продвижение ценностей здорового образа
жизни, популяризация занятий физической культурой и спортом среди
детей и подростков, профилактика курения и алкоголизма в молодежной
среде.

<<Задача общества --- создать благоприятные условия, позволяющие нашим
детям гармонично развиваться, --- сказал глава Правительства. --- И
средства массовой информации, и деятели культуры, все родители,
общественность в целом должны озаботиться этой проблемой>>.

Формирование навыков здорового образа жизни у молодого поколения
должно стать одним из <<реально важнейших приоритетов>> деятельности
органов власти всех уровней, указал Владимир Путин.

Премьер-министр выделил ряд направлений, по которым необходимо
работать, чтобы поставленные задачи решить.

Во-первых, спортивные сооружения должны стать доступными для всех
детей и подростков, независимо от уровня доходов их родителей. Для
этого в том числе можно задействовать во внеурочное время школьные
спортивные залы и стадионы, предусмотрев материальные стимулы в
системах оплаты труда учителей физкультуры и спортивных организаторов
для повышения показателей здоровья учащихся, продвижения принципов
здорового образа жизни.

Во-вторых, активизировать решение проблемы качественного и
полноценного питания в школах, поскольку она тесно связана и с
укреплением здоровья детей в период обучения, и с формированием
навыков здорового образа жизни.

По словам Владимира Путина, в текущем году современные стандарты
организации питания планируется внедрить в школах 21 субъекта
Российской Федерации, в дальнейшем этот опыт должен распространиться и
на другие регионы.

Третья проблема --- соблюдение норм законов, запрещающих курение и
употребление алкоголя в образовательных учреждениях, на стадионах, в
других общественных местах, а также продажу спиртного и сигарет
детям. <<Нужно прямо сказать, соблюдаются эти запреты не очень-то
строго. А где-то вообще не соблюдаются. Так быть не должно>>, --- заявил
премьер.

Правоохранительные органы, региональные и местные власти, руководители
соответствующих учреждений должны принять жесткие меры по выполнению
норм закона. <<Нельзя наживаться на здоровье детей, за принятыми на
законодательном уровне мерами нужно следить строго>>,~--- считает
Владимир Путин.

В заключение глава Правительства подчеркнул, что вопросы здорового
образа жизни актуальны всегда, а <<в условиях кризисных явлений
подобные меры особенно важны>>.

<<Надо помочь семьям, помочь родителям, помочь самим детям. Особенно
это важно для тех городов и поселков, тех населенных пунктов, где
сегодня из-за спада производства возникают серьезные социальные
проблемы>>, --- заключил премьер.

\subsection{Мнения экспертов}

Основная проблема национального проекта <<Здоровье>> заключается в
том, что он по существу является латанием дыр или обновлением фасада,
в то время как всеми признается, что наше здравоохранение находится в
системном кризисе и нужны принципиальные меры для изменения ситуации.
Безусловно, дополнительное финансирование, предусмотренное в проекте,
полезно, но решение частных задач в лучшем случае только отсрочит (а
то и осложнит) решение общих вопросов. Плохо и то, что ход реализации
национального проекта <<Здоровье>> широко освещается, результаты
преподносятся в прессе как важные достижения, которые на самом деле
имеют разрозненный характер и принципиально не влияют на решение
основных проблем российского здравоохранения. Для этого нужна
постановка главной цели, достижению которой должны быть подчинены все
ресурсы. Предусмотренные в проекте меры обречены на то, чтобы иметь
несистемный характер, так как не создают соответствующих внутренних
механизмов к развитию~\cite{Spero}.
     
В национальный проект включены достаточно разноплановые меры. Если
закупки оборудования и реанимобилей могут быть отнесены к одноразовым
мероприятиям, то программа вакцинации должна быть долгосрочной по
определению. Это же касается и ряда других мер, предусмотренных
национальным проектом, например, скрининга новорожденных. В
Правительстве рассматривалась программа профилактики и лечения
отдельных инфекционных заболеваний, и в рамках этой программы
предполагается выделение довольно значительных денег на период до
2010--2012~гг. Однако не ясно, будет ли эта программа официальной
частью национального проекта или самостоятельным мероприятием.
     
Возможно, федеральные власти полагают, что финансирование большинства
мероприятий, заложенных в проекте, возьмут на себя регионы.  Во всяком
случае, расходы на национальный проект <<Здоровье>> в бюджете не
выделяются в отдельную статью, хотя в одном из комментариев к бюджету
расписано, сколько средств планируется потратить.
     
Кроме того, в официальных источниках отмечается, что национальный
проект в 2007~г. включил в себя мероприятия по апробированию новой
системы финансирования здравоохранения. В 19 регионах реализованы
пилотные проекты по введению новой отраслевой системы оплаты труда и
переходу к так называемой одноканальной форме финансирования
здравоохранения. Предполагается, что все источники (в разных регионах
их может быть от четырех до шести) сливаются в единый поток. Цель
эксперимента~--- обеспечить более оперативное и эффективное
использование средств. Однако непонятно, какие именно мероприятия
будут проведены по данному направлению.

Введение доплат врачам общей практики выявило и другую проблему
национального проекта: не учитывается долгосрочная перспектива в угоду
достижения сиюминутной цели. Очевидным последствием такого повышения
стал отток специалистов, которые в регионах теперь стремятся стать
врачами общей практики. В результате в ряде регионов не хватает врачей
и медсестер в больницах. Главная цель достигнута, но возникли новые
проблемы, в том числе и раскол между врачами. Кроме того, доплата
врачам первичного звена при ближайшем рассмотрении оказалась вовсе не
повышением заработной платы, а средством повышения интенсивности
труда, так как в рамках национального проекта <<Здоровье>> круг их
задач существенно расширился.

Высокотехнологичная помощь~--- также приоритет весьма неоднозначный. С
одной стороны, существует много проблем в организации основных видов
медицинской помощи, в том числе в реализации Программы государственных
гарантий отмечается недофинансирование основных видов медицинской
помощи в большинстве субъектов Российской Федерации и в первую очередь
скорой, амбулаторно-поликлинической и медицинской помощи, оказываемой
в дневных стационарах. С другой стороны, специалисты изыскивают
возможности оказания высокотехнологичной помощи в уже существующих
медицинских учреждениях.
\newpage

\section{Жилье}

\subsection{Проблемы и решения}

Готовясь к реализации приоритетного национального проекта, мы
постарались выделить основные проблемы российского жилищного
рынка~\cite{Hub_Problems}:

\begin{itemize}
\item Большинство людей нуждается в жилье, но не может себе позволить
  его покупку.
\item В России отсутствует эффективная система долгосрочного жилищного
  кредитования.
\item Нынешних объемов жилищного строительства не хватает для
  удовлетворения потребностей населения.
\item В стране не выработана эффективная схема реализации земельных
  участков и выделения земель под жилищное строительство.
\item В муниципальных образованиях отсутствуют схемы территориального
  планирования и градостроительная документация.
\item Качество жилищных и коммунальных услуг остается очень плохим, а
  уровень износа коммунальной инфраструктуры~--- высоким.
\item Социальное жилье и жилье для инвалидов, ветеранов и других
  категорий граждан выделяется очень низкими темпами.
\item Процедуры согласования строительной документации затруднены.
\item Граждане слабо защищены от махинаций при покупке и продаже
  жилья.
\end{itemize}

Для решения этих проблем Правительство Российской Федерации утвердило
федеральную целевую программу <<Жилище>>, которая является базовым
механизмом реализации национального проекта.

\begin{wrapfigure}{r}{0.2\textwidth}
  \begin{center}
    \includegraphics[width=0.2\textwidth]{NPR_Hab}
  \end{center}
\end{wrapfigure}

Особенность новой программы~--- ее сбалансированность. Государство не
сможет способствовать созданию рынка доступного жилья, занимаясь
каждым направлением в отдельности.

Обеспечивая сбалансированное стимулирование спроса и предложения на
жилищном рынке, поддерживая строительство нового жилья рыночными
механизмами, государство рассчитывает уже в течение ближайших пяти лет
переломить ситуацию. К 2010 году объемы жилищного строительства должны
увеличиться как минимум до 80 млн кв. м, а ставки по ипотечным
кредитам уменьшатся до 8\% годовых. При этом каждый третий гражданин
страны должен иметь возможность приобрести жилье с помощью собственных
или заемных средств.

Кроме того, национальный проект предусматривает меры, которые раньше
не применялись в рамках федеральных целевых программ. Внедряя рыночные
механизмы, мы намерены продолжить работу и по совершенствованию
жилищного законодательства. Правительство будет способствовать
развитию массовой жилищной застройки, сокращать сроки согласования
строительной документации и упрощать выделение земель под застройку;
ликвидировать <<коррупционные схемы>> и локальные монополии; развивать
малоэтажное домостроение и содействовать формированию рынка
строительных материалов.


\subsection{Реализация проекта на фоне кризиса}

Предпринимаемые государством меры по снижению процентных ставок по
ипотечным кредитам должны в ближайшее время привести к тому, что
ипотека подешевеет до докризисного уровня. Об этом в интервью НТВ
заявил Президент России Дмитрий Медведев.

Глава государства рассказал, что за последние годы объем ипотечного
кредитования в стране рос с высокой скоростью. Так, в 2006 году
россияне взяли порядка 260 млрд руб. ипотечных кредитов, в 2007 году~---
556 млрд руб., а в 2008 году~--- 633 млрд руб.

<<Мы существенно нарастили темпы, и, конечно, исключительно обидно
сейчас все было бы остановить, но мы этого делать не будем>>, -
подчеркнул Президент.

Он пояснил, что для выравнивания ситуации в системе ипотечного
кредитования принят целый ряд решений. В частности, на 20 млрд
руб. увеличен уставной капитал ОАО <<Агентство по ипотечному жилищному
кредитованию>>, и еще 40 млрд руб. АИЖК получит в виде кредитных
средств. Такая мера позволит снизить ставки по ипотеке и <<получить
относительно разумный по деньгам кредит>>.

По словам Дмитрия Медведева, сегодня средняя ставка по ипотечным
займам составляет 20\% годовых. <<Ни один нормальный наш человек за
такие деньги ипотеку брать не будет>>, - считает Президент. В
докризисный период банки предлагали ипотечные кредиты под
11--12\%. Увеличение финансирования АИЖК позволит снизить ставки до
14--15\%, а региональные программы развития системы ипотечного
кредитования~--- еще на 2--3\%. При этом эффект от предпринятых шагов
должен быть виден в ближайшее время, <<потому что деньги на это
выделены>>.

Кроме того, как рассказал Президент, за последние три года существенно
возросли объемы жилищного строительства. <<Начинали где-то с 50 млн
кв. м жилья в год, в прошлом году уже вышли почти на 64 млн кв.м>>, -
уточнил Дмитрий Медведев. Он подчеркнул, что <<задача и была, и
остается строить в конечном счете не меньше, чем 1 кв.м жилья в
расчете на одного гражданина России в год>>.

<<Это макрозадача, сейчас мы к ней не готовы, но ее никто не
снимает>>,~---подытожил глава государства.

Помимо этого, стоит задача по обеспечению жильем отдельных категорий
граждан. <<Мы эти программы начали, и мы их завершать не будем,
несмотря на кризис>>, - заявил Дмитрий Медведев.

Речь идет, прежде всего, о строительстве жилья для военнослужащих, а
также о предоставлении благоустроенных квартир ветеранам Великой
Отечественной войны к 9 мая 2010 года.

<<Мы должны обеспечить всех наших ветеранов комфортным жильем. Эту
задачу мы не снимали, и мы ее исполним>>,~--- заверил Президент.

\subsection{Ожидания}

Что можно ожидать при успешной реализации проекта~\cite{Hub_Waitings}:

\begin{itemize}
\item Увеличить ежегодные объемы жилищного строительства в России до
  80 млн~кв.~м.
\item Увеличить долю семей, которым будет доступно приобретение жилья
  (соответствующего стандартам), с 9 до 30\%.
\item Добиться существенного увеличения объемов ипотечного жилищного
  кредитования (до~415 млрд руб.).
\item Внедрить новые институты, позволяющие приобрести жилье в кредит,
  и совершенствовать механизм долевого строительства жилья.
\item Содействовать в приобретении или строительстве жилья
  181,7~тыс. молодых семей.
\item Улучшить жилищные условия более 132,3~тыс. семей граждан,
  относящихся к категориям, установленным федеральным
  законодательством.
\item Совершенствовать нормативную правовую базу РФ.
\item Совершенствовать процедуры предоставления земельных участков под
  застройку.
\item Существенно сократить сроки согласования разрешительной
  документации на строительство жилья и государственной экспертизы.
\item Увеличить долю малоэтажного жилья в общем объеме строительства.
\item Обеспечить поддержку крупным инвестиционным проектам в области
  жилищного строительства.
\item Снизить среднюю продолжительность ожидания в очереди на
  улучшение жилищных условий с 20 до 5--7 лет.
\item Повысить уровень адресной поддержки населения, связанной с
  оплатой жилых помещений и коммунальных услуг.
\item Повысить качество коммунальных услуг, снизить уровень износа
  основных фондов предприятий ЖКХ с 60 до 50\%.
\end{itemize}

\subsection{Результаты}

Из всех национальных проектов именно этот нуждается в самых крупных и
долгосрочных затратах, требующих масштабных финансовых ресурсов, и
именно этот проект вызывает наибольшее количество негативных оценок.

Законодательной базой проекта является пакет из 27 законов, принятых в
основном в конце 2004~г., в том числе Жилищный и Градостроительный
кодексы, а главным механизмом реализации выступает обновленная
Федеральная целевая программа <<Жилище>> на 2002--2010~гг., в которой,
в отличие от самого проекта, названы сроки, показатели, определена
ответственность за выполнение положений программы.

Основной акцент в нацпроекте, который, как и другие национальные
проекты, разрабатывался в закрытом режиме, без широкого обсуждения,
сделан на преимущественном развитии (при существенной поддержке
государства) рыночных механизмов приобретения жилья, вопросы
строительства муниципального жилья некоммерческого использования или
формирования сектора арендуемого жилья в проект не вошли. На второй
план отошли проблемы ремонта и восстановления уже существующего жилого
фонда. Однако острота данной проблемы оказалась настолько большой, что
потребовала дополнительных решений, на что было указано в Послании
Федеральному собранию Президента России. Удельный вес непригодного для
проживания жилья продолжает увеличиваться. За последние 5 лет объем
ветхого и аварийного жилья удвоился и превысил 100 млн квадратных
метров. Задолженность бюджетов всех уровней по непроизведенному
капитальному ремонту, по официальным оценкам, превышает 3 трлн
рублей. По официальным подсчетам Федерального агентства по
строительству и жилищно-коммунальному хозяйству, ежегодные расходы по
модернизации ветхого жилья должны составлять более 70~млрд~рублей (в
настоящее время выделяется в 5 раз меньше). Такое положение снижает
возможности для улучшения жилищных условий, кроме того, территории
развития нового строительства в основном не совпадают с географией
концентрации ветхого и аварийного жилья.
            
Ипотечное кредитование~--- наиболее динамично развивающаяся часть
программы. За 2006~г. в стране выдано ипотечных кредитов на сумму 240
млрд рублей, что в 2 раза превышает целевой показатель программы на
2006~г.  За первые 6 месяцев 2007~г. объем выданных ипотечных кредитов
увеличился на 60\% и превысил 200 млрд рублей.

Тем не менее, несмотря на рост предложения банковских ипотечных
продуктов, для отечественного рынка ипотеки доступен пока
стандартизированный продукт, способный удовлетворить потребности очень
узкой группы заемщиков. На зарубежном ипотечном рынке действует
несравнимо более широкий диапазон кредитных продуктов, учитывающих не
только предмет залога, но и возраст, карьерный рост заемщика, прогнозы
по росту стоимости объекта и~др., особенно для молодых семей.
                                                                                           
Приходится констатировать, что субсидии гражданам в форме льготных
кредитов или субсидирования процентных ставок ложатся тяжелым бременем
на местные бюджеты и не всегда доходят до групп населения, в
наибольшей степени нуждающихся в поддержке, коммерческие банки в этих
программах не играют активной роли субъектов рынка, а выступают только
как уполномоченные посредники.

Первые итоги реализации нацпроекта показали, что программа в
современном виде может привлечь лишь очень ограниченный слой населения
и нуждается в серьезной корректировке и смене акцентов. Ипотечное
кредитование развивается быстрыми темпами, но основная масса кредитов
выдается в регионах с высокими средними доходами населения, 90\%
вводимого в эксплуатацию жилья достается 14--15\% населения — самым
обеспеченным и верхней прослойке среднего класса. Доступность жилья
снижается и за счет опережающего роста цен по сравнению с ростом
реальных доходов и вследствие вымывания с рынка дешевого
жилья. Участники социальных программ жилищных сертификатов и поддержки
молодых семей в обеспечении жильем сталкиваются с тем, что реальная
цена имеющихся на рынке квартир на порядок превышает расчетные
показатели. Общее число участников программ оказывается ниже даже
весьма скромных запланированных показателей.

Реализация нацпроекта привела к значительной активизации строительства
во всех областях~--- от малоэтажного строительства до комплексной
застройки отдельных районов, но его темпы еще далеки от современного
европейского стандарта~--- 1 кв. м. на душу населения. Собственно
строительный рынок пока отстает от ипотечного.

Ключевой момент реализации нацпроекта связан с тем, что его основные
задачи возложены на регионы и муниципалитеты. Движение жилищного
вопроса, обеспечение коммунальной инфраструктурой, ремонт аварийного
жилья, программы поддержки молодых семей зависят от местных властей.
Общий недостаток этих~--- программ их небольшие масштабы из-за
ограниченности местных и региональных бюджетов. Запланированные
крупномасштабные проекты~--- в основном проекты строительства
дорогостоящего жилья. Стимулирование строительства типового жилья,
снижение цен до уровня себестоимости и разумной рентабельности наряду
с экономическими мерами (льготное выделение земли под строительство,
госбюджетная поддержка проектов создания инфраструктуры) должно
осуществляться и иными способами~--- от законодательных до
административных. Застройщики должны получить гарантированные права
долгосрочного владения земельными участками, инвесторы~--- правовую
информацию о потенциальных вложениях в обустройство недвижимости,
земельные участки для застройки должны предоставляться на основе
определенной типовой процедуры.
\newpage

\section{Образование}

\subsection{Цели и задачи проекта}

Сегодня связь между современным, качественным образованием и
перспективой построения гражданского общества, эффективной экономики и
безопасного государства очевидна. Для страны, которая ориентируется на
инновационный путь развития, жизненно важно дать системе образования
стимул к движению вперед~--- это и есть первоочередная задача
приоритетного национального проекта
<<Образование>>~\cite{Edu_Problems}.

\begin{wrapfigure}{r}{0.2\textwidth}
  \begin{center}
    \includegraphics[width=0.2\textwidth]{NPR_Edu}
  \end{center}
\end{wrapfigure}

Для реализации данной задачи в проекте предусмотрено несколько
взаимодополняющих подходов. Во-первых, выявление и поддержка <<точек
роста>>. Государство стимулирует учреждения и целые регионы,
внедряющие инновационные программы и проекты, поощряет лучших
учителей, выплачивает премии талантливой молодежи~--- то есть делает
ставку на лидеров и содействует распространению их опыта. Государство
поощряет тех, кто может и хочет работать,~--- это касается и учащихся
школ, и студентов вузов, и преподавателей. Поддержку получают наиболее
эффективные и востребованные образовательные практики~--- образцы
качественного образования, обеспечивающего прогресс и профессиональный
успех.

С другой стороны, ряд направлений проекта нацелен на обеспечение
доступности, выравнивание условий получения образования: обеспечение
для всех школ высокоскоростного доступа к глобальным информационным
ресурсам, размещенным в сети Интернет, поставки учебного оборудования
и школьных автобусов, организация образования для военнослужащих.

При этом проект предполагает внедрение новых управленческих
механизмов. Создание в школах попечительских и управляющих советов,
привлечение общественных организаций (советы ректоров, профсоюзы и~
т.\,д.) к управлению образованием~--- вот способы сделать
образовательную систему более прозрачной и восприимчивой к запросам
общества. Реализацию этого подхода обеспечивают конкурсные процедуры
поддержки, предусмотренные в большинстве мероприятий проекта.

Кроме того, проект привносит значительные изменения в механизмы
финансирования образовательных учреждений. Бюджетные средства на
реализацию программ развития, как правило, направляются
непосредственно в образовательные учреждения, что способствует
развитию их финансовой самостоятельности. Распределение средств в
общем образовании осуществляется на основе подушевого принципа
финансирования с учетом объективных особенностей организации
образования для городских и сельских учащихся. Принципы установления
поощрений лучшим учителям и доплат за классное руководство задают
основы для введения новой системы оплаты труда учителей,
ориентированной на стимулирование качества и результативности
педагогической работы.

\subsection{Основные направления}

\emph{Дополнительное вознаграждение за классное руководство}. Для
осуществления дополнительных выплат классным руководителям
общеобразовательных учреждений из федерального бюджета в бюджеты
субъектов РФ перечисляются средства из расчета 1 тыс. руб. в месяц в
классе наполняемостью 25 человек для городской местности и 14 и более
человек для сельской местности, в классе с меньшей наполняемостью~---
с учетом уменьшения размера вознаграждения пропорционально численности
обучающихся~\cite{Edu_Goals}.

\emph{Поощрение лучших учителей}. Ежегодно начиная с 2006 года путем
открытых региональных конкурсов на основе общественной экспертизы
отбираются 10 тыс. учителей, достигших востребованного и признанного
обществом уровня педагогической работы. Они получают денежное
вознаграждение в размере 100 тыс. руб.

\emph{Стимулирование общеобразовательных учреждений}, внедряющих
инновационные образовательные программы. Ежегодно 3 тыс. школ,
отобранных в субъектах Российской Федерации на конкурсной основе,
получают государственную поддержку в размере 1 млн руб. каждая.

\emph{Информатизация образования}. До конца 2007 года все российский
школы, в том числе сельские, не имеющие доступа к сети Интернет,
получат качественное подключение к глобальной сети со скоростью не
менее 128Кбит/с.

\emph{Оснащение российских школ учебным оборудованием}. В рамках
этого направления в российские школы производится поставка
интерактивных аппаратных комплексов, нового учебного и
учебно-наглядного оборудования для кабинетов физики, химии, географии
и биологии, а также аппаратно-программного комплекса (интерактивные
доски). В 2007 году, объявленном Годом русского языка, этот перечень
дополнен кабинетами русского языка и литературы.

\emph{Поставка школьных автобусов в сельские территории}. Для
облегчения доставки сельских школьников к местам обучения проводится
закупка школьных автобусов за счет средств федерального и региональных
бюджетов.

\emph{Поддержка подготовки рабочих кадров и специалистов для
  высокотехнологичных производств} в учреждениях начального и среднего
профессионального образования. В рамках этого направления в 2007 году
на конкурсной основе государственная поддержка будет оказана 76
образовательным учреждениям начального и среднего профессионального
образования (НПО и СПО). Каждое из учреждений-победителей получит от
20 до 30 млн руб. из федерального бюджета, в зависимости от объемов
софинансирования, которое оно сможет привлечь.

\emph{Стимулирование учреждений высшего профессионального
  образования}, активно внедряющих инновационные образовательные
программы. На конкурсной основе инновационным вузам предоставляются
субсидии из федерального бюджета на закупку лабораторного
оборудования, приобретение и разработку программного и методического
обеспечения, повышение квалификации. В настоящее время по итогам
федеральных конкурсов 2006 и 2007 годов производится государственная
поддержка 57 инновационных вузов-победителей.

\emph{Формирование сети национальных университетов и
  бизнес-школ}. Цель создания национальных университетов~---
комплексное кадровое и научное обеспечение перспективного
социально-экономического развития регионов. В 2006 году путем
объединения региональных вузов было начато создание двух новых крупных
университетов в Сибирском и Южном федеральных округах~--- Сибирского
федерального университета и Южного федерального
университета. Сверхзадача новых бизнес-школ~--- подготовка современных
управленческих кадров мирового уровня. В рамках этого направления
ведется работа по созданию двух бизнес-школ: Высшей школы менеджмента
в Санкт-Петербурге на базе факультета менеджмента федерального
государственного образовательного учреждения высшего профессионального
образования <<Санкт-Петербургский государственный университет>> и
Московской школы управления <<Сколково>>. С начала 2007 года в школах
сформированы попечительские советы, избраны руководители (деканы),
прошли выборы в ученый совет Высшей школы менеджмента СПбГУ.

\emph{Государственная поддержка талантливой молодежи}. Ежегодно
начиная с 2006 года 1250 молодых людей~--- победителей международных и
всероссийских олимпиад и конкурсов получают индивидуальные премии по
60 тыс. руб., а 4100 победителей и призеров всероссийских олимпиад и
конкурсов, а также победителей региональных и межрегиональных олимпиад
поощряются премиями в размере 30 тыс. руб.

\emph{Организация профессионального образования для военнослужащих},\linebreak
проходящих военную службу по призыву и по контракту. В течение
2006--2007 годов в 24 воинских частях организованы учебные центры, в
которых проводится эксперимент по расширению возможностей получения
начального профессионального образования военнослужащими, проходящими
службу по призыву. В его рамках военнослужащие во время службы по
призыву помимо диплома по военно-учетной специальности образца
Министерства обороны РФ смогут также получить дипломы государственного
образца о начальном профессиональном образовании. Кроме того, в рамках
данного направления в 2007 году за счет средств федерального бюджета
будет организовано обучение на подготовительных отделениях вузов 5
тыс. человек, отслуживших не менее трех лет по контракту в Вооруженных
Силах РФ в воинских должностях солдат, матросов, сержантов, старшин.

\emph{Государственная поддержка субъектов РФ, внедряющих комплексные
  проекты модернизации образования}. В 2007 году на конкурсной основе
отобран 21 субъект РФ, который в 2007--2009 годах получит субсидии из
федерального бюджета. Эти субсидии будут предназначены для введения
новой системы оплаты труда работников образования, целями которой
являются повышение доходов учителей, переход на нормативно-подушевое
финансирование общеобразовательных учреждений, развитие региональной
системы оценки качества образования, развитие сети общеобразовательных
учреждений регионов и расширение общественного участия в управлении
образованием.

\subsection{Финансирование на фоне экономического кризиса}

В текущем году на реализацию приоритетного национального проекта
<<Образование>>, не смотря на экономический кризис, планируется
израсходовать 33,197 млрд руб. федеральных средств. Эти данные
приведены в отчете Федерального агентства по образованию РФ о
расходовании средств, выделенных на реализацию нацпроекта, в 1
квартале 2009 года.

В частности, на дополнительное вознаграждение за классное руководство
в федеральном бюджете зарезервировано 11,181 млрд руб. Из них 7,453
млрд уже перечислено в регионы. На поощрение педагогов~--- победителей
конкурса лучших учителей будет направлен 1 млрд руб. Поддержка
талантливой молодежи в 2009 году обойдется государству в 200 млн руб.

На совершенствование организации школьного питания пилотные регионы
получат в общей сложности 1 млрд руб., на внедрение комплексных
проектов модернизации образования~--- 5,25 млрд руб. На обеспечение
доступа образовательных учреждений к сети Интернет будет израсходовано
1,1 млрд руб.

На организацию обучения на подготовительных отделениях вузов лиц,
отслуживших не менее трех лет по контракту в Вооруженных Силах РФ, из
федерального бюджета уже направлено 13,588 млн руб. из запланированных
200 млн.

2,27 млрд руб. зарезервировано на создание сети образовательных
учреждений, обеспечивающих подготовку высококвалифицированных рабочих
и специалистов среднего звена. Что касается сферы высшего образования,
то на создание сети национальных исследовательских университетов в
федеральном бюджете предусмотрено 3 млрд руб., на развитие федеральных
университетов~--- 4 млрд. Новые бизнес-школы мирового уровня, созданные
в рамках национального проекта, получат на свое развитие 3,996 млрд
руб.


\subsection{Главные проблемы}

Очевидно, что за относительно короткий срок для сферы образования было
сделано много полезного, то, что раньше не делалось десятилетиями. И
это позитивная сторона Приоритетного Национального проекта
<<Образование>>. А потому проделанную огромную работу надо признать
нужной и полезной для страны. Но значит ли это, что этот проект не
имеет недостатков, которые можно было конструктивно критиковать? Нет,
не значит~\cite{Edu_Result}.

С момента объявления проектов обозначилась три главные проблемы в их
реализации.

Во-первых, Национальные проекты не превратились в научно проработанные
программы. Это произошло, вероятно, из-за нехватки времени, а также
из-за общей практики недооценки ученых, которая сложилась в России
после распада СССР. А потому в Национальных проектах оказалось больше
политики выборов и косвенной пропаганды, чем продуманной социальной
политики развития человеческого потенциала России.

Во-вторых, восприятие национальных проектов в социуме оказалось
противоречивым. Факты таковы, что сама идея Национального проекта, как
способа решения важных проблем страны, представляется плодотворной
только 29\% российских граждан. По мнению 27\% россиян, национальные
проекты не являются эффективным способом решения проблем. Особенно
часто эту точку зрения выражают москвичи (42\%), а также граждане с
высшим образованием (34\%). Граждане правы в том, что проблем у страны
слишком много, и эти проблемы нуждаются в системном, сбалансированном
подходе, чего отдельные проекты, даже если их много, обеспечить не
смогут.

Например, ускоренная компьютеризация школ~--- шаг в правильном
направлении, если бы сами школы не оказались в бедственном
положении. Уже сейчас из 64 тысяч с небольшим российских школ 27 тысяч
непригодны для занятий. Спросите, что сейчас важнее для этих школ~---
отремонтированная крыша или компьютер? Степень износа основных фондов
учебных заведений~--- более 37\%. А коэффициент их обновления примерно
1,2\% в год. Если двигаться в таком темпе, то буквально через 10 лет
мы потеряем вообще всю материально-техническую базу учебных заведений.

Сказанное означает, что компьютером из Национального проекта крышу не
залатаешь. Нужна государственная программа, которая решала бы одну
задачу в связке с другой. Кроме того, компьютеризация школ без
внедрения системы электронного образования (e-Learning) не
эффективна. Но об этом в Приоритетном Национальном проекте нет ни
строчки.

Проекты не стали Национальными, в западном смысле этого слова. В
странах, где нация представляет единство общества и государства,
государство служит обществу. У нас сложилось наоборот: общество
фактически оказалось в тисках играющего мускулами государства.

Существенный недостаток нынешнего Приоритетного Национального проекта
<<Образование>>~--- его избирательность. Или, говоря языком
чиновников, точечность. Считается, что лучшие помогут сдвинуть с места
всю образовательную сферу в сторону обновления. И действительно, часть
школ начала работать по-новому. Но что делать с остальными? Закрывать
их, как и больницы, нельзя. Возить детей плохими дорогами в соседние
сёла и в район~--- тоже не выход, к тому же опасно и дорого. В
подлинной образовательной деятельности внимание к слабым и помощь им
должны быть больше, чем к сильным учащимся. Следовательно, надо
заниматься улучшением работы всех тех образовательных учреждений,
которые ещё остались.

Полезно задать и важный вопрос:~--- А улучшилось ли образование в
России в результате осуществления Приоритетного Национального проекта
<<Образование>>? Два года вложения огромных сумм в систему образования
не привели к заметному улучшению образования в целом по
стране. Наоборот, накопилось слишком много фактов, свидетельствующих о
реальном ухудшении качества образования во многих, особенно отдалённых
регионах. Массовое школьное образование постоянно ухудшается. Знания
естественных наук и родного языка массой выпускников школ стало
низким, как никогда. Математика примерно половиной школьников не
усваивается в необходимом объёме, знание русского языка низкое, как
никогда ранее.  80\% негодных учебников~--- это результат многолетнего
опыта безответственного бюрократического руководства сферой
образования. Не случайно по индексу развития человеческого потенциала
Россия уже опустилась до 57-го места, и перспектив на повышение пока
нет.
\newpage

\section{Развитие АПК}

\subsection{О проекте}

\begin{wrapfigure}{r}{0.2\textwidth}
  \begin{center}
    \includegraphics[width=0.2\textwidth]{NPR_APK}
  \end{center}
\end{wrapfigure}

Приоритетный национальный проект <<Развитие АПК>> включает в себя три
направления~\cite{APK_Problems}:

\begin{enumerate}
\item <<Ускоренное развитие животноводства>>;
\item <<Стимулирование развития малых форм хозяйствования>>;
\item <<Обеспечение доступным жильем молодых специалистов на селе>>.
\end{enumerate}

Реализация \emph{первого направления} Национального проекта позволит
повысить рентабельность животноводства и промышленного рыбоводства,
провести техническое перевооружение действующих животноводческих
комплексов (ферм), предприятий промышленного рыбоводства
(аквакультуры) и ввести в эксплуатацию новые мощности.

Это станет возможным за счет:

\begin{itemize}
\item повышения доступности долгосрочных кредитов, привлекаемых на
  срок до 5 и 8 лет;
\item роста поставок по системе федерального лизинга племенного скота,
  техники и оборудования для животноводства и промышленного
  рыбоводства благодаря увеличению уставного капитала ОАО
  <<Росагролизинг>>, снижению ставки за использование средств
  уставного капитала ОАО <<Росагролизинг>> и продлению срока лизинга
  техники и оборудования для животноводческих комплексов и предприятий
  промышленного рыбоводства (аквакультуры) до 10 лет;
\item совершенствования мер таможенно-тарифного регулирования путем \linebreak
  утверждения объемов квот и таможенных пошлин на мясо в 2006-2007
  годах и вплоть до 2009 года и отмены ввозных таможенных пошлин на
  технологическое оборудование для животноводства, не имеющее
  отечественных аналогов.
\end{itemize}

В октябре 2006 года решением Совета при Президенте РФ по реализации
приоритетных национальных проектов в направление <<Ускоренное развитие
животноводства>> включено пять дополнительных мероприятий.

В их числе: государственная поддержка племенного животноводства,
отечественного овцеводства, северного оленеводства и табунного
коневодства, развитие промышленного рыбоводства (аквакультуры), а
также дополнение в связи с этим финансовой составляющей нацпроекта
субсидированием 5-летних кредитов на развитие животноводства и
аквакультуры.

\emph{Второе направление} Национального проекта направлено на
увеличение объема реализации продукции, произведенной крестьянскими
(фермерскими) хозяйствами и гражданами, ведущими личное подсобное
хозяйство.

Это предполагается достичь путем:

\begin{itemize}
\item удешевления кредитных ресурсов, привлекаемых малыми формами
  хозяйствования АПК;
\item развития инфраструктуры обслуживания малых форм хозяйствования в
  АПК~--- сети сельскохозяйственных потребительских кооперативов
  (заготовительных, снабженческо-сбытовых, перерабатывающих,
  кредит\-ных).
\item развития системы земельно-ипотечного кредитования, что позволит
  выдавать кредиты под залог земельных участков и поможет решить
  проблему отсутствия залоговой базы для малых форм хозяйствования в
  АПК.
\end{itemize}

Реализация третьего направления позволит обеспечить доступным жильем
молодых специалистов на селе, создаст условия для формирования
эффективного кадрового потенциала агропромышленного комплекса.

\subsection{Цели}

По направлению <<Ускоренное развитие
животноводства>>~\cite{APK_Goals}:

\begin{itemize}
\item увеличение производства мяса на 7\%, молока на 4,5\% при
  стабилизации поголовья крупного рогатого скота (КРС), в том числе
  коров, не ниже уровня 2005~г.;
\item увеличение реализации молодняка племенных животных на 15\% к
  уровню 2006~г.
\item увеличение численности поголовья овец и коз на 3\% к уровню 2005
  года.
\item увеличение численности поголовья к уровню 2006~г.: оленей на
  3,2\%, лошадей на 2,8\% к уровню 2005 года.
\item увеличение выпуска товарной продукции аквакультуры на 4,0\% к
  уровню 2005~г.
\end{itemize}


По направлению <<Стимулирование развития малых форм хозяйствования>>:

\begin{itemize}
\item увеличение объемов реализации продукции, произведенной ЛПХ и КФХ
  к 2008 году на 6\% (относительно уровня 2005 года);
\item развитие сети сельскохозяйственных потребительских кооперативов \linebreak
  (снаб\-жен\-чес\-ко-сбытовых, заготовительных, перерабатывающих, кредитных
  кооперативов);
\item создание системы земельно-ипотечного кредитования;
\end{itemize}

По направлению <<Обеспечение жильем молодых специалистов на селе>>:

\begin{itemize}
\item строительство (приобретение) 1392,9 тыс. кв. м жилья и улучшение
  жилищных условий не менее 31,64 тыс. молодых специалистов на селе.
\end{itemize}


\subsection{Мероприятия}

15 мая 2009 года на совещании в Кирове президент затронул тему агарного
национального проекта и те проблемы, с которыми он сегодня
сталкивается. Президент выразил обеспокоенность снижением поголовья
крупного рогатого скота и предложил вернуться к вопросу кредитования
крупных сельскохозяйственных проектов.

<<Снижение поголовья крупного рогатого скота – это очень тревожный
симптом>>,~--- подчеркнул Дмитрий Медведев. Он напомнил, что <<мы
последние несколько лет потратили в неимоверных усилиях, пытаясь
выправить ту ситуацию, которая складывалась в 90-е годы и в первой
половине этого десятилетия, когда у нас поголовье чудовищно падало>>.

<<Все вы, здесь присутствующие, развивали собственные объекты
животноводства, получали кредиты, закупали скот, в том числе по
лизингу. И, конечно, это сейчас потерять было бы самой опасной
ошибкой. Поэтому я считаю, что нам нужно вернуться к вопросу
кредитования крупных сельскохозяйственных проектов, в том числе и в
рамках тех субсидированных кредитов, которые мы давали для таких
проектов>>,~--- сказал он.

Говоря об усилиях руководства субъектов РФ по поддержанию поголовья
крупного рогатого скота, Дмитрий Медведев заметил также, что
Правительство должно в максимальной степени помогать губернаторам в
этом вопросе.

Основные мероприятия Национального проекта в
предусматривают~\cite{APK_Waitings}:

\emph{По направлению <<Ускоренное развитие животноводства>>:}

\begin{itemize}
\item создание условий для привлечения инвестиционных ресурсов,
  необходимых для развития животноводства и промышленного рыбоводства
  за счет выделения дополнительных бюджетных средств на субсидирование
  части затрат на уплату процентов по кредитам на срок до 8 лет,
  направленным на строительство и модернизацию животноводческих
  комплексов и предприятий промышленного рыбоводства, а также по
  кредитам на срок до 5 лет на приобретение племенного скота,
  племенного материала рыб, техники и оборудования для
  животноводческих комплексов (ферм) и предприятий промышленного
  рыбоводства. 
\item развитие лизинга племенного скота, оборудования для
  животноводства и промышленного рыбоводства;
\item повышение эффективности таможенно-тарифной политики;
\item субсидирование расходов на поддержку племенного животноводства,
  а также северного оленеводства, табунного коневодства и овцеводства.
\end{itemize}

\emph{По направлению <<Стимулирование развития малых форм
  хозяйствования>>}:

\begin{itemize}
\item создание условий для привлечения малыми формами хозяйствования в
  АПК кредитов и займов для улучшения их материально-технической базы
  за счет субсидирования части затрат на уплату процентов по кредитам
  и займам, полученным ЛПХ, КФХ и создаваемыми ими
  сельскохозяйственными потребительскими кооперативами в российских
  кредитных организациях и сельскохозяйственных кредитных
  потребительских кооперативах;
\item развитие сети сельскохозяйственных потребительских кооперативов,
  в том числе: кредитных, перерабатывающих, заготовительных и
  снаб\-жен\-чес\-ко-сбытовых;
\item развитие (создание) системы земельно-ипотечного кредитования,
  выдача кредитов под залог земельных участков.
\end{itemize}

\emph{По направлению <<Обеспечение жильем молодых специалистов (или их
  семей) на селе>>}:

\begin{itemize}
\item строительство (приобретение) жилья для молодых специалистов на
  селе.
\end{itemize}
\newpage

\noheadingtag
\section{Заключение}

Финансирование приоритетных национальных проектов, несмотря на
экономический кризис, осталось одним из приоритетов при подготовке
новой редакции федерального бюджета.

Об этом сообщил Председатель Правительства Российской Федерации
Владимир Путин, открывая в среду 25 февраля очередное заседание
президиума Совета при Президенте РФ по реализации приоритетных
национальных проектов и демографической политике.

По мнению премьер-министра, этому принципу должны следовать и
региональные власти при внесении изменений в региональные и
муниципальные бюджеты.

<<Это необходимо сделать потому, что нацпроекты показали свою
востребованность, экономическую и социальную эффективность, –
подчеркнул Владимир Путин.~--- Было бы ошибкой растратить сделанные
заделы>>.

Глава Правительства привел данные, что в результате реализации
нацпроектов ожидаемая продолжительность жизни в России увеличилась
почти на три года. В 2008 году родилось на 260 тыс. детей больше, чем
тремя годами ранее, это самый высокий показатель с 1992 года.

В 2009 году не только будет продолжена работа по ранее намеченным
направлениям национальных проектов, но появится ряд новых, сообщил
премьер. Важнейшее из них~--- продвижение ценностей здорового образа
жизни, популяризация занятий физической культурой и спортом среди
детей и подростков, профилактика курения и алкоголизма в молодежной
среде.

Анализ структуры и хода реализации национальных проектов позволяет
сделать несколько принципиальных выводов~\cite{Spero}.

\begin{enumerate}
\item Направления, выбранные для национальных проектов, действительно
  актуальны и поэтому требуют концентрации ресурсов государства и
  общества. Единственным крупным упущением является отсутствие в
  списке этих направлений культуры.

\item Выбранная в качестве первоначальной форма управленческого
  решения~--- национальный проект~--- так и не была обличена в
  эффективную форму, которой потенциально является, например,
  федеральная целевая программа (ФЦП). Тем самым была нарушена
  классическая цепочка управления, ориентированного на достижение
  конкретного результата: целеполагание $\rightarrow$ соотнесение с
  имеющимися ресурсами $\rightarrow$ выстраивание этапов достижения
  цели и распределение под них имеющихся ресурсов $\rightarrow$
  создание системы мониторинга и контроля $\rightarrow$ накопление
  сигналов обратной связи $\rightarrow$ корректировка (при
  необходимости) этапов движения к цели или (в неблагоприятном
  варианте) первоначальной цели.

  \textbf{P.S}. Приоритетные национальные проекты будут переформатированы в
  государственные программы, реализация которых начнется с 2009
  года. Об этом сообщил в среду 24 сентября заместитель Председателя
  Правительства Российской Федерации Александр Жуков. <<Мы решили
  сохранить в госпрограммах развитие здравоохранения, образования,
  жилищного строительства, сельского хозяйства, все оправдывающие себя
  эффективные направления, по которым развиваются нацпроекты>>,~---
  заявил вице-премьер по итогам заседания президиума Совета при
  Президенте России по реализации приоритетных национальных проектов и
  демографической политике. По его словам, госпрограммы будут
  подготовлены в течение ближайшего месяца и вынесены на рассмотрение
  на следующем заседании Совета по нацпроектам. Александр Жуков
  подчеркнул, что необходимые на реализацию новых госпрограмм средства
  уже заложены в бюджете. <<Процесс не прекращается. Нацпроекты
  принимают иную форму~--- вид госпрограмм»>>,~--- сказал
  вице-премьер.

\item Отсутствие комплексной управленческой формы привело к
  разрозненности конкретных мероприятий, многие из которых давно
  назрели, но реализуемые вне единого замысла зачастую не давали (не
  дают) ожидаемого положительного эффекта.

\item Во многом конъюнктурная природа национальных проектов
  подтверждается тем, что в принятом федеральном бюджете на 2008--2010
  гг. вместо них по текущим расходным статьям рассыпано финансирование
  мероприятий этих проектов. При этом к 2010~г. значительная часть
  этих мероприятий уже не финансируется. Иными словами, и через три
  года Россия будет стоять перед лицом давно назревших социальных
  реформ.

\item В первую очередь неутешительный вывод относится к сфере охраны
  здоровья, где до сих пор нет ясности с оптимальной для нынешнего
  состояния страны моделью. Образовательная политика отличается
  <<вузоцентризмом>>, оставляя за бортом кричащие проблемы школы,
  начального и среднего профессионального образования. Остается
  открытым вопрос о масштабах занятости и системе оплаты труда в так
  называемой бюджетной сфере.

\item Судьба национального проекта <<Доступное и комфортное жилье~---
  гражданам России>> во многом будет зависеть от способности
  государства, во-первых, стимулировать массовое жилищное
  строительство с соответствующим снижением стоимости вводимого
  квадратного метра и, во-вторых, эффективно бороться с монополизмом и
  коррупцией в сфере выделения участков под застройку и их
  фактического получения инвестором.

\item В связи со сменой политического цикла~--- избранием новой
  Государственной Думы и нового президента у политически ответственной
  элиты есть шанс подхватить во многом исчерпавшие себя национальные
  проекты с тем, чтобы, переформатировав их, приступить к системному
  решению проблем социальной сферы. Это надо делать совместно с
  образовательным, медицинским и, в целом, экспертным сообществом, с
  профессиональными сообществами учителей, врачей, работников
  культуры, бизнесом. Одним словом, как говорят в России, всем миром
  навалиться.
\end{enumerate}

\begin{thebibliography}{99}
\bibitem{Putin_RG} Это курс на инвестиции в человека, а значит, и в
  будущее России // Российская газета~--- Центральный выпуск~---
  2005. — \No3867.~--- URL:
  \url{http://www.rg.ru/2005/09/07/prezident-poslanie.html} (дата
  обращения: 28.09.2009).

\bibitem{NPR_Idea} Инвестиции в человека //
  Официальный сайт Совета при Президенте России по реализации
  приоритетных национальных проектов и демографической политике. Дата
  обновления: 16.03.2006. URL:
  \url{http://www.rost.ru/main/what/01/01.shtml}(дата обращения:
  28.09.2009).

\bibitem{Putin_2000} Послание Федеральному Собранию Российской
  Федерации // Президент России. Дата обновления:
  08.07.2000.  URL:
  \url{http://archive.kremlin.ru/appears/2000/07/08/0000_type63372type63374type82634_28782.shtml}
  (дата обращения: 28.09.2009).

\bibitem{Putin_2001}Послание Президента Российской Федерации
  В.\,В.~Путина Федеральному Собранию Российской Федерации
  // Интернет-портал Правительства Российской
  Федерации. 2001.  URL:
  \url{http://www.government.ru/content/31051180-3d4a-416a-a145-4c754f12f38b.htm}
  (дата обращения: 28.09.2009).

\bibitem{Putin_2002} Послание Федеральному Собранию Российской
  Федерации // Президент России. Дата обновления:
  18.04.2002. URL:
  \url{http://archive.kremlin.ru/appears/2002/04/18/0001_type63372type63374type82634_28876.shtml}
  (дата обращения: 28.09.2009).

\bibitem{Putin_2003} Послание Федеральному Собранию Российской
  Федерации // Президент России. Дата обновления:
  16.05.2003. URL:
  \url{http://archive.kremlin.ru/text/appears/2003/05/44623.shtml}
  (дата обращения: 28.09.2009).

\bibitem{Putin_2004} Послание Федеральному Собранию Российской
  Федерации // Президент России. Дата обновления:
  26.05.2004. URL:
  \url{http://archive.kremlin.ru/text/appears/2004/05/71501.shtml}
  (дата обращения: 28.09.2009).

\bibitem{Putin_2005} Послание Федеральному Собранию Российской
  Федерации // Президент России. 2005. Дата
  обновления: 25.04.2005. URL:
  \url{http://archive.kremlin.ru/text/appears/2005/04/87049.shtml}
  (дата обращения: 28.09.2009).

\bibitem{Putin_Gov} Выступление на встрече с членами Правительства,
  руководством Федерального Собрания и членами президиума
  Государственного совета 5 сентября 2005 года //
  Президент России. Дата обновления: 05.09.2005. URL:
  \url{http://archive.kremlin.ru/appears/2005/09/05/1531_type63374type63378type82634_93296.shtml}
  (дата обращения: 28.09.2009).

\bibitem{Health_Problems} Современное здравоохранение: о проекте
  [Электронный ресурс]: Основные проблемы // Приоритетный национальный
  проект <<Здоровье>>. Дата обновления: 07.08.2007. URL:
  \url{http://www.rost.ru/projects/health/p01/p12/a12.shtml} (дата
  обращения: \today).

\bibitem{Health_Goals} Современное здравоохранение: о проекте
  [Электронный ресурс]: Цели и задачи проекта // Приоритетный
  национальный проект <<Здоровье>>. Дата обновления: 07.08.2007. URL:
  \url{http://www.rost.ru/projects/health/p01/p13/a13.shtml} (дата
  обращения: 05.10.2009).

\bibitem{Health_Waitings} Современное здравоохранение: о проекте
  [Электронный ресурс]: Ожидания от проекта // Приоритетный
  национальный проект <<Здоровье>>. Дата обновления: 07.08.2007. URL:
  \url{http://www.rost.ru/projects/health/p01/p16/a16.shtml} (дата
  обращения: 05.10.2009).

\bibitem{Hub_Problems} Доступное и комфортное жилье: о проекте
  [Электронный ресурс]: Проблемы и решения // Приоритетный
  национальный проект <<Доступное и комфортное жилье>>. Дата
  обновления: 13.06.2007. URL:
  \url{http://www.rost.ru/projects/habitation/hab1/h12/ah12.shtml}
  (дата обращения: 05.10.2009).

\bibitem{Hub_Goals} Доступное и комфортное жилье: о проекте
  [Электронный ресурс]: Реализация // Приоритетный национальный проект
  <<Доступное и комфортное жилье>>. Дата обновления: 13.06.2007. URL:
  \url{http://www.rost.ru/projects/habitation/hab1/h14/ah14.shtml}
  (дата обращения: 05.10.2009).

\bibitem{Hub_Waitings} Доступное и комфортное жилье: о проекте
  [Электронный ресурс]: Ожидания // Приоритетный национальный проект
  <<Доступное и комфортное жилье>>. Дата обновления: 13.06.2007. URL:
  \url{http://www.rost.ru/projects/habitation/hab1/h19/ah19.shtml}
  (дата обращения: 05.10.2009).

\bibitem{Edu_Problems} Качественное образование: о проекте
  [Электронный ресурс]: Зачем нам нужен масштабный национальный проект
  в образовании? // Приоритетный национальный проект
  <<Образование>>. Дата обновления: 11.10.2007. URL:
  \url{http://www.rost.ru/projects/education/ed1/ed11/aed11.shtml}
  (дата обращения: 07.10.2009).

\bibitem{Edu_Goals} Качественное образование: о проекте [Электронный
  ресурс]: Основные направления // Приоритетный национальный проект
  <<Образование>>. Дата обновления: 11.10.2007. URL:
  \url{http://www.rost.ru/projects/education/ed1/ed15/aed15.shtml}
  (дата обращения: 07.10.2009).

\bibitem{Edu_Waitings} Качественное образование: о проекте
  [Электронный ресурс]: Ожидаемые результаты // Приоритетный
  национальный проект <<Образование>>. Дата обновления:
  11.10.2007. URL:
  \url{http://www.rost.ru/projects/education/ed1/ed18/aed18.shtml}
  (дата обращения: 07.10.2009).

\bibitem{APK_Problems} Развитие АПК [Электронный ресурс]: О проекте //
  Приоритетный национальный проект <<Развитие агропромышленного
  комплекса>>. Дата обновления: 25.05.2007. URL:
  \url{http://www.rost.ru/projects/agriculture/agr1/agr11/aagr11.shtml}
  (дата обращения: 07.10.2009).

\bibitem{APK_Goals} Развитие АПК [Электронный ресурс]: Цели //
  Приоритетный национальный проект <<Развитие агропромышленного
  комплекса>>. Дата обновления: 25.05.2007. URL:
  \url{http://www.rost.ru/projects/agriculture/agr1/agr12/aagr12.shtml}
  (дата обращения: 07.10.2009).

\bibitem{APK_Waitings} Развитие АПК [Электронный ресурс]: Мероприятия //
  Приоритетный национальный проект <<Развитие агропромышленного
  комплекса>>. Дата обновления: 25.05.2007. URL:
  \url{http://www.rost.ru/projects/agriculture/agr1/agr13/aagr13.shtml}
  (дата обращения: 07.10.2009).
  
\end{thebibliography}


\end{document}

