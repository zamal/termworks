\hfill\parbox{6.5cm}{<<Утверждаю>>\\
  Зав. кафедрой ИБСТ\\
  \hbox to 6.5cm{\hrulefill С.\,Л.\,Зефиров}\\
  \def\hrf#1{\hbox to#1{\hrulefill}}
  <<\hrf{2em}>> \hrf{6em} \the\year~г.}	
	
\begin{center}\textbf{\normalfont\bfseries\large ЗАДАНИЕ}\\\textbf{на
    курсовую работу}\end{center}

\begin{singlespace}
\noindent по теме: \uline{<<Проектирование локальной телекоммуникационной системы библиотеки>>\hfill}\\
1 Дисциплина \uline{\qquad Системы и сети передачи информации\hfill}\\
2 Вариант задания \uline{\qquad 6\hfill}\\
3 Студент \uline{\qquad Захаров М.\,А.\qquad } группа \uline{\qquad 06УИ1\hfill}\\
4 Исходные данные на проектирование\\
4.1 Цель: \uline{создание проекта локальной телекоммуникационной системы библиотеки.\hfill\quad}\\
4.2 Требования к проекту локальной телекоммуникационной системы:
\begin{itemize}
\item \uline{разработка структурной и информационно-логической (инфо\-логической) моделей информационной системы библиотеки, формулировка и описание цели использования сети и её от\-дельных составляющих компонентов; \hfill \quad}
\item \uline{выбор и обоснование размера и структуры сети, выбор кабель\-ной подсистемы сети. Оценка основных характеристик и раз\-меров кабельной подсистемы;\hfill \quad}
\item \uline{организация беспроводного доступа к Интернет с использова\-нием технологии Wi-Fi; \hfill \quad}
\item \uline{выбор сетевого оборудования и описание его основных харак\-теристик и параметров. Присвоение адресов основным эле\-ментам сети. Формирование таблиц маршрутизации или вы\-бор способов маршрутизации; \hfill \quad}
\item \uline{выбор сетевых программных средств и способов сетевого ад\-министрирования; \hfill \quad}
\item \uline{обеспечение информационной безопасности локальной теле\-коммуникационной системы; \hfill \quad}
\item \uline{стоимостная оценка принятых решений. \hfill \quad}
\end{itemize}
5 Структура проекта\\
5.1 Пояснительная записка (содержание работы):
\begin{itemize}
\item \uline{структурная схема информационной системы библиотеки; \hfill \quad}
\item \uline{инфологическая модель библиотеки; \hfill \quad}
\item \uline{физическая модель кабельной системы библиотеки; \hfill \quad}
\item \uline{схема адресации узлов сети; \hfill \quad}
\item \uline{моделирование сети; \hfill \quad}
\item \uline{функции сетевого администратора; \hfill \quad}
\item \uline{стоимостная оценка проекта. \hfill \quad}
\end{itemize}
5.2 Графическая часть
\begin{itemize}
\item \uline{структурная схема информационной системы библиотеки;\hfill}
\item \uline{инфологическая модель библиотеки.\hfill}
\end{itemize}
5.3 Экспериментальная часть
\begin{itemize}
\item \uline{не предусмотрена.\hfill}
\end{itemize}
6 Календарный план выполнения проекта\\
6.1 Сроки выполнения работ по разделам:
\begin{itemize}
\item \uline{выдача задания \hfill} к \uline{06.09.\the\year~г. }
\item \uline{оформление задания \hfill} к \uline{16.09.\the\year~г. }
\item \uline{разработка инфологической модели \hfill} к \uline{27.09.\the\year~г. }
\item \uline{разработка физической модели \hfill} к \uline{11.10.\the\year~г. }
\item \uline{разработка схемы адресации \hfill} к \uline{25.10.\the\year~г. }
\item \uline{стоимостная оценка проекта \hfill} к \uline{08.11.\the\year~г. }
\item \uline{обеспечение ИБ \hfill} к \uline{22.11.\the\year~г. }  
\item \uline{оформление ПЗ \hfill} к \uline{06.12.\the\year~г.}
\end{itemize}
\hbox to 12cm {Дата защиты проекта \uline{\hfill 13 декабря \the\year~г.}}
Руководитель работы \uline{\hfillМали~В.\,А.}\\
\hbox to 12cm {Задание получил \uline{\hfill6 сентября \the\year~г.}}\\
Студент    \uline{\hfillЗахаров~М.\,А.}\\
Нормоконтролёр    \uline{\hfillМали~В.\,А.}\\
\end{singlespace}
\newpage

%%% Local Variables: 
%%% mode: latex
%%% TeX-master: "../SiSPI_TermWork"
%%% End: 
