\newpage

\begin{center}
  \Large{\textbf{ВВЕДЕНИЕ}}
\end{center}
\addcontentsline{toc}{section}{Введение}

В настоящее время практически вся электронная техника включает в себя
такие частотноизбирательные цепи, как электрические фильтры. Расчётные
требования, предъявляемые к электрическим фильтрам, стали достаточно
сложными и строгими [3].

Под синтезом цифрового фильтра понимается построение фильтра с
характеристиками, удовлетворяющими заданным параметрам [2]. В качестве
таких параметров могут фигурировать требования, предъявляемые к
частотным характеристикам фильтра~--- его АЧХ и ФЧХ.

Нерекурсивный фильтр обеспечивает строгую линейность ФЧХ при заданной
форме АЧХ, рекурсивный фильтр обеспечивает лишь заданную форму
АЧХ. Методы проектирования этих фильтров различаются.

Курсовая работа посвящена проектированию нерекурсивных и рекурсивных
цифровых фильтров [1]. Для этого предусмотрен расчёт нерекурсивного
фильтра верхних частот (ФВЧ) и рекурсивного полосового фильтра (ППФ) с
аппроксимацией Чебышева.

Задача синтеза цифровых фильтров включает в себя расчёт относительных
значений характерных частот фильтра, порядка фильтра, оценку АЧХ, а
также построение по результатам расчётов графиков АЧХ, частотной
характеристики затухания и разработка структурной схемы
синтезированного фильтра.

Курсовая работа состоит из 2 разделов и приложения.

В первом разделе рассчитывается нерекурсивный фильтр верхних частот
методом взвешивания: определяется его порядок, оценивается АЧХ,
частотная характеристика затухания. Разработывается структурная схема
синтезированного ФВЧ.

Во втором разделе рассчитывается рекурсивный полосовой фильтр с
аппроксимацией Чебышева методом билинейного преобразования:
определяется его порядок, оценивается АЧХ, частотная характеристика
затухания. Разработывается структурная схема синтезированного ППФ.

В приложении содержится результат расчёта ФВЧ, сведённый в таблицу.

\newpage


%%% Local Variables: 
%%% mode: latex
%%% TeX-master: "../TermWork_TES"
%%% End: 
