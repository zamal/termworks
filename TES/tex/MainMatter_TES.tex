
\section{Расчёт нерекурсивного фильтра}
\label{sec:nerekurs}

Расчёт цифрового фильтра высоких частот (ФВЧ) со строго линейной ФЧХ
выполняется методом взвешивания. Данный метод не позволяет
синтезировать оптимальные фильтры, но гораздо более удобен для
расчётов и даёт вполне приемлемые для практики результаты [1].

Исходные данные:

\begin{itemize}
\item тип фильтра~--- ФВЧ;
\item затухания в полосе задержания $a_0 = 45$ дБ;
\item характерные частоты фильтра $f_1 = 3000$ Гц;
\item ширина переходной полосы $\Delta f = 900$ Гц;
\item частота дискретизации $f_{\text{\textit{д}}} = 8000$ Гц;
\item мощность выходного шума квантования
    $\sigma^2_{\text{\textit{вых}}} = 5 \cdot 10^{-6}$.
\end{itemize}

\subsection{Расчёт относительных значений характерных
  частот фильтра}

Для упрощения обозначений удобно использовать нормализованную шкалу
относительных (или нормированных) частот:

\begin{equation}
  \label{eq:otn_freq}
  f_0 = \frac{f}{f_{\text{д}}},
\end{equation}

\begin{ESKDexplanation}
\item[где ] $f_{\text{д}}$~--- частота дискретизации;
\item $f_0$~--- относительное значение частоты;
\item $f$~--- абсолютное значение частоты.
\end{ESKDexplanation}

Таким образом, по формуле~(\ref{eq:otn_freq}):

\begin{gather*}
  f_{01} = \frac{f_1}{f_{\text{д}}} = \frac{3000}{8000} = 0{,}375;\\
  \Delta f_0 = \frac{\Delta f}{f_{\text{д}}} = \frac{900}{8000} = 0{,}113.
\end{gather*}

\subsection{Определение порядка фильтра}

В соответствии с заданной величиной затухания в полосе задерживания
$a_0 = 45$ и графиками для окна Ланцоша, определяется положительная
постоянная~$L$ (см. формулу~(\ref{eq:lancosh})) и порядок фильтра~$N$:

\begin{gather*}
  (N-1)\Delta f_0 = 3;\\
  N = 27;\\
  L = 1{,}5.
\end{gather*}

\subsection{Определение коэффициентов разложения в ряд Фурье}

Коэффициенты разложения в ряд Фурье идеальной АЧХ ФВЧ:

\begin{gather}
  \nonumber
  h(0) = 1 - 2f_{01};\\
  \label{eq:koef_furier}
  h(k) = - \frac{\sin(2 \pi k f_{01})}{k \pi}.
\end{gather}

\begin{equation*}
  k = - \frac{N}{2}, \ldots , \frac{N}{2} = -13, \ldots, 13.
\end{equation*}

Результаты вычислений занесены в таблицу

\subsection{Вычисление весовых множителей}

Для уменьшения амплитуды пульсаций усечение импульсной реакции
производят с использованием весовой последовательности конечной длинны
$w(k)$, называемой временным окном. Весовые множители $w(k)$
вычисляются из следующего выражения:

\begin{gather}
  \label{eq:lancosh}
  w(k) = \left[\frac{\displaystyle\sin\left(\frac{2\pi
          k}{N-1}\right)}{\displaystyle\frac{2\pi k}{N-1}}\right]^L,\\
  \nonumber
  k = -13, \ldots, 13,\\
  \nonumber
  L = 1{,}5.
\end{gather}

Результаты вычислений занесены в таблицу

%%% Local Variables: 
%%% mode: latex
%%% TeX-master: "../TermWork_TES"
%%% End: 
