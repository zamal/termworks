\hfill\parbox{6.5cm}{
	\textit{<<УТВЕРЖДАЮ>>}\\
	Зав. кафедрой ИБСТ\\
	\hbox to 6.5cm{\hrulefill С.\,Л.\,Зефиров}\\
	\def\hrf#1{\hbox to#1{\hrulefill}}
	<<\hrf{2em}>> \hrf{6em} \the\year~г.}	
	
      \begin{center}\textbf{\normalfont\bfseries\large ЗАДАНИЕ}\\\textbf{на
        курсовую работу}\end{center}

%\begin{singlespace}
\noindent по теме: \uline{<<Синтез цифровых фильтров>>\hfill}\\
1 Дисциплина \uline{\qquad Теория электрической связи\hfill}\\
2 Вариант задания \uline{\qquad 7\hfill}\\
3 Студент \uline{\qquad Захаров М.\,А.\qquad } группа \uline{\qquad 06УИ1\hfill}\\
4 Исходные данные на курсовую работу\\
4.1 Рассчитать цифровой фильтр со строго линейной ФЧХ, к АЧХ которого
предъявляются следующие требования:
\begin{itemize}
\item \uline{тип фильтра~--- ФВЧ;\hfill\quad}
\item \uline{затухания в полосе задержания $a_0 = 45$ дБ;\hfill\quad}
\item \uline{характерные частоты фильтра $f_1 = 3000$ Гц;\hfill\quad}
\item \uline{ширина переходной полосы $\Delta f = 900$ Гц;\hfill\quad}
\item \uline{частота дискретизации $f_{\text{\textit{д}}} = 8000$
    Гц;\hfill\quad}
\item \uline{мощность выходного шума квантования
    $\sigma^2_{\text{\textit{вых}}} = 5 \cdot 10^{-6}$.\hfill\quad}
\end{itemize}
4.2 Рассчитать АЧХ синтезированного фильтра.\\
4.3 Рассчитать цифровой фильтр, к ФЧХ которого не предъявляется жестких
требований, а параметры АЧХ являются следующими:
\begin{itemize}
\item \uline{тип фильтра, характер аппроксимации~--- ППФ,
    Чебышева;\hfill \quad}
\item \uline{затухание в полосе задержания $a_0 = 40$ дБ;\hfill \quad}
\item \uline{верхняя граница затухания в полосе пропускания $\Delta a$ 
    $=$ $=$~0{,}8~дБ;\hfill \quad}
\item \uline{характерные частоты фильтра $f_{11} = 900$ Гц, $f_{12} =
    1800$ Гц, $f_{21} = 500$ Гц, $f_{22} = 2200$ Гц;\hfill \quad}
\item \uline{частота дискретизации $f_{\text{\textit{д}}} = 4800$ Гц;\hfill \quad}
\item \uline{мощность выходного шума квантования
    $\sigma^2_{\text{\textit{вых}}} = 10^{-5}$.\hfill\quad}
\end{itemize}
4.4 Рассчитать АЧХ синтезированного фильтра.\\
5 Структура проекта\\
5.1 Пояснительная записка (содержание работы):
\begin{itemize}
\item \uline{расчётно-пояснительная записка объёмом 15--20 страниц
    со\-держит 4 раздела (в соответствии с пунктами
    задания). \hfill \quad}
\end{itemize}
5.2 Графическая часть
\begin{itemize}
\item \uline{4 листа формата А4, на которых представлены схемы
    рас\-считываемых фильтров и их частотные характеристики.\hfill}
\end{itemize}
5.3 Экспериментальная часть
\begin{itemize}
\item \uline{не предусмотрена.\hfill}
\end{itemize}
6 Календарный план выполнения проекта\\
6.1 Сроки выполнения работ по разделам:
\begin{itemize}
\item \uline{раздел первый\hfill}
  к \uline{25.09.\the\year~г. }
\item \uline{раздел второй \hfill} к \uline{20.10.\the\year~г.}
\item \uline{раздел третий \hfill} к \uline{10.11.\the\year~г.}
\item \uline{раздел четвёртый \hfill} к \uline{25.11.\the\year~г.}
\item \uline{оформление пояснительной записки \hfill} к \uline{06.12.\the\year~г.}

\end{itemize}
\hbox to 9.5cm {Дата защиты проекта \uline{\hfill 6 декабря \the\year~г.}}
Руководитель работы \uline{\hfillСултанов~Б.\,В.}\\
\hbox to 9.5cm {Задание получил \uline{\hfill7 сентября \the\year~г.}}\\
Студент    \uline{\hfillЗахаров~М.\,А.}\\
Нормоконтролёр    \uline{\hfillСултанов~Б.\,В.}\\
%\end{singlespace}
\newpage



%%% Local Variables: 
%%% mode: latex
%%% TeX-master: "../TermWork_TES"
%%% End: 
