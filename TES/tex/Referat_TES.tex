\ESKDthisStyle{formII}
\begin{center}
  \Large{\textbf{РЕФЕРАТ}}
\end{center}

Отчёт \ESKDtotal{page}~с., 7~рис., 1~табл., 3~источника,
\ESKDtotal{appendix}~прил.


ЦИФРОВОЙ ФИЛЬТР, НЕРЕКУРСИВНЫЙ ФИЛЬТР, РЕКУРСИВНЫЙ ФИЛЬТР, МЕТОД
ВЗВЕШИВАНИЯ, МЕТОД БИЛИНЕЙНОГО ПРЕОБРАЗОВАНИЯ

Объектом исследования являются являются нерекурсивные и рекурсивные
фильтры и методы из расчёта.

Цель работы~--- расчёт нерекурсивного цифрового фильтра верхних
частот, расчёт рекурсивного цифрового полосового пропускающего
фильтра, расчёт и построение амплитудно-частотной характеристики
и структурных схем фильтров.

В результате исследования были рассчитаны нерекурсивный и рекурсивный
цифровые фильтры, построены их амплитудно-частотные характеристики и
структурные схемы.

\newpage



%%% Local Variables: 
%%% mode: latex
%%% TeX-master: "../TermWork_TES"
%%% End: 
