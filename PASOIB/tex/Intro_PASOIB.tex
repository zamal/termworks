\newpage

\begin{center}
  \Large{\textbf{ВВЕДЕНИЕ}}
\end{center}
\addcontentsline{toc}{section}{Введение}

Целью данного курсового проектирования является разработка программной
защиты приложения от несанкционированного копирования.

Защита интеллектуальных ресурсов от незаконного использования является
на сегодняшний день одним из основных направлений разработки
программного обеспечения. Не существует абсолютно надежных методов
защиты. Можно утверждать, что достаточно квалифицированные системные
программисты, пользующиеся современными средствами анализа работы
программного обеспечения (отладчики, дизассемблеры, перехватчики
прерываний и т. д.), располагающие достаточным временем, смогут
преодолеть практически любую защиту. Поэтому при проектировании
системы защиты следует исходить из предположения, что рано или поздно
эта защита окажется снятой~\cite{1}. Целью проектирования должен быть выбор
такого способа защиты, который обеспечит невозможность
несанкционированного копирования для заранее определенного круга лиц и
в течение ограниченного времени.

В процессе выполнения курсового проекта производится разработка защиты
программы-генератора псевдослучайной последовательности от
несанкционированного копирования, за счет ограничения времени работы
программы. Разработанная защита должна быть исследована с целью
анализа стойкости.

\newpage

%%% Local Variables: 
%%% mode: latex
%%% TeX-master: "../TermWork_PASIOB"
%%% End: 
