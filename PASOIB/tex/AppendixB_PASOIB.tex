\begin{center}
  \Large{\textbf{ПРИЛОЖЕНИЕ Б}}\\
  \normalsize (справочное) \\ \textbf{Листинги разработанных программ}
\end{center}
\addcontentsline{toc}{section}{Приложение Б Листинги разработанных программ}
\label{AppendixB}

\small

\begin{lstlisting}[caption = {ПРОГРАММА УСТАНОВКИ}, label = {4.cpp}]

#pragma once


namespace GPSP_install {

using namespace System;
using namespace System::ComponentModel;
using namespace System::Collections;
using namespace System::Windows::Forms;
using namespace System::Data;
using namespace System::Drawing;
using namespace System::IO;

/// <summary>
/// Сводка для Form1
///
/// Внимание! При изменении имени этого класса необходимо также изменить
///	     свойство имени файла ресурсов ("Resource File Name") для средства компиляции управляемого ресурса,
///	     связанного со всеми файлами с расширением .resx, от которых зависит данный класс. В противном случае,
///	     конструкторы не смогут правильно работать с локализованными
///	     ресурсами, сопоставленными данной форме.
/// </summary>
public ref class Form1 : public System::Windows::Forms::Form
{
public:
Form1(void)
{
	InitializeComponent();
	//
	//TODO: добавьте код конструктора
	//
}

protected:
/// <summary>
/// Освободить все используемые ресурсы.
/// </summary>
~Form1()
{
	if (components)
	{
		delete components;
	}
}
private: System::Windows::Forms::Button^  button1;
protected: 

private:
/// <summary>
/// Требуется переменная конструктора.
/// </summary>
System::ComponentModel::Container ^components;

#pragma region Windows Form Designer generated code
/// <summary>
/// Обязательный метод для поддержки конструктора - не изменяйте
/// содержимое данного метода при помощи редактора кода.
/// </summary>
void InitializeComponent(void)
{
	this->button1 = (gcnew System::Windows::Forms::Button());
	this->SuspendLayout();
	// 
	// button1
	// 
	this->button1->Location = System::Drawing::Point(12, 22);
	this->button1->Name = L"button1";
	this->button1->Size = System::Drawing::Size(75, 23);
	this->button1->TabIndex = 0;
	this->button1->Text = L"Установить";
	this->button1->UseVisualStyleBackColor = true;
	this->button1->Click += gcnew System::EventHandler(this, &Form1::button1_Click);
	// 
	// Form1
	// 
	this->AutoScaleDimensions = System::Drawing::SizeF(6, 13);
	this->AutoScaleMode = System::Windows::Forms::AutoScaleMode::Font;
	this->ClientSize = System::Drawing::Size(198, 58);
	this->Controls->Add(this->button1);
	this->Name = L"Form1";
	this->Text = L"Установка";
	this->ResumeLayout(false);

}
#pragma endregion
private: System::Void button1_Click(System::Object^  sender, System::EventArgs^	 e) {
	 FolderBrowserDialog^ folder_dialog = gcnew FolderBrowserDialog();
	 folder_dialog->ShowDialog();
	 if(folder_dialog->SelectedPath != "")
	 {
		 StreamWriter^ write = gcnew StreamWriter(folder_dialog->SelectedPath + "\\param");
		 DateTime^ dt = DateTime::Now;
		 write->WriteLine(dt->ToLongDateString());
		 write->Close();
		 File::Copy("GPSP.exe", folder_dialog->SelectedPath + "\\GPSP.exe");
		 MessageBox::Show("Установлено");
	 }
 }
};
}
\end{lstlisting}

\begin{lstlisting}[caption = {ОСНОВНАЯ ПОЛЕЗНАЯ ПРОГРАММА}, label = {4.cpp}]
#pragma once
namespace GPSP {

using namespace System;
using namespace System::ComponentModel;
using namespace System::Collections;
using namespace System::Windows::Forms;
using namespace System::Data;
using namespace System::Drawing;
using namespace System::IO;
using namespace Microsoft::Win32;

	

/// <summary>
/// Сводка для Form1
///
/// Внимание! При изменении имени этого класса необходимо также изменить
///	     свойство имени файла ресурсов ("Resource File Name") для средства компиляции управляемого ресурса,
///	     связанного со всеми файлами с расширением .resx, от которых зависит данный класс. В противном случае,
///	     конструкторы не смогут правильно работать с локализованными
///	     ресурсами, сопоставленными данной форме.
/// </summary>
public ref class Form1 : public System::Windows::Forms::Form
{
public:
	Form1(void)
	{
		InitializeComponent();
		//
		//TODO: добавьте код конструктора
		//
	}

protected:
	/// <summary>
	/// Освободить все используемые ресурсы.
	/// </summary>
	~Form1()
	{
		if (components)
		{
			delete components;
		}
	}
private: System::Windows::Forms::Label^	 label1;
protected: 
private: System::Windows::Forms::NumericUpDown^	 numericUpDown1;
private: System::Windows::Forms::Label^	 label2;
private: System::Windows::Forms::Label^	 label3;
private: System::Windows::Forms::NumericUpDown^	 numericUpDown2;
private: System::Windows::Forms::Label^	 label4;
private: System::Windows::Forms::Label^	 label5;
private: System::Windows::Forms::NumericUpDown^	 numericUpDown3;
private: System::Windows::Forms::Label^	 label6;
private: System::Windows::Forms::NumericUpDown^	 numericUpDown4;
private: System::Windows::Forms::Label^	 label8;
private: System::Windows::Forms::Button^  button1;
private: System::Windows::Forms::NumericUpDown^	 numericUpDown5;
private: System::Windows::Forms::Label^	 label7;
private: System::Windows::Forms::GroupBox^  groupBox1;
private: System::Windows::Forms::Button^  button2;
private: System::Windows::Forms::TextBox^  textBox1;
private: System::Windows::Forms::GroupBox^  groupBox2;

private:
	/// <summary>
	/// Требуется переменная конструктора.
	/// </summary>
	System::ComponentModel::Container ^components;

#pragma region Windows Form Designer generated code
/// <summary>
/// Обязательный метод для поддержки конструктора - не изменяйте
/// содержимое данного метода при помощи редактора кода.
/// </summary>
void InitializeComponent(void)
{
this->label1 = (gcnew System::Windows::Forms::Label());
this->numericUpDown1 = (gcnew System::Windows::Forms::NumericUpDown());
this->label2 = (gcnew System::Windows::Forms::Label());
this->label3 = (gcnew System::Windows::Forms::Label());
this->numericUpDown2 = (gcnew System::Windows::Forms::NumericUpDown());
this->label4 = (gcnew System::Windows::Forms::Label());
this->label5 = (gcnew System::Windows::Forms::Label());
this->numericUpDown3 = (gcnew System::Windows::Forms::NumericUpDown());
this->label6 = (gcnew System::Windows::Forms::Label());
this->numericUpDown4 = (gcnew System::Windows::Forms::NumericUpDown());
this->label8 = (gcnew System::Windows::Forms::Label());
this->button1 = (gcnew System::Windows::Forms::Button());
this->numericUpDown5 = (gcnew System::Windows::Forms::NumericUpDown());
this->label7 = (gcnew System::Windows::Forms::Label());
this->groupBox1 = (gcnew System::Windows::Forms::GroupBox());
this->button2 = (gcnew System::Windows::Forms::Button());
this->textBox1 = (gcnew System::Windows::Forms::TextBox());
this->groupBox2 = (gcnew System::Windows::Forms::GroupBox());
(cli::safe_cast<System::ComponentModel::ISupportInitialize^  >(this->numericUpDown1))->BeginInit();
(cli::safe_cast<System::ComponentModel::ISupportInitialize^  >(this->numericUpDown2))->BeginInit();
(cli::safe_cast<System::ComponentModel::ISupportInitialize^  >(this->numericUpDown3))->BeginInit();
(cli::safe_cast<System::ComponentModel::ISupportInitialize^  >(this->numericUpDown4))->BeginInit();
(cli::safe_cast<System::ComponentModel::ISupportInitialize^  >(this->numericUpDown5))->BeginInit();
this->groupBox1->SuspendLayout();
this->groupBox2->SuspendLayout();
this->SuspendLayout();
// 
// label1
// 
this->label1->AutoSize = true;
this->label1->Location = System::Drawing::Point(22, 20);
this->label1->Name = L"label1";
this->label1->Size = System::Drawing::Size(104, 13);
this->label1->TabIndex = 0;
this->label1->Text = L"Выберите полином";
// 
// numericUpDown1
// 
this->numericUpDown1->Location = System::Drawing::Point(25, 54);
this->numericUpDown1->Maximum = System::Decimal(gcnew cli::array< System::Int32 >(4) {31, 0, 0, 0});
this->numericUpDown1->Minimum = System::Decimal(gcnew cli::array< System::Int32 >(4) {4, 0, 0, 0});
this->numericUpDown1->Name = L"numericUpDown1";
this->numericUpDown1->Size = System::Drawing::Size(33, 20);
this->numericUpDown1->TabIndex = 1;
this->numericUpDown1->Value = System::Decimal(gcnew cli::array< System::Int32 >(4) {4, 0, 0, 0});
// 
// label2
// 
this->label2->AutoSize = true;
this->label2->Location = System::Drawing::Point(5, 61);
this->label2->Name = L"label2";
this->label2->Size = System::Drawing::Size(14, 13);
this->label2->TabIndex = 2;
this->label2->Text = L"Х";
// 
// label3
// 
this->label3->AutoSize = true;
this->label3->Location = System::Drawing::Point(65, 56);
this->label3->Name = L"label3";
this->label3->Size = System::Drawing::Size(13, 13);
this->label3->TabIndex = 3;
this->label3->Text = L"+";
// 
// numericUpDown2
// 
this->numericUpDown2->Location = System::Drawing::Point(104, 54);
this->numericUpDown2->Maximum = System::Decimal(gcnew cli::array< System::Int32 >(4) {31, 0, 0, 0});
this->numericUpDown2->Minimum = System::Decimal(gcnew cli::array< System::Int32 >(4) {3, 0, 0, 0});
this->numericUpDown2->Name = L"numericUpDown2";
this->numericUpDown2->Size = System::Drawing::Size(30, 20);
this->numericUpDown2->TabIndex = 4;
this->numericUpDown2->Value = System::Decimal(gcnew cli::array< System::Int32 >(4) {3, 0, 0, 0});
// 
// label4
// 
this->label4->AutoSize = true;
this->label4->Location = System::Drawing::Point(84, 61);
this->label4->Name = L"label4";
this->label4->Size = System::Drawing::Size(14, 13);
this->label4->TabIndex = 5;
this->label4->Text = L"Х";
// 
// label5
// 
this->label5->AutoSize = true;
this->label5->Location = System::Drawing::Point(159, 61);
this->label5->Name = L"label5";
this->label5->Size = System::Drawing::Size(14, 13);
this->label5->TabIndex = 8;
this->label5->Text = L"Х";
// 
// numericUpDown3
// 
this->numericUpDown3->Location = System::Drawing::Point(179, 54);
this->numericUpDown3->Maximum = System::Decimal(gcnew cli::array< System::Int32 >(4) {31, 0, 0, 0});
this->numericUpDown3->Minimum = System::Decimal(gcnew cli::array< System::Int32 >(4) {2, 0, 0, 0});
this->numericUpDown3->Name = L"numericUpDown3";
this->numericUpDown3->Size = System::Drawing::Size(30, 20);
this->numericUpDown3->TabIndex = 7;
this->numericUpDown3->Value = System::Decimal(gcnew cli::array< System::Int32 >(4) {2, 0, 0, 0});
// 
// label6
// 
this->label6->AutoSize = true;
this->label6->Location = System::Drawing::Point(140, 56);
this->label6->Name = L"label6";
this->label6->Size = System::Drawing::Size(13, 13);
this->label6->TabIndex = 6;
this->label6->Text = L"+";
// 
// numericUpDown4
// 
this->numericUpDown4->Location = System::Drawing::Point(242, 54);
this->numericUpDown4->Maximum = System::Decimal(gcnew cli::array< System::Int32 >(4) {1, 0, 0, 0});
this->numericUpDown4->Name = L"numericUpDown4";
this->numericUpDown4->Size = System::Drawing::Size(30, 20);
this->numericUpDown4->TabIndex = 10;
// 
// label8
// 
this->label8->AutoSize = true;
this->label8->Location = System::Drawing::Point(223, 56);
this->label8->Name = L"label8";
this->label8->Size = System::Drawing::Size(13, 13);
this->label8->TabIndex = 9;
this->label8->Text = L"+";
// 
// button1
// 
this->button1->Location = System::Drawing::Point(3, 80);
this->button1->Name = L"button1";
this->button1->Size = System::Drawing::Size(131, 23);
this->button1->TabIndex = 11;
this->button1->Text = L"Сгенерировать ПСП";
this->button1->UseVisualStyleBackColor = true;
this->button1->Click += gcnew System::EventHandler(this, &Form1::button1_Click);
// 
// numericUpDown5
// 
this->numericUpDown5->Location = System::Drawing::Point(310, 82);
this->numericUpDown5->Maximum = System::Decimal(gcnew cli::array< System::Int32 >(4) {1000, 0, 0, 0});
this->numericUpDown5->Minimum = System::Decimal(gcnew cli::array< System::Int32 >(4) {32, 0, 0, 0});
this->numericUpDown5->Name = L"numericUpDown5";
this->numericUpDown5->Size = System::Drawing::Size(62, 20);
this->numericUpDown5->TabIndex = 12;
this->numericUpDown5->Value = System::Decimal(gcnew cli::array< System::Int32 >(4) {32, 0, 0, 0});
// 
// label7
// 
this->label7->AutoSize = true;
this->label7->Location = System::Drawing::Point(159, 89);
this->label7->Name = L"label7";
this->label7->Size = System::Drawing::Size(138, 13);
this->label7->TabIndex = 13;
this->label7->Text = L"длина генерируемой ПСП";
// 
// groupBox1
// 
this->groupBox1->Controls->Add(this->button2);
this->groupBox1->Controls->Add(this->textBox1);
this->groupBox1->Location = System::Drawing::Point(12, 127);
this->groupBox1->Name = L"groupBox1";
this->groupBox1->Size = System::Drawing::Size(400, 50);
this->groupBox1->TabIndex = 14;
this->groupBox1->TabStop = false;
this->groupBox1->Text = L"Регистрация";
// 
// button2
// 
this->button2->Location = System::Drawing::Point(218, 16);
this->button2->Name = L"button2";
this->button2->Size = System::Drawing::Size(117, 23);
this->button2->TabIndex = 1;
this->button2->Text = L"Зарегистрировать";
this->button2->UseVisualStyleBackColor = true;
this->button2->Click += gcnew System::EventHandler(this, &Form1::button2_Click);
// 
// textBox1
// 
this->textBox1->Location = System::Drawing::Point(7, 20);
this->textBox1->Name = L"textBox1";
this->textBox1->Size = System::Drawing::Size(183, 20);
this->textBox1->TabIndex = 0;
// 
// groupBox2
// 
this->groupBox2->Controls->Add(this->label1);
this->groupBox2->Controls->Add(this->numericUpDown1);
this->groupBox2->Controls->Add(this->label7);
this->groupBox2->Controls->Add(this->label2);
this->groupBox2->Controls->Add(this->numericUpDown5);
this->groupBox2->Controls->Add(this->label3);
this->groupBox2->Controls->Add(this->button1);
this->groupBox2->Controls->Add(this->numericUpDown2);
this->groupBox2->Controls->Add(this->numericUpDown4);
this->groupBox2->Controls->Add(this->label4);
this->groupBox2->Controls->Add(this->label8);
this->groupBox2->Controls->Add(this->label6);
this->groupBox2->Controls->Add(this->label5);
this->groupBox2->Controls->Add(this->numericUpDown3);
this->groupBox2->Location = System::Drawing::Point(12, 7);
this->groupBox2->Name = L"groupBox2";
this->groupBox2->Size = System::Drawing::Size(400, 114);
this->groupBox2->TabIndex = 15;
this->groupBox2->TabStop = false;
// 
// Form1
// 
this->AutoScaleDimensions = System::Drawing::SizeF(6, 13);
this->AutoScaleMode = System::Windows::Forms::AutoScaleMode::Font;
this->ClientSize = System::Drawing::Size(419, 188);
this->Controls->Add(this->groupBox2);
this->Controls->Add(this->groupBox1);
this->Name = L"Form1";
this->Text = L"Генератор псведослучайной последовательности";
this->Load += gcnew System::EventHandler(this, &Form1::Form1_Load);
(cli::safe_cast<System::ComponentModel::ISupportInitialize^  >(this->numericUpDown1))->EndInit();
(cli::safe_cast<System::ComponentModel::ISupportInitialize^  >(this->numericUpDown2))->EndInit();
(cli::safe_cast<System::ComponentModel::ISupportInitialize^  >(this->numericUpDown3))->EndInit();
(cli::safe_cast<System::ComponentModel::ISupportInitialize^  >(this->numericUpDown4))->EndInit();
(cli::safe_cast<System::ComponentModel::ISupportInitialize^  >(this->numericUpDown5))->EndInit();
this->groupBox1->ResumeLayout(false);
this->groupBox1->PerformLayout();
this->groupBox2->ResumeLayout(false);
this->groupBox2->PerformLayout();
this->ResumeLayout(false);

}
#pragma endregion

	unsigned long line;

private: System::Void Form1_Load(System::Object^  sender, System::EventArgs^  e) {
 if(!File::Exists("param"))
 {
	 MessageBox::Show("Ошибка");
	 Application::Exit();
 }
 bool reg = true;
 RegistryKey^ rk = nullptr;
 rk = Registry::CurrentUser->OpenSubKey("GPSP",true);
 if (rk==nullptr)
 {
	reg = false;
 }
 else
 {
	 String^ value = rk->GetValue("line")->ToString();
	 if(value!="forza") reg = false; 
 }
 DateTime^ dt = DateTime::Now;
 if(!reg)
 {
	 
	 StreamReader^ read_date = gcnew StreamReader("param");
	 DateTime^ start = Convert::ToDateTime(read_date->ReadLine()); 
	 if(DateTime::Compare(start->AddDays(5), dt->Date)<0)
	 {
		 MessageBox::Show("Срок использования незарегистрированной версии истек. Хотите зарегистрировать программу");
		 this->groupBox2->Hide();
	 }
	 else this->groupBox1->Hide();
 }
 else this->groupBox1->Hide();
	
 Random^ rnd = gcnew Random(dt->ToBinary());
 line = line & 0x00;
 unsigned long cur_bit = 0;
 for(int i = 0; i < 31; i++)
 {
	 cur_bit = rnd->Next()%2;
	 line = (line << 1) ^ (cur_bit & 0x01);
 }	
			
			 
		 }
private: System::Void button1_Click(System::Object^  sender, System::EventArgs^  e) {
int first = Convert::ToInt32(this->numericUpDown1->Value);
int second = Convert::ToInt32(this->numericUpDown2->Value);
int third = Convert::ToInt32(this->numericUpDown3->Value);
int free = Convert::ToInt32(this->numericUpDown4->Value);
int length = Convert::ToInt32(this->numericUpDown5->Value);
unsigned long in_file = 0;
Stream^ str = File::Open("PSP", FileMode::OpenOrCreate, FileAccess::Write);
BinaryWriter^ bwr = gcnew BinaryWriter(str);
for(int i = 0; i < length; i++)
{
	for(int j = 0; j < 31; j++)
	{
		line = (line << 1) ^ ((((line >> (first-1)) & 0x01)^((line >> (second-1)) & 0x01)^((line >> (third-1)) & 0x01)^(free & 0x01))&0x01);
					in_file = (in_file << 1) ^ (line & 0x01);
				}
				bwr->Write((int)in_file);
			}
			bwr->Close();
			str->Close();
			MessageBox::Show("Сгенерировано");
		 }
private: System::Void button2_Click(System::Object^  sender, System::EventArgs^  e) {
	 int pass = this->textBox1->Text->GetHashCode();
	 int right_pass = 1364505728; 
	 if(pass==right_pass)
	 {
		  RegistryKey^ rk = nullptr;
		  rk = Registry::CurrentUser->OpenSubKey("GPSP",true);
		  if (rk==nullptr)
		  {
				 RegistryKey^ rk = Registry::CurrentUser->CreateSubKey("GPSP");
				 rk->SetValue("line","forza");
				 rk->Close();
		  }					
		  else
		  {
				 Registry::CurrentUser->DeleteSubKey("GPSP");
				 RegistryKey^ rk = Registry::CurrentUser->CreateSubKey("GPSP");
				 rk->SetValue("line","forza");
				 rk->Close();
		  }
		  this->groupBox1->Hide();
		  this->groupBox2->Show();
		  MessageBox::Show("Зарегистрировано");
	 }
	 else
	 {
		 MessageBox::Show("Неправильный пароль");
	 }

 }
};
\end{lstlisting}


%%% Local Variables: 
%%% mode: latex
%%% TeX-master: "../TermWork_PASIOB"
%%% End: 
