\section{Общие положения}

\point Данный документ был создан в целях проверки шаблона для
оформления курсовых работ.

\point Нумерованное перечисление:

\begin{enumerate}
\item первый элемент состоит из:
\begin{enumerate}
\item первого подэлемента;
\item второго подэлемента;
\end{enumerate}
\item второй элемент.
\end{enumerate}

\point Знак сноски выполняют арабскими цифрами со скобкой и помещают
на уровне верхнего обреза шрифта. Пример "--- <<$\ldots$ печатающее
устройство\footnote{текст сноски}$\ldots$>>.

\subsection{Иллюстрации, таблицы}

Пример иллюстрации изображен на рисунке~\ref{fig:1}.

\begin{figure}[h!]
\begin{center}
\setlength{\unitlength}{50mm}
\begin{picture}(1,1)
\linethickness{\ESKDlineThin}
\put(0,0){\line(1,0){1}}
\put(1,0){\line(0,1){1}}
\put(0,1){\line(1,0){1}}
\put(0,0){\line(0,1){1}}
\put(0,0){\line(1,1){1}}
\put(1,0){\line(-1,1){1}}
\end{picture}
\end{center}
\caption{Перечеркнутый квадрат}
\label{fig:1}
\end{figure}

В таблице~\ref{t:second} приведен пример игры в крестики нолики.

\begin{table}[h!]
\caption{Вторая таблица}
\label{t:second}
\begin{tabular}{|c|c|c|}
\hline
o& &x\\\hline
 &x& \\\hline
x& &o\\\hline
\end{tabular}
\end{table}

\subsection{Формулы}

Плотность каждого образца $\rho, \text{кг}/\text{м}^3$,
вычисляют по формуле:

\begin{equation}
\label{eq:1}
\rho = \frac{m}{V},
\end{equation}
\begin{ESKDexplanation}
\item[где ] $m$ "--- масса образца, кг;
\item $V$ "--- объем образца, $\text{м}^3$.
\end{ESKDexplanation}


%%% Local Variables: 
%%% mode: latex
%%% TeX-master: "../TermWork"
%%% End: 
